%\documentclass[12pt, a4paper,addpoints]{exam}
\documentclass[12pt, a4paper]{article}

% Languages and fonts
\usepackage{cmap} 
\usepackage[T2A]{fontenc}
\usepackage[utf8]{inputenc} 
\usepackage[english, russian]{babel}
\usepackage{microtype}
\usepackage{indentfirst}
\frenchspacing

% Mathematics
\usepackage{amsmath, amssymb, amsfonts, amsthm, mathtools, fixmath}
\usepackage{esint, esvect} % integrals and vectors
\usepackage{systeme} % equation system
\usepackage{commath} % partials and differentials
\usepackage{icomma} % smart comma ($0,2$ is a number)
\usepackage{mathabx}% astronomy

% Floats
\usepackage{float}

% Tables
\usepackage{array,tabularx,tabulary,booktabs} % better tables
\usepackage{longtable}
\usepackage{multirow}

% Graphics
\usepackage[pdftex]{graphicx}
\usepackage{wrapfig}

% Theorems
\renewcommand{\proofname}{Доказательство}

%\theoremstyle{plain}
\newtheorem{theorem}{Теорема}[section]

%\theoremstyle{definition}
\newtheorem{definition}{Определение}
\newtheorem{corollary}{Следствие}[theorem]

\theoremstyle{remark}
\newtheorem{remark}{Замечание}

\usepackage[top=20mm,bottom=20mm,left=20mm,right=20mm]{geometry}

\usepackage{soul}
\usepackage{enumerate} % better numbered lists
\usepackage{hyperref}
\usepackage{xcolor}
\usepackage{tikz} % drawing

%\usepackage{csquotes}
%\usepackage[style=authoryear,maxcitenames=2,backend=biber,sorting=nty]{biblatex}
%\bibliography{}

\renewcommand{\epsilon}{\ensuremath{\varepsilon}}
\renewcommand{\phi}{\ensuremath{\varphi}}
\renewcommand{\theta}{\vartheta}
\renewcommand{\kappa}{\ensuremath{\varkappa}}
\renewcommand{\le}{\ensuremath{\leqslant}}
\renewcommand{\leq}{\ensuremath{\leqslant}}
\renewcommand{\ge}{\ensuremath{\geqslant}}
\renewcommand{\geq}{\ensuremath{\geqslant}}

\usepackage[useregional]{datetime2}

% exam
%\pointsinrightmargin
%\marginpointname{ б.}

% custom maketitle
\usepackage{titling}
\setlength{\droptitle}{-4em}
\posttitle{\end{center}\vspace{-4em}}

\title{{\Large Геодезическая гравиметрия 2018}\\ 
    {\bf\Large Домашнее задание № 6} \\
    {\bf\Large Абсолютные измерения силы тяжести}
}
\author{}
\DTMsavedate{deadline}{2018-04-17}

\date{\normalsize\bf Крайний срок сдачи: \DTMusedate{deadline}}

\begin{document}
\maketitle
\thispagestyle{empty}
\begin{center}
    \textbf{Основы баллистического метода}
\end{center}

Теоретической основой баллистического метода определения абсолютного значения ускорения силы тяжести
является закон ускоренного движения пробного тела в поле силы тяжести. Для системы координат,
связанной с гравитационным полем ось $z$ совпадает с направлением силы тяжести, поэтому можно записать
\begin{equation*}
    m \ddot{z} = mg\left( z \right),
\end{equation*}
где $m$ --- масса пробного тела, $\ddot{z} = \dod[2]{z}{t}$ ($t$ --- время),
$g$ --- сила тяжести. 
В общем случае гравитационное поле Земли является неоднородным (то есть величина ускорения силы
тяжести меняется с высотой), поэтому раскладывая $g\left( z
\right)$ в ряд, получаем дифференциальное уравнение движения в неоднородном поле
силы тяжести
\begin{equation*}
    \ddot{z} = g_0 + g_z z,
\end{equation*}
где $g_0$ --- ускорение силы тяжести в начале координат при $z = 0$ и $t = 0$; $g_z = W_{zz} =
\pd{g}{z}$ --- вертикальный градиент силы тяжести.

Решая уравнение для начальных условий 
\begin{equation*}
    z\left( 0 \right) = z_0,\quad \dot{z}\left( 0 \right) = 0,\quad g_z = const, 
\end{equation*}
получаем
\begin{equation*}
    z\left( t \right) = \dfrac{g_0 t^2}{2} + z_0\ch\sqrt{g_z}t +
    \dfrac{v_0}{\sqrt{g_z}}\sh\sqrt{g_z}t.
\end{equation*}
Разлагая гиперболические функции в ряд и пренебрегая членами, содержащими градиент во второй
степени, получаем
\begin{equation*}
    z\left( t \right) = z_0 + v_0t + \dfrac{1}{2} g_0 t^2 + \dfrac{1}{2}g_zt^2\left( 
        z_0 + \dfrac{1}{3} v_0 t + \dfrac{1}{12} g_0 t^2
    \right),
\end{equation*}
или
\begin{equation}
    \label{eq:basic-freefall}
    z\left( t \right) = z_0 + v_0t + \dfrac{1}{2} g_0 t^2 + 
    \dfrac{1}{6}g_z v_0 t^3 +  
    \dfrac{1}{24}g_z g_0 t^4.
\end{equation}
Уравнение связывает величину ускорения силы тяжести $g_0$ с параметрами движения пробной массы ---
пройденным путём $z$ и временем $t$, которые могут быть измерены. Это исходное уравнение для всех
баллистических гравиметров.

Вертикальный градиент силы тяжести $g_z$ имеет нормальное значение $-308,6\, \text{мкГал/м}$, однако это
величина не постоянна на поверхности Земли. Градиент, как и сила тяжести, меняется из-за влияния
близлежащих масс. На каждом гравиметрическом пункте он свой. Реальную величину $g_z$ определяют из отдельных 
измерений со статическими гравиметрами, однако это делают не всегда или не всегда делают до момента
начала абcолютных измерений. Поэтому полезно избавиться от величины $g_z$ в уравнении движения
(\ref{eq:basic-freefall}). Записываем упрощённо
\begin{equation}
    \label{eq:heff-freefall}
    z\left( t \right) = z_0^* + v_0^*t + \dfrac{1}{2} g_0^* t^2.
\end{equation}
Высота, на которой сила тяжести равна величине $g_0^*$ называется эффективной высотой и обычно
обозначается как $h_{eff}$. Значение ускорения силы тяжести на эффективной высоте не зависит от
вертикального градиента. 

Если за время свободного падения произведено $N$ измерений пути
$z_i$ и времени $t_i$ отсчитываемых от одного момента времени, 
то получим набор из $N$ функционально связанных зависимостей вида 
\begin{equation*}
    z_i = z_0^* + v_0^* t_i + \dfrac{1}{2} g_0^* t_i^2 + \epsilon_i,
\end{equation*}
где $\epsilon$ --- неучтенные (случайные) ошибки измерений времени и расстояния. Это простое
уравнение можно решить методом полиномиальной регрессии ($\epsilon$ --- нормально распределённая
случайная величина с нулевым математическим ожиданием). Согласно методу
наименьших квадратов
\begin{equation*}
    \sum\limits_{i = 1}^N\epsilon_i^2 \to \min,
\end{equation*}
а искомый вектор параметров $\mathbf{x} = \left[ z_0^*, v_0^*, g_0^* \right]^T$
является решением нормального уравнения
\begin{equation}
    \label{eq:normal}
    \mathbf{x} = \left( \mathbf{A}^T \mathbf{A} \right)^{-1}\mathbf{A}^T\mathbf{y},
\end{equation}
где 
\begin{equation*}
    \mathbf{A} =  
    \begin{bmatrix}
        \dpd{z_1}{x_1} & \dpd{z_1}{x_2} & \dpd{z_1}{x_3} \\
        \dpd{z_2}{x_1} & \dpd{z_2}{x_2} & \dpd{z_2}{x_3} \\
        \dpd{z_3}{x_1} & \dpd{z_3}{x_2} & \dpd{z_3}{x_3} \\
        \vdots & \vdots & \vdots \\
        \dpd{z_N}{x_1} & \dpd{z_N}{x_2} & \dpd{z_N}{x_3}
    \end{bmatrix} =
    \begin{bmatrix}
        1 & t_1 & \frac{1}{2} t_1^2 \\
        1 & t_2 & \frac{1}{2} t_2^2 \\
        1 & t_3 & \frac{1}{2} t_3^2 \\
        \vdots & \vdots & \vdots \\
        1 & t_N & \frac{1}{2} t_N^2
    \end{bmatrix}
\end{equation*}
--- матрица коэффициентов, $\mathbf{y} = [z_1, z_2, z_3, \dots, z_N]$ --- вектор измеренных
расстояний. Решая уравнение (\ref{eq:normal}), получим величины $z_0^*$, $v_0^*$ и $g_0^*$ на
эффективной высоте.

Эффективная высота находится внутри интервала измерений и её нельзя непосредственно измерить. 
Для того, что бы связать её с измеренной высотой
установки гравиметра над пунктом, необходимо вычислить расстояние между эффективной высотой и
начальным положением пробного тела ($z = 0$, $t = 0$, $v=0$), высота $h$ которого известна
(измерена). Приближённо (существуют более точные, но и более громоздкие зависимости) 
это можно сделать так 
\begin{equation*}
    h_{eff} = h - \underbrace{\dfrac{1}{2}\dfrac{{v^*_0}^2}{g_0^*}}_{h_0} - 
    \underbrace{\left( \dfrac{1}{3}\Delta z + \dfrac{1}{6}
    v_0^*\Delta t \right)}_{h_1},
\end{equation*}
где $h_0$ --- расстояние от начального положения тела до начала измерений, $h_1$ --- расстояние от
начала измерений до эффективной высоты, $\Delta z$ --- весь пройденный путь за полное время броска $\Delta t$.
\newpage
\begin{center}
    \textbf{Исходные данные}
\end{center}
Исходными данными для домашнего задания служат результаты абсолютных измерений с лазерным
баллистическим гравиметром ГБЛ--М на
пункте Государственной фундаментальной гравиметрической сети (ГФГС) <<ЦНИИГАиК>> (Москва): 
        \begin{table}[h]
            \centering
            \begin{tabular}{|c|c|c|}
                \hline
                 $\phi\, [^\circ]$ & $\lambda\, [^\circ]$ &$H\, [\text{м}]$ \\\hline
                 55,85503 & 37,51604 & 153\\\hline
            \end{tabular}
        \end{table}

Измерения выполнены несимметричным способом методом многих ($N = 5000$) станций. Для гравиметрa типа
ГБЛ--М время одного броска $\Delta t \approx 0,2 \,\text{с}$, за которое пробное тело проходит расстояние
$\Delta z\approx 0,66\,\text{м}$.

Значения $g_0^*$ осреднены программой постобработки результатов измерений для отдельных серий измерений.
Одна серия обычно составляет 100 бросков (единичных измерений) длительностью $\approx$ 17 минут.
Каждый бросок обрабатывается отдельно, при этом каждый раз решается уравнение типа (\ref{eq:normal}).

Для каждой серии измерений приведены
\begin{enumerate}
    \item дата и среднее время серии (UTC+00);
    \item среднее значение ускорения силы тяжести $g^*_0$ по серии, мкГал;
    \item стандартное отклонение $\sigma_{g^*_0}$ среднего значения, мкГал;
    \item число бросков, принятых в обработку $N$ (после отбраковки грубых вылетов);
    \item среднее значение атмосферного давления $P$ по серии, мм.рт.ст;
    \item среднее значение скорости $v_0^*$ в начале измерений;
    \item остаточное давление в вакуумной камере $B_i$;
    \item среднее значение поправки за прилив, мкГал;
    \item эффективная высота $h_{eff}$, для которой определена величина $g^*_0$, м;
\end{enumerate}

\begin{center}
    \textbf{Содержание задания}
\end{center}
Окончательная обработка результатов нескольких серий измерений на пункте и оценка точности.

\begin{center}
    \textbf{Порядок обработки}
\end{center}
\begin{enumerate}
    \item 
Абсолютное значение силы тяжести на пункте из наблюдений в каждой серии вычисляется по формуле
\begin{equation*}
    g = g^*_0 + \Delta g_B + \Delta g_{\lambda} + \Delta g_a  + \Delta g_{tide} + \Delta g_{polar},
\end{equation*}
где $\Delta g_B$ --- поправка за остаточное давление воздуха в баллистической камере; $\Delta
g_{\lambda}$
--- поправка за конечность скорости света; $\Delta g_a$ --- поправка за изменение атмосферного
давления; $\Delta g_{tide}$ --- поправка за прилив; $\Delta g_{polar}$ --- поправка за движение
полюса.

Поправку за остаточное давление находят по формуле
\begin{equation*}
    \Delta g_B = \underbrace{5,0}_\alpha B_i\quad[\text{мкГал}],
\end{equation*}
величина $B_i$ выражена в единицах $10^{-8}\,\text{Па}$, $\alpha = 5,0$ --- коэффициент,
определяемый экспериментально. 

Поправка за конечность скорость света (доплеровское сокращение  длины волны) равна
\begin{equation*}
    \Delta g_\lambda = \dfrac{3}{c} g_0^* \left( v_0^* + \dfrac{1}{2}g_0^* \Delta t\right),
\end{equation*}
где $c = 299 792 458\,\text{м/c}$ --- скорость света, $g_0^*$ --- вычисленное значение силы тяжести, 
$v_0^*$ --- скорость падающего тела в момент начала измерений; 
$\Delta t$ --- полное время измерений в броске.

Поправка за изменение атмосферного давления вычисляется по формуле
\begin{equation*}
    \Delta g_a = 0,4 \left( P - P_0 \right),
\end{equation*}
где $P$ --- атмосферное давление в \textbf{мм.рт.ст}, $P_0$ --- нормальное
(модельное) атмосферное давление в \textbf{мм.рт.ст}.
которое вычисляется так
\begin{equation*}
    P_0 = 1013,25 \left( 1 - \dfrac{0,0065 H}{288,15} \right)^{5.2599}\quad[\textbf{мбар}],
\end{equation*}
где $H$ --- высота пункта над уровнем моря. Барометрический фактор $K = 0,4$ здесь изменён по
сравнению с рекомендованным IAG значением $K = 0,3$ по причине того, что давление выражено в мм.рт.ст, 
а не в милибарах.

Поправка за движение полюса вычисляется по формуле
\begin{equation*}
    \Delta g_{polar} = - 1,164\times 10^8 \omega^2 a \sin{2\phi} \left( \dfrac{x_p}{\rho''} \cos\lambda -
    \dfrac{y_p}{\rho''}\sin\lambda \right)\quad[\text{мкГал}],
\end{equation*}
где $\omega$ --- угловая скорость вращения Земли; $a$ --- большая полуось, $\phi$, $\lambda$ --- широта
и долгота пункта; $x_p$, $y_p$ --- координаты полюса; $\rho'' = \dfrac{360^\circ \times 60'
\times 60''}{2\pi} \approx 206265''$.\\
Координаты полюса на заданную дату взять с сайта Международной службы вращения Земли (IERS):\par
\url{https://datacenter.iers.org/eop/-/somos/5Rgv/latest/9}\\
Описание данных:\par
\url{https://datacenter.iers.org/eop/-/somos/5Rgv/getMeta/9/finals2000A.all}.

Величина поправки за прилив приведена в задании. 

\item Окончательный результат $\bar{g}$ находят как среднее из результатов $n$ серий
    \begin{equation}
        \label{eq:mean}
        \bar{g} = \dfrac{1}{n}\sum\limits_{i=1}^{n} g.
    \end{equation}

\item Для оценки точности сначала вычисляют выборочное стандартное отклонение (среднюю
    квадратическую ошибку)
    \begin{equation}
        \sigma_g = \sqrt{\dfrac{\sum\limits_{i = 1}^n \left( g - \bar{g} \right)^2}{n-1}},
    \end{equation}
    которое характеризует разброс значений $g$ относительно среднего значения $\bar{g}$. 

    Затем находят выборочное стандартное отклонение среднего значения (средняя квадратическая ошибка
    среднего)
    \begin{equation}
        \label{eq:stdm}
        \sigma_{\bar{g}} = \dfrac{\sigma_g}{\sqrt{n}}.
    \end{equation}
\item Если точность окончательного результата оценивать только по внутренней сходимости измерений,
    то она может получиться существенно заниженной из-за неучета прочих источников ошибок, присущих
    всем измерениям с данным прибором на данном пункте. К ним относятся ошибки учёта влияния внешних
    условий (величина которых может меняться от пункта к пункту) и ошибки, связанные с конструкцией
    прибора. В геодезии такие ошибки часто называют <<систематическими>>, однако в современной теории
    оценки точности этот термин почти не применяется в силу своей ограниченности, как, впрочем, и
    термин <<ошибка>>.

    Оценкой стандартного отклонения результата измерения служит суммарная стандартная
    неопределенность $u_c$, которую получают из оценки стандартного отклонения результатов измерений
    каждой входной величины $x_i$ в виде стандартных неопределённостей $u(x_i)$

    Каждую входную оценку $x_i$ и связанную с ней стандартную неопределенность $u(x_i)$ получают 
    из вероятностного распределения значений входной величины $X_i$. Это вероятностное
    распределение можно интерпретировать как частотную вероятность, основанную на серии $k$
    наблюдений $X_{i,k}$ величины $X_i$, или как априорное распределение. Оценки составляющие
    стандартной неопределенности по типу А основаны на частотном представлении вероятности, а по
    типу B --- на априорных распределениях. Следует понимать, что в обоих случаях распределения
    отражают некоторое модельное представление знаний о случайной величине.
    
    Некоторые априорные составляющие суммарной стандартной неопределёности для гравиметра ГБЛ-М
    приведены в таблице.
    \begin{table}[h]
        \centering
        \begin{tabular}{|c|c|}
            \hline
            Источник неопределённости & Величина $u (x_i)$, мкГал \\\hline
            Определение длины волны лазера & $\pm 1$ \\
            Измерение интервалов времени & $\pm 1$ \\\hline
            Поправка за остаточное давления  & $\pm 2$ \\
            Поправка за изменение атмосферного давления & $\pm 0,5$\\
            Поправка за прилив & $\pm 2$\\  
            Поправка за движение полюса & $\pm 0,5$\\\hline  
            Вращение пробного тела и ошибка дифракции & $\pm 2$\\  
            Влияние электромагнитных сил & $\pm 1$\\  
            Микросейсмы & $\pm 2$\\  
            Выставление вертикали & $\pm 1$\\\hline
            $\sum$ & $\pm 4,5$ \\\hline
        \end{tabular}
    \end{table}

    Сумма в последней строке априорных составляющих суммарной стандартной неопределённости 
    (пользуясь более привычными
    терминами - сумма систематических составляющих) вычисляется по величинам, указанным в таблице
    стандартным методом
    \begin{equation}
        u_2 =  \sqrt{\sum u^2(x_i)}
    \end{equation}
    Суммарная стандартная неопределённость, то есть оценка точности окончательного результата,
    вычисляется так
    \begin{equation}
        u_c = \sqrt{u_1^2 + u_2^2},
    \end{equation}
    где $u_1 = \sigma_{\bar{g}}$ вычисляется по формуле (\ref{eq:stdm}).

    В качестве окончательной оценки значения ускорения силы тяжести $\bar{g}$ принимается
    расширенная суммарная стандартная неопределенность
    \begin{equation}
        U_c = k u_c,
    \end{equation}
    где $k$ --- коэффициент охвата, который обычно принимается равным 2, что соотвествует 
    интервалу доверия с уровнем, близким к 95\%.
\end{enumerate}

\begin{center}
    \textbf{Дополнительно}
\end{center}

Попробуйте изменить формулы (\ref{eq:mean}) -- (\ref{eq:stdm}) для учёта неравноточности измерений в
сериях.

%\printbibliography
\end{document}
