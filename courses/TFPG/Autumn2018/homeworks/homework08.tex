\documentclass[11pt, a4paper,addpoints]{exam}

% Languages and fonts
\usepackage{cmap} 
\usepackage[T2A]{fontenc}
\usepackage[utf8]{inputenc} 
\usepackage[english, russian]{babel}
\usepackage{microtype}
\usepackage{indentfirst}
\frenchspacing

% Mathematics
\usepackage{amsmath, amssymb, amsfonts, amsthm, mathtools, fixmath}
%\mathtoolsset{showonlyrefs=true}
\usepackage{esint, esvect} % integrals and vectors
\usepackage{systeme} % equation system
\usepackage{commath} % partials and differentials
\usepackage{icomma} % smart comma ($0,2$ is a number)

% Floats
\usepackage{float}

% Tables
\usepackage{array,tabularx,tabulary,booktabs} % better tables
\usepackage{longtable}
\usepackage{multirow}

% Graphics
\usepackage[pdftex]{graphicx}
\usepackage{wrapfig}

% Theorems
\renewcommand{\proofname}{Доказательство}

%\theoremstyle{plain}
\newtheorem{theorem}{Теорема}[section]
\newtheorem*{theorem*}{Теорема}

%\theoremstyle{definition}
\newtheorem{definition}{Определение}
\newtheorem{corollary}{Следствие}[theorem]

\theoremstyle{remark}
\newtheorem{remark}{Замечание}

\usepackage[top=20mm,bottom=20mm,left=20mm,right=20mm]{geometry}

\usepackage{soul}
\usepackage{enumerate} % better numbered lists
\usepackage{hyperref}
\usepackage{xcolor}
\usepackage{tikz} % drawing

\usepackage{csquotes}
\usepackage[style=numeric,maxcitenames=2,backend=biber,sorting=nty]{biblatex}
\bibliography{../../../../bibliography}

\footer{}{\thepage}{}
{}

\renewcommand{\epsilon}{\ensuremath{\varepsilon}}
\renewcommand{\phi}{\ensuremath{\varphi}}
\renewcommand{\theta}{\vartheta}
\renewcommand{\kappa}{\ensuremath{\varkappa}}
\renewcommand{\le}{\ensuremath{\leqslant}}
\renewcommand{\leq}{\ensuremath{\leqslant}}
\renewcommand{\ge}{\ensuremath{\geqslant}}
\renewcommand{\geq}{\ensuremath{\geqslant}}

\usepackage[useregional]{datetime2}

% exam
\pointsinrightmargin
\marginpointname{ б.}

% custom maketitle
\usepackage{titling}
\setlength{\droptitle}{-4em}
\posttitle{\end{center}\vspace{-4em}}

\title{{\Large Теория фигур планет и гравиметрия 2018}\\ 
    {\bf\Large Домашнее задание № 8}}
\author{}
\DTMsavedate{deadline}{2018-11-30}

\date{\normalsize\bf Крайний срок сдачи: \DTMusedate{deadline}}

\begin{document}
\maketitle
\begin{questions}
    \question Вычислить аномалии в свободном воздухе $g-\gamma$ и аномалии Буге $\Delta g_{B}$
    ($\delta = 2,67\,\textrm{г/см}^3$) для
    всех пунктов каталога согласно своему варианту. Точность вычисления $0,1\,\textrm{мГал}$.

    Аномалию в свободном воздухе находят по формуле
    \begin{equation*}
        \Delta g = g - \gamma = g - \left( \gamma_0 + \dpd{\gamma}{H} H \right),
    \end{equation*}
    где $g$ --- измеренное значение силы тяжести из каталога, $\gamma_0$ --- значение нормальной
    силы тяжести на эллипсоиде, вычисляемое по формуле Гельмерта 1901---1909 г.:
    \begin{equation*}
        \gamma_0 = 978030 \left( 1 + 0,005302\sin^2{B} - 0,000007\sin^2{2B} \right) \quad \left(
        \textrm{мГал} \right),
    \end{equation*}
    где $B$ --- геодезическая широта. Редукция в свободном воздухе (поправка за высоту) вычисляется так
    \begin{equation*}
        \dpd{\gamma}{H}H = -0,3086 H \left( \textrm{мГал} \right),
    \end{equation*}
    где $\dpd{\gamma}{H} = -0,3086\,\textrm{мГал/м}$ --- нормальный вертикальный градиент силы
    тяжести, $H$ --- высота в принятой системе высот. 

    Аномалию Буге вычисляют по формуле
    \begin{equation*}
        \Delta g_B = \Delta g - 2\pi G \delta H,
    \end{equation*}
    где $2\pi G \delta H$ --- поправка за промежуточный слой (редукция Буге). 
    $G = 6,673\cdot 10^{-8}\,\textrm{см}^3/(\textrm{г}\cdot\textrm{с}^2)$ --- постоянная тяготения,
    используя значение которой для аномалии Буге можно записать
    \begin{equation*}
        \Delta g_B = \Delta g - 0,0419 \delta H.
    \end{equation*}
    \question Построить карту аномалий в свободном воздухе и аномалий Буге только по пунктам с
    подписанными аномалиями (10 пунктов из каталога исключаются).
    \question Построить график зависимости аномалий в свободном воздухе и аномалий Буге от высоты
    для всех пунктов каталога. Предполагая зависимость аномалий от высоты линейной, провести
    регрессионную прямую и получить коэффициент $\Delta g / \Delta H$ в мГал/м для обеих аномалий.
    \question Вычислить ошибку $E$ интерполяции гравиметрической карты по десяти исключённым из каталога
    пунктам.
\end{questions}
%\printbibliography
\end{document}
