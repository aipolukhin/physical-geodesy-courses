\documentclass[11pt, a4paper,addpoints]{exam}

% Languages and fonts
\usepackage{cmap} 
\usepackage[T2A]{fontenc}
\usepackage[utf8]{inputenc} 
\usepackage[english, russian]{babel}
\usepackage{microtype}
\usepackage{indentfirst}
\frenchspacing

% Mathematics
\usepackage{amsmath, amssymb, amsfonts, amsthm, mathtools, fixmath}
\mathtoolsset{showonlyrefs=true}
\usepackage{esint, esvect} % integrals and vectors
\usepackage{systeme} % equation system
\usepackage{commath} % partials and differentials
\usepackage{icomma} % smart comma ($0,2$ is a number)

% Floats
\usepackage{float}

% Tables
\usepackage{array,tabularx,tabulary,booktabs} % better tables
\usepackage{longtable}
\usepackage{multirow}

% Graphics
\usepackage[pdftex]{graphicx}
\usepackage{wrapfig}

% Theorems
\renewcommand{\proofname}{Доказательство}

%\theoremstyle{plain}
\newtheorem{theorem}{Теорема}[section]

%\theoremstyle{definition}
\newtheorem{definition}{Определение}
\newtheorem{corollary}{Следствие}[theorem]

\theoremstyle{remark}
\newtheorem{remark}{Замечание}

\usepackage[top=20mm,bottom=20mm,left=20mm,right=20mm]{geometry}

\usepackage{soul}
\usepackage{enumerate} % better numbered lists
\usepackage{hyperref}
\usepackage{xcolor}
\usepackage{tikz} % drawing

\usepackage{csquotes}
\usepackage[style=numeric,maxcitenames=2,backend=biber,sorting=nty]{biblatex}
\bibliography{../../../../bibliography}

\footer{}{\thepage}{}

\renewcommand{\epsilon}{\ensuremath{\varepsilon}}
\renewcommand{\phi}{\ensuremath{\varphi}}
\renewcommand{\theta}{\vartheta}
\renewcommand{\kappa}{\ensuremath{\varkappa}}
\renewcommand{\le}{\ensuremath{\leqslant}}
\renewcommand{\leq}{\ensuremath{\leqslant}}
\renewcommand{\ge}{\ensuremath{\geqslant}}
\renewcommand{\geq}{\ensuremath{\geqslant}}

\usepackage[useregional]{datetime2}

% exam
\pointsinrightmargin
\marginpointname{ б.}

% custom maketitle
\usepackage{titling}
\setlength{\droptitle}{-4em}
\posttitle{\end{center}\vspace{-4em}}

\title{{\Large Теория фигур планет и гравиметрия 2018}\\ 
    {\bf\Large Домашнее задание № 5}}
\author{}
\DTMsavedate{deadline}{2018-11-02}

\date{\normalsize\bf Крайний срок сдачи: \DTMusedate{deadline}}

\begin{document}
\maketitle
\thispagestyle{empty}
    \begin{questions}
        \question Пусть потенциал притяжения представлен в видя ряда по шаровым функциям
        \begin{equation}
        V \left( r, \phi, \lambda \right) = \dfrac{GM}{r} \left[ 1 +  
            \sum\limits_{n=2}^{\infty} \left( \dfrac{a}{r} \right)^n
        \sum\limits_{k=0}^{n} \left(  
        \bar{C}_{nk}\cos{k\lambda} + \bar{S}_{nk}\sin{k\lambda}
        \right) \bar{P}_{nk} \left( \sin{\phi} \right)\right],
    \end{equation}
    где $r, \phi, \lambda$ --- сферические координаты, $n$ и $k$ --- степень и порядок,
    $\bar{C}_{nk},\bar{S}_{nk}$ --- полностью нормированные стоксовы постоянные, 
    $\bar{P}_{nk}\left( \sin{\phi} \right)$ --- полностью нормированные присоединённые функции
    Лежандра, $GM$ --- планетоцентрическая гравитационная постоянная.
        \begin{parts}
        \part[1] Написать в общем виде выражения для частных производных потенциала притяжения
        по сферическим координатам ($V_r = \pd{V}{r}$, $V_{\theta} = \pd{V}{\theta}$, $V_\lambda =
        \pd{V}{\lambda}$). Как по полученным величинам точно вычислить силу притяжения? 
        \paragraph{Подсказка.} Градиент в ортогональных криволинейных координатах выражается так
        \begin{equation*}
            \nabla V \left( q_1, q_2, q_3 \right) = \left\{
                \dfrac{1}{h_1}\dpd{V}{q_1} ;
            \dfrac{1}{h_2}\dpd{V}{q_2} ;
        \dfrac{1}{h_3}\dpd{V}{q_3} \right\},
        \end{equation*}
        где $h_1, h_2, h_3$ --- коэффициенты Ламе.
        \part[1] Написать в явном виде разложение силы тяжести в ряд по шаровым функциям до
        4--го порядка в радиальном приближении ($|\vec{g}| = |\nabla W | \approx \partial W /
        \partial r$). 
        \part[1] Вычислить значение потенциала притяжения, потенциала силы тяжести и силы тяжести по
        глобальной модели гравитационного поля Земли, ограничившись первыми четырьмя степенями
        разложения ($n = 4$), для точки на территории МИИГАиК с координатами (эллипсоид $WGS84$):
        \begin{equation*}
            B = 55,764058^\circ,\quad L = 37,661425^\circ,\quad H = 158,064\,\textrm{м}.
        \end{equation*}
    \end{parts}
        \question[2] Определить моменты инерции и их отношения по полностью нормированным
        коэффициентам разложения потенциала в ряд по шаровым функциям (см. задание ниже).
    \end{questions}
    \section*{\centering Получение модели}
    Глобальные модели гравитационного поля можно получить с сайта Международного центра для глобальных
    моделей (ICGEM). Для этого необходимо сделать следующее.
    \begin{enumerate}
        \item Зайти в раздел Static Models (статические модели) по адресу:\\
            \url{http://icgem.gfz-potsdam.de/tom_longtime}.
        \item Из списка найти модель с номером $Nr = 167 - i$, где $i$ --- номер варианта.
        \item Скачать файл модели с расширением *.gfc.
        \item В заголовке файла: \\
            earth\_gravity\_constant -- геоцентрическая гравитационная постоянная, \\
            radius --- значение $a$, которое использовалось для получения безразмерных
            коэффициентов, \\
            norm --- норма коэффициентов (необходимо убедиться, что здесь стоит fully\_normalized
            --- полностью нормированные).
    \end{enumerate}
    \newpage
    \section*{\centering Изучение механических свойств планет}
    \subsection*{\centering Теоретическая справка}
    Потенциал притяжения планеты может быть представлен в виде ряда шаровых функций
    \begin{equation*}
        V\left( r, \theta, \lambda \right) = \sum\limits_{n=0}^{\infty} \dfrac{1}{r^{n+1}}
        \sum\limits_{k=0}^{n}\left( A_{nk}\cos{k\lambda} + B_{nk}\sin{k\lambda} \right) P_n^k \left(
        \cos\theta \right)
    \end{equation*}
    где $r, \theta, \lambda$ --- сферические координаты, $n$ и $k$ --- степень и порядок,
    $P_{nk}\left( \cos{\theta} \right)$ --- присоединённые функции
    Лежандра, $A_{nk}, B_{nk}$ --- гармонические коэффициенты, определяемые как 
    \begin{equation*}
    \begin{aligned}
        A_{n0} &= G\int\limits_\tau \delta {r'}^n P_n \left( \cos{\theta'} \right) d\tau, \\
        \begin{Bmatrix}
            A_{nk} \\ B_{nk}
        \end{Bmatrix} &= 
        2 \dfrac{\left( n - k \right)!}{\left( n + k \right)!}
        G \int\limits_\tau \delta {r'}^n 
        \begin{Bmatrix}
            \cos{k\lambda'} \\
            \sin{k\lambda'} \\
        \end{Bmatrix}
        P_n^k \left( \cos{\theta'} \right) d\tau. \\
    \end{aligned}
    \end{equation*}
    Все коэффициенты имеют одинаковый вид: это интегралы по объёму $\tau$ от произведения плотности
    $\delta$ на гармоническую функцию ${r'}^n \cos{k\lambda'} P_n^k\left( \theta' \right)$ или
    ${r'}^n \sin{k\lambda'} P_n^k\left( \theta' \right)$. Такие интегралы называются  
    \textbf{стоксовыми постоянными}. Таким образом, коэффициенты разложения потенциала притяжения в
    ряд по шаровым функциям являются стоксовыми постоянными. Стксовы постоянные первых степеней
    имеют вполне определённый физический смысл\cite{Ogorodova2013}.

    \subsubsection*{Масса}
    При $n = 0$:
    \begin{equation*}
        A_{00} = G \int\limits_\tau \delta d\tau = GM
    \end{equation*}
    --- произведение гравитационной постоянной на массу тела. 

    В системе СИ рекомендованное Комитетом данных для науки и техники (CODATA) значение в 2014 году\cite{CODATA2014} такое
    \begin{equation*}
        G = 6,67408 \pm 0,00031)\times10^{-11}\,\text{м}^3\text{кг}^{-1}\text{с}^{-2}.
    \end{equation*}
    Теперь можно получить массу планеты
    \begin{equation}
        \label{eq:mass}
        M = \dfrac{GM}{G}.
    \end{equation}

    \subsubsection*{Координаты центра масс}
    При $n = 1$:
    \begin{align*}
        A_{10} &= G \int\limits_\tau \delta r' \cos{\theta'}d\tau =
        G \int\limits_\tau \delta z'd\tau = GMz_0, \\
        A_{11} &= G \int\limits_\tau \delta r' \sin{\theta'}\cos{\lambda'}d\tau = 
        G \int\limits_\tau \delta x'd\tau = GMx_0, \\
        B_{11} &= G \int\limits_\tau \delta r' \sin{\theta'}\cos{\lambda'}d\tau = 
        G \int\limits_\tau \delta y'd\tau = GMy_0,
    \end{align*}
    где $x_0, y_0, z_0$ --- координаты центра масс относительно начала выбранной системы координат. 
    Здесь учтена связь  прямоугольных и сферических координат
    \begin{equation*}
        x = r\sin\theta\cos\lambda, \quad y = r\sin\theta\cos\lambda, \quad z = r\cos\theta,
    \end{equation*}
    а также то, что координаты центра масс получаются из выражений
    \begin{equation*}
        x_0 = \dfrac{1}{M} \int \delta x' d\tau, \quad
        y_0 = \dfrac{1}{M} \int \delta y' d\tau, \quad
        z_0 = \dfrac{1}{M} \int \delta z' d\tau.
    \end{equation*}
    Если система координат выбрана так, что её начало совпадает с центром масс, то $x_0 = y_0 = z_0
    = 0$ и $A_{10} = A_{11} = B_{11} = 0$.

    В геодезических и астрономических приложениях коэффициенты первой степени 
    разложения часто не принимают во внимание: предполагают,
    что начало системы координат выбрано точно в центре масс. Однако, более детальный анализ
    гравитационных полей планет иногда приводит к выводу о смещении центра масс по отношении к
    геометрическому центру объема планеты.

    \subsubsection*{Моменты инерции}
    При $n = 2$:
    \begin{align*}
        A_{20} &= G \int\limits_\tau \delta r'^2 \left( \dfrac{3}{2}\cos^2{\theta'} - \dfrac{1}{2} \right)d\tau =
        G \int\limits_\tau \delta \left(\dfrac{x'^2 + z'^2}{2} + \dfrac{y'^2 + z'^2}{2} - \left(
        x'^2 + y'^2 \right)\right) d\tau, \\
        A_{21} &= G \int\limits_\tau \delta r'^2 \cos{\theta'}\sin{\theta'} \cos{\lambda'} d\tau =
        G\int\limits_\tau \delta x'z' d\tau, \\
        B_{21} &= G \int\limits_\tau \delta r'^2 \cos{\theta'}\sin{\theta'} \sin{\lambda'} d\tau =
        G\int\limits_\tau \delta y'z' d\tau, \\
        A_{22} &= \dfrac{G}{12} \int\limits_\tau \delta r'^2 \sin{2\theta'} \cos{2\lambda'} d\tau =
        \dfrac{G}{4}\int\limits_\tau \delta \left( \left( x'^2 + z'^2 \right) - 
        \left( y'^2 + z'^2 \right) \right)  d\tau, \\
        B_{22} &= G \int\limits_\tau \delta r'^2 \sin^2{\theta'} \sin{2\lambda'} d\tau =
        G\int\limits_\tau \delta y'x' d\tau.
    \end{align*}
    Все коэффициенты второй степени связаны с моментами инерции.

    Момент инерции --- скалярная мера инертности тела (планеты) во вращательном движении, подобно тому, как
    масса является мерой инертности тела в поступательном движении. Моменты инерции определяются
    относительно точки, оси или плоскости и зависят от распределения масс внутри тела. В общем виде
    момент инерции определяется так
    \begin{equation*}
        I = \int\limits_{\tau} r^2 d\tau,
    \end{equation*}
    где $r$ --- расстояние до точки, оси или плоскости, относительно которых определяется момент
    инерции.

    Моменты инерции играют важнейшую роль в \textbf{теории вращения Земли и планет} --- ещё
    одной геодезической дисциплине. Если в первом приближении считать планету абсолютно твёрдым
    телом, то для изучения её вращения можно воспользоваться законом сохранения момента
    импульса, который гласит, что момент импульса замкнутой системы при равенстве момента внешних
    сил нулю сохраняется. В инерциальной системе отсчёта можно записать
    \begin{equation*}
        \od{\mathbf{L}}{t} = \mathbf{G},
    \end{equation*}
    где $\mathbf{L}$ --- момент импульса (угловой момент), $t$ --- время, $\mathbf{G}$ --- 
    вращающий момент внешних сил, действующих на тело. Если $\mathbf{G} = 0$, 
    то, очевидно, $\mathbf{L} = const$. 

    Если на планету действуют силы, момент которых не равен нулю, то под их действием происходит
    изменение ориентации вектора углового момента планеты. По определению момент импульса
    равняется произведению тензора инерции $\pmb{I}$ на вектор угловой скорости $\pmb{\omega}$:
    \begin{equation*}
        \mathbf{L} = \pmb{I}\pmb{\omega}.
    \end{equation*}
    Если $\pmb{G}\neq 0$, то векторы $\pmb{I}$ и $\pmb{\omega}$ будут изменять свое положение в
    инерциальной системе отсчёта. Основными внешними силами, возмущающими вращение Земли, являются
    притяжение Луны и Солнца, то есть приливные силы. Смещение вектора момента импульса под их
    действием называется лунно--солнечной прецессией. Периодические, более частые, чем прецессия, 
    изменения положения вектора момента импульса, вызванные непостоянством положения Луны и Солнца,
    называются нутацией.

    Во вращающейся с планетой системой отсчёта можно записать
    \begin{equation*}
        \od{\mathbf{L}}{t} + \pmb{\omega}\times\pmb{L} = \mathbf{G},
    \end{equation*}

    Последнее уравнение используется для определения влияния геофизических процессов, таких как
    перемещение масс в атмосфере и океанах, тектоническое движение плит коры Земли, землетрясений и
    т.д., на вращение Земли и описывает движение вектора в земной системе координат. Эти процессы
    приводят к изменению тензора инерции Земли, влияя, следовательно, на вращение Земли. Если
    считать, что атмосфера и океаны связаны с Землей и составляют с Землей замкнутую систему, то
    $\pmb{L} = 0$. Это значит, что вектор углового момента под действием геофизических процессов сохраняет свое
    положение в пространстве. Но так как из-за перемещения масс происходит изменение тензора инерции
    Земли, то вектор $\pmb{L}$ изменяет свою ориентацию относительно вектора $\pmb{\omega}$, т.е.
    вектор $\pmb{L}$ движется относительно самой Земли. Наблюдателю, находящемуся на поверхности Земли, кажется, что Земля
    качается относительно оси углового момента. Поэтому иногда это движение называется качанием
    Земли (по--английски <<wobble>>), но чаще --- движением полюса\cite{Zharov2006}.

    Тензор инерции относительно точки О состоит из девяти компонентов и является симметричным:
    \begin{equation*}
        \pmb I = \begin{pmatrix}
            I_{xx} & I_{xy} & I_{xz} \\
            I_{yx} & I_{yy} & I_{yz} \\
            I_{zx} & I_{zy} & I_{zz} \\
        \end{pmatrix}.
    \end{equation*}
    Компоненты $I_{ik}$ определяются через координаты $(x_1, x_2, x_3) = (x, y, z)$ так:
    \begin{equation*}
        I_{ik} = \int\limits_M 
        \left[ \sum\limits_{j = 1}^{3} \delta_{ik} x_j^2 - x_i x_k \right] dm, \qquad
        I_{ik} = I_{ki},
    \end{equation*}
    откуда записываем  выражения для диагональных элементов тензора инерции
    \begin{equation*}
        I_{xx} = A = \int\limits_\tau \delta \left( y'^2 + z'^2 \right) d\tau, \quad
        I_{yy} = B = \int\limits_\tau \delta \left( x'^2 + z'^2 \right) d\tau, \quad
        I_{zz} = C = \int\limits_\tau \delta \left( x'^2 + y'^2 \right) d\tau, \quad
    \end{equation*}
    которые называются \textbf{осевыми моментами инерции}. Здесь $I_{xx} = A$ --- момент инерции относительно
    оси $x$, $I_{yy} = B$ --- момент инерции относительно оси $y$,
    $I_{zz} = C$ --- момент инерции относительно оси $z$ или или полярный момент инерции.
    Обозначения $A$ и $B$ никак не связаны с обозначениями стоксовых постоянных $A_{nk}$ и
    $B_{nk}$.

    Для  \textbf{центробежных моментов} (произведений инерции), которые являются недиагональными элементами
    тензора инерции, получим
    \begin{equation*}
        I_{23} = -D = -\int\limits_\tau \delta y'^2 z'^2 d\tau, \quad
        I_{13} = -E = -\int\limits_\tau \delta x'^2 z'^2 d\tau, \quad
        I_{12} = -F = -\int\limits_\tau \delta x'^2 y'^2 d\tau, \quad
    \end{equation*}
    где $I_{23} = -D$ --- центробежный момент инерции относительно оси $x$,
    $I_{13} = -E$ --- центробежный момент инерции относительно оси $y$,
    $I_{12} = -F$ --- центробежный момент инерции относительно оси $z$.

    Теперь для коэффицентов второй степени можно записать
    \begin{align*}
        A_{20} &= G\left( \dfrac{A + B}{2} - C \right), \qquad
        A_{22} = \frac{1}{4}G\left( B - A \right), \\
        A_{21} &= GE,  \qquad
        B_{21} = GD,  \qquad
        B_{22} = \dfrac{1}{2}GF.
    \end{align*}
    Следовательно, коэффициент $A_{20}$ является разностью среднего экваториального момента инерции
    и полярного момента инерции; коэффициент $A_{22}$ --- разность экваториальных моментов инерции;
    $A_{21}, B_{21}, B_{22}$ связаны с произведениями инерции. Коэффициенты высших порядков являются
    моментами высших порядков или мультимоментами, однако их механическая интерпретация
    затруднительна.

    Если два центробежных момента инерции относительно двух осей равны нулю, то третья ось,
    перпендикулярная этим двум, будет называться \textbf{главной осью инерции}. При этом, через
    любую точку тела можно провести три перпендикулярные главные оси инерции. Если тело имеет ось
    симметрии, то эта ось является главной осью инерции (обратное может и не выполняться).
    Если все центробежные моменты инерции равны нулю, то каждая из координатных осей является
    главной осью инерции для начала координат.

    Осевые моменты инерции относительно главных осей инерции называются \textbf{главными моментами
    инерции}. Если главные оси инерции проходят через центр масс, то они называются \textbf{главными
    центральными осями инерции тела}.

    Для Земли средняя ось вращения --- полярная ось, проходящая через средний полюс --- 
    приближённо является главной осью симметрии, поэтому она является и одной из главных осей
    инерции и коэффициенты $A_{21}$ и $B_{21}$ будут очень малыми величиными, но всё же отличными от
    нуля.

    Оставшийся центробежный момент инерции $F = \int\limits_\tau \delta x'^2 y'^2 d\tau$
    относительно оси $z$ обратился бы  в нуль  в том случае, если бы оси координат $x$ и $y$
    были совмещены с оставшимися двумя главными осями инерции. Однако направление экваториальных осей
    установлено действующими конвенциями и не выбиралось под условие совпадения их с главными осями
    инерции, поэтому $F \neq 0$ и $B_{22} \neq 0$. Именно выбор Гринвичского меридиана в качестве
    направления оси $x$ и определяет величину коэффициента $B_{22}$.

    Таким образом, например, для других планет, выбирая значения коэффициентов $A_{21}, B_{21},
    B_{22}$ равными нулю можно добиться совмещения координатных осей с главными осями инерции, а
    приравнивая коэффициенты $A_{10}, A_{00}, B_{11}$ нулю --- добиться совмещения начала системы координат с
    центром масс небесного тела.

    Геометрическая интерпретация оставшихся коэффициентов такая. Коэффициент $A_{20}$ пропорционален
    разности между средним экваториальным $\frac{A+B}{2}$ и полярным
    $C$ моментами инерции, поэтому он характеризует полярное сжатия планеты. Кэффициент
    $A_{22}$ пропорционален разности экваториальных моментов $B-A$ и потому он характеризует
    экваториальное сжатие (<<трёхосность>>). Если бы Земля была идеальном телом вращения, сферой или
    эллипсоидом, то $A = B < C$ и коэффициент $A_{22}$ обратился бы в нуль.

    \subsubsection*{Вычисление моментов инерции}
    Моменты инерции невозможно напрямую измерить, поэтому их вычисляют по стоксовым
    постоянным. Обычно разложение потенциала притяжения в ряд по шаровым функциям представляют в виде
    \begin{equation}
        V \left( r, \phi, \lambda \right) = \dfrac{GM}{r} \left[ 1 +  
            \sum\limits_{n=2}^{\infty} \left( \dfrac{a}{r} \right)^n
        \sum\limits_{k=0}^{n} \left(  
        \bar{C}_{nk}\cos{k\lambda} + \bar{S}_{nk}\sin{k\lambda}
        \right) \bar{P}_{nk} \left( \sin{\phi} \right)\right],
    \end{equation}
    где $r, \phi, \lambda$ --- сферические координаты, $n$ и $k$ --- степень и порядок,
    $\bar{C}_{nk},\bar{S}_{nk}$ --- полностью нормированные стоксовы постоянные, 
    $\bar{P}_{nk}\left( \sin{\phi} \right)$ --- полностью нормированные присоединённые функции
    Лежандра, $GM$ --- планетоцентрическая гравитационная постоянная.

    Полностью нормированные коэффициенты $\bar{C}_{nk}, \bar{S}_{nk}$ связаны с ненормированными
    коэффициентами $C_{nk}, S_{nk}$ следующими выражениями
    \begin{equation*}
        \bar{C}_{n0} = \dfrac{C_{n0}}{\sqrt{2n+1}}, \qquad
        \begin{Bmatrix}
            \bar{C}_{nk} \\
            \bar{S}_{nk} \\
        \end{Bmatrix}  =
    \sqrt{\dfrac{1}{2(2n + 1)}\dfrac{(n + k)!}{(n-k)!}}
        \begin{Bmatrix}
            C_{nk} \\
            S_{nk} \\
        \end{Bmatrix},
        \quad k \neq 0.
    \end{equation*}
    Безразмерные ненормированные коэффициенты $C_{nk}, S_{nk}$ связаны с коэффицнетами $A_{nk}$ и
    $B_{nk}$ следующим образом
    \begin{equation*}
        C_{nk} = \dfrac{A_{nk}}{GMa^n}, \qquad
        S_{nk} = \dfrac{B_{nk}}{GMa^n},
    \end{equation*}
    где $GM$ --- планетоцентрическая гравитационная постоянная, $a$ --- некоторая линейная
    константа, для Земли выбираемая близкой или равной большой полуоси эллипсоида.

    Тогда можно получить связь моментов инерции с полностью нормированными коэффициентами разложения
    потенциала притяжения. Например, для коэффициентов $\bar{C}_{20}$ и $\bar{C}_{22}$ получаем
    \begin{equation}
        \label{eq:c20c22}
        \bar{C}_{20} = \dfrac{\dfrac{A + B}{2} - C}{\sqrt{5}Ma^2},\quad
        \bar{C}_{22} = \dfrac{\sqrt{15}\left( B - A \right)}{10 M a^2}
    \end{equation}
    В этих двух уравнениях три неизвестных, массу можно
    вычислить по коэффициенту $\bar{C}_{00}$.

    Введём величину постоянной прецессии (астрономического динамического сжатия), величина которой
    для Земли получается из измерений с интерферометрами со сверхдлинной базой (РСДБ)
    \begin{equation*}
        \label{eq:H}
        H = \dfrac{C - \dfrac{A + B}{2}}{C}
    \end{equation*}
    и примем следующую его величину
    \begin{equation*}
        H = (327376,34 \pm 0,32) \times 10^{-8}.
    \end{equation*}
    Теперь можем записать для осевых моментов инерции
    \begin{equation*}
    \begin{cases} 
        A = \sqrt{5} Ma^2 \left[ \left( 1 - \dfrac{1}{H} \right)\bar{C}_{20} -
        \dfrac{\bar{C}_{22}}{\sqrt{3}} \right] \\
        B = \dots \\
        C = \dots \\
    \end{cases}
    \end{equation*}
    Выражение для центробежных моментов инерции, очевидно, получаются проще.

    Одной из важных характеристик планет является так называемый безразмерный момент инерции,
    который определяется как отношение осевого момента инерции к максимальному моменту инерции. Для
    полярного момента имеем:
    \begin{equation*}
        I^* = \dfrac{C}{Ma^2}.
    \end{equation*}
    Произведение $Ma^2$ является максимальным моментом инерции, который можно рассматривать как
    момент инерции точки, равной по массе планете и расположенной на расстоянии $a$ от оси,
    относительно которой определяется момент. Величина принимает занчения $0.0 < I^* < 0,4$ и
    характеризует внутреннее строение планеты. Он будет тем меньше, чем больше массы у планеты
    расположено в её центре. Предельный случай $0,4$ соответствует однородному
    шару. Именно безразмерный момент инерции позволяет определить, есть у планеты
    тяжелое жидкое ядро. 
    \subsection*{\centering Задание}
    \begin{enumerate}
        \item Получить в явном виде формулы для вычисления массы, координат центра масс, 
            а также всех осевых и центробежных моментов
            инерции по полностью нормированным коэффициентам разложения потенциала притяжения в ряд
            по шаровым функциям.
        \item Вычислить следующие величины: 
            \begin{enumerate}
                \item Массу планеты $M$.
                \item Координаты центра масс $x_0$, $y_0$, $z_0$.
                \item Разности осевых моментов инерции
                    \begin{equation*}
                        C - A, \quad C - B, \quad B - A.
                    \end{equation*}
                \item Максимальный момент инерции $Ma^2$.
                \item Отношение разностей осевых моментов к максимальному моменту инерции.
                \item Осевые и центробежные моменты инерции (элементы тензора инерции).
                \item Безразмерный момент инерции $I^*$ для всех осевых моментов инерции.
            \end{enumerate}
    \end{enumerate}
\printbibliography
\end{document}
