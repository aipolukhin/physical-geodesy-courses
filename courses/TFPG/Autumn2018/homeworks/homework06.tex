\documentclass[11pt, a4paper,addpoints]{exam}

% Languages and fonts
\usepackage{cmap} 
\usepackage[T2A]{fontenc}
\usepackage[utf8]{inputenc} 
\usepackage[english, russian]{babel}
\usepackage{microtype}
\usepackage{indentfirst}
\frenchspacing

% Mathematics
\usepackage{amsmath, amssymb, amsfonts, amsthm, mathtools, fixmath}
%\mathtoolsset{showonlyrefs=true}
\usepackage{esint, esvect} % integrals and vectors
\usepackage{systeme} % equation system
\usepackage{commath} % partials and differentials
\usepackage{icomma} % smart comma ($0,2$ is a number)

% Floats
\usepackage{float}

% Tables
\usepackage{array,tabularx,tabulary,booktabs} % better tables
\usepackage{longtable}
\usepackage{multirow}

% Graphics
\usepackage[pdftex]{graphicx}
\usepackage{wrapfig}

% Theorems
\renewcommand{\proofname}{Доказательство}

%\theoremstyle{plain}
\newtheorem{theorem}{Теорема}[section]
\newtheorem*{theorem*}{Теорема}

%\theoremstyle{definition}
\newtheorem{definition}{Определение}
\newtheorem{corollary}{Следствие}[theorem]

\theoremstyle{remark}
\newtheorem{remark}{Замечание}

\usepackage[top=20mm,bottom=20mm,left=20mm,right=20mm]{geometry}

\usepackage{soul}
\usepackage{enumerate} % better numbered lists
\usepackage{hyperref}
\usepackage{xcolor}
\usepackage{tikz} % drawing

\usepackage{csquotes}
\usepackage[style=numeric,maxcitenames=2,backend=biber,sorting=nty]{biblatex}
\bibliography{../../../../bibliography}

\footer{}{\thepage}{}
{}

\renewcommand{\epsilon}{\ensuremath{\varepsilon}}
\renewcommand{\phi}{\ensuremath{\varphi}}
\renewcommand{\theta}{\vartheta}
\renewcommand{\kappa}{\ensuremath{\varkappa}}
\renewcommand{\le}{\ensuremath{\leqslant}}
\renewcommand{\leq}{\ensuremath{\leqslant}}
\renewcommand{\ge}{\ensuremath{\geqslant}}
\renewcommand{\geq}{\ensuremath{\geqslant}}

\usepackage[useregional]{datetime2}

% exam
\pointsinrightmargin
\marginpointname{ б.}

% custom maketitle
\usepackage{titling}
\setlength{\droptitle}{-4em}
\posttitle{\end{center}\vspace{-4em}}

\title{{\Large Теория фигур планет и гравиметрия 2018}\\ 
    {\bf\Large Домашнее задание № 6}}
\author{}
\DTMsavedate{deadline}{2018-11-09}

\date{\normalsize\bf Крайний срок сдачи: \DTMusedate{deadline}}

\begin{document}
\maketitle
\begin{questions}
    \question[1] Системы координат и основные геометрические соотношения.
    \begin{parts}
    \part В геодезии основными параметрами эллипсоида вращения считают полуоси $a$ и $b$, сжатие
        $\alpha$, первый $e$ и второй $e'$ эксцентриситеты. Напишите соотношения между этими
        параметрами.
    \part Покажите на одном рисунке для произвольной точки на поверхности Земли:
        \begin{enumerate}
            \item геодезическую широту и высоту,
            \item геоцентрические широту и радиус--вектор,
            \item приведенную широту $u$ и координату $b$ для системы координат $b$, $u$, $L$.
            \item все возможные параметры эллипсоида, включая линейный эксцентриситет $E$.
        \end{enumerate}
    \end{parts}
        \question[3] Пусть задан уровенный эллипсоид со следующими параметрами
        \begin{align*}
            GM &= 3,986004415 \times 10^{14}\,\textrm{м}^3\textrm{с}^{-2},\\
            a &= 6378136,0 + 0,1\times \left(i - 1\right)\,\textrm{м}, \\
            J_2 &= 0,00108263,\\
            \omega &= 7,292115 \times 10^{-5}\,\textrm{рад}^{-1}, 
        \end{align*}
        где $i$ --- вариант. Вычислите следующие величины:
        \begin{enumerate}
            \item \textbf{Геометрические постоянные}: малую полуось $b$, линейный эксцентриситет $E$, первый эксцентриситет $e$ и его
                квадрат $e^2$, второй эксцентриситет $e'$ и его квадрат $e'^2$, геометрическое
                сжатие $\alpha$ и его обратную величиу $1/\alpha$.
            \item \textbf{Физические постоянные}: значение нормального потенциала силы тяжести на эллипсоиде $U_0$, коэффициенты
                $J_4$, $J_6$, $J_8$, значение нормальной силы тяжести на экваторе $\gamma_e$ и полюсах
                $\gamma_p$, коэффициент $k$ нормальной формулы, гравиметрическое сжатие $\beta$ и
                коэффициент $\beta_1$ приближённой формулы.
        \end{enumerate}
    \question[1] Постройте графики: 
    \begin{parts}
        \part изменения нормальной силы тяжести $\gamma_0$ на эллипсоиде по формуле Сомильяны, 
        \part разности формулы Сомильяны и приближённой формулы с параметрами $\gamma_e$, $\beta$,
        $\beta_1$, 
    \end{parts}
    Как меняется нормальная сила тяжести на эллипсоиде? Какое максимальное отличие точной и
    приближённой формул?
\end{questions}

\newpage
    
\section*{\centering Нормальное поле. Фундаментальные геодезические постоянные}
\subsection*{\centering Теоретическая справка}

В геодезии обычно используют Нормальную Землю --- модель Земли, обладающей теми или иными свойствами
--- в виде идеальной планеты, имеющей форму эллипсоида вращения
\begin{equation*}
    \dfrac{x_0^2 + y_0^2}{a_0^2} + \dfrac{z_0^2}{b_0^2} = 1,
\end{equation*}
где $x_0, y_0, z_0$ --- прямоугольные координаты поверхности эллипсоида, $a_0$, $b_0$ --- большая и
малая полуоси.

Эта поверхность является уровенной относительно потенциала силы тяжести, то есть на поверхности
эллипсоида
\begin{equation*}
    U = U_0 = const.
\end{equation*}

Такой эллипсоид называют уровенным. Использование поля силы
тяжести уровенного эллипсоида в качестве нормального поля удобно
в геодезии потому, что в этом случае одна и та же поверхность --- 
эллипсоид --- является отсчетной при решении и 
геометрических и физических задач.

\subsubsection*{Специальная система координат сжатого эллипсоида вращения}
Потенциал уровенного эллипсоида находят из решения краевой задачи. Её удобно решать в специальных
эллипсоидально--гармонических (гармонические потому, что уравненеи Лапласа разрешимо в них методом
разделения переменных) координатах $b, u, L$, которые связаны с прямоугольными координатами
$x, y, z$ следующими соотношениями 
\begin{align*}
    x &= \sqrt{b^2 + E^2}\cos{u}\cos{L} \\
    y &= \sqrt{b^2 + E^2}\cos{u}\sin{L} \\
    z &= b\cos{u},
\end{align*}
где $b$ --- малая полуось софокусного отсчетному эллипсоида, проходящего через определяемую точку,
$u$ --- приведённая широта, $L=\phi$ --- долгота, $E^2 = a^2 - b^2$ --- линейный эксцентриситет.\par

Далее нам понадобятся коэффициенты Ламе этой системы координат. Полагая $q_1 = b, q_2 = u, q_3 = L$, имеем
\begin{equation}
    h_1 = \dfrac{\sqrt{b^2 + E^2\sin^2{u}}}{a}, \quad h_2 = \sqrt{b^2 + E^2\sin^2 u},\quad
    h_3 = \sqrt{b^2 + E^2\cos{u}}.
    \label{eq:lame}
\end{equation}

\subsubsection*{Внешний потенцал уровенного эллипсоида}

Легко показать, что центробежный потенциал
\begin{equation}
    Q = \dfrac{\omega^2}{2}\left( x^2 + y^2 \right) = \dfrac{\omega^2}{2}r^2\sin^2\theta = 
    \dfrac{\omega^2}{2}\left( b^2 + E^2 \right)\cos^2{u}
    \label{eq:centrifugal}
\end{equation}
не удовлетворяет уравнению Лапласа и не является регулярным на бесконечности. Такими же свойствами
обладает и потенцал силы тяжести, поэтому его нельзя напрямую определить из решения краевой задачи.

Возможность решения внешней краевой задачи по форме уровенной поверхности основана на следующей
теореме.
\begin{theorem*}{Теорема Стокса.}
    Если известна форма внешней уровенной поверхности, масса, заключённая внутри этой поверхности и
    угловая скорость вращения, то внешнее гравитационное поле определено независимо от распределения
    масс внутри поверхности.
    \label{the:stokes}
\end{theorem*}

Краевая задача может быть сформулирована в следующем виде
\begin{equation}
    \quad V_0 = U_0 - Q_0,\quad U_0 = const,\quad \Delta V = 0,\quad \lim\limits_{r\to\infty} V = 0,
    \label{eq:bvp}
\end{equation}
где $V_0 = U_0 - Q_0$ --- краевое условие на поверхности эллипсоида, из которого следует, что потенциал притяжения $V_0$ не
является постоянным на поверхности эллипсоида и зависит, как следует из (\ref{eq:centrifugal}), от
широты. Эта задача --- иногда называемая задачей Стокса --- является частным случаем задачи Дирихле, когда краевая поверхность является
уровенной.

Решением (за подробностями стоит обратиться к литературе \cite{Moritz2007,Ogorodova2006}) краевой
задачи будет следующее выражение

\begin{equation}
    V = \dfrac{GM}{E} \arctg{\dfrac{E}{b}} + \dfrac{\omega^2}{3} \left( b_0^2 + E^2 \right)
    \dfrac{q}{q_0} P_2 \left( \sin{u} \right),
    \label{eq:V}
\end{equation}
где
\begin{equation}
    q = \left[ \left( 3\dfrac{b^2}{E^2} + 1 \right)\arctg{\dfrac{E}{b}} - 3\dfrac{b}{E} \right],
    \label{eq:q}
\end{equation}
а $q_0$ --- значение $q$ на поверхности эллипсоида при $b=b_0$.

На поверхности эллипсоида 
\begin{equation}
    V_0 = \dfrac{GM}{E} \arctg{\dfrac{E}{b_0}} + \dfrac{\omega^2}{3} \left( b_0^2 + E^2 \right)
    P_2 \left( \sin{u} \right).
    \label{eq:V0}
\end{equation}
Добавляя к (\ref{eq:V}) центробежный потенциал (\ref{eq:centrifugal}), получаем внешний потенциал
силы тяжести уровенного эллипсоида вращения
\begin{equation}
    U = \dfrac{GM}{E} \arctg{\dfrac{E}{b}} + 
    \dfrac{\omega^2}{3} \left( b^2 + E^2 \right) +
    \left(\dfrac{\omega^2}{3} \left( b_0^2 + E^2 \right)\dfrac{q}{q_0} -
    \dfrac{\omega^2}{3} \left( b^2 + E^2 \right)
    \right) P_2 \left( \sin{u} \right),
    \label{eq:U}
\end{equation}
где учтено, что $Q = \dfrac{\omega^2}{2}\left( b^2 + E^2 \right)\cos^2{u} =
\dfrac{\omega^2}{3} \left( b^2 + E^2 \right)\left[ 1 - P_2\left( \sin{u} \right) \right]$.

Наконец, на поверхности уровенного эллипсоида при $b = b_0$ получаем выражение достаточно простого вида
\begin{equation}
    U_0 = \dfrac{GM}{E} \arctg{\dfrac{E}{b_0}} + 
    \dfrac{\omega^2}{3} \left( b_0^2 + E^2 \right).
    \label{eq:U0}
\end{equation}


\subsubsection*{Фундаментальные геодезические постоянные}
Из выражений (\ref{eq:V}), (\ref{eq:V0}), (\ref{eq:U}), (\ref{eq:U0}) видно, что внешнее
гравитационное поле уровенного эллипсоида однозначно может быть задано четырьмя параметрами: геоцентрической
гравитационное постоянной $GM$, угловой скоростью вращения $\omega$, линейным
эксцентриситетом $E$ и малой полуосью~$b_0$.

Параметры, однозначно определяющие отсчетное поле уровенного эллипсоида называются
\textbf{фундаментальными геодезическими постоянными}.

Различают первичные и производные фундаментальные постоянные. К первичным относят постоянные,
которые существуют у реальной Земли. Это стоксовы постоянные $GM$, $G\left( C - A \right)$ и угловая
скорость вращения $\omega$. К первичным относят также параметр $J_2 = - C_{2,0} = \dfrac{G\left( C -
A_m\right)}{GMa^2}$.

Для определения нормального поля необходимо также задать четвёртую постоянную, определяющую размер
Нормальной Земли. Этими параметрами могут служить большая полуось $a_0$, значение потенциала на
поверхности эллипсоида $U_0$ или значение нормальной силы тяжести на экваторе $\gamma_e$.

Таким образом, исходными параметрами могут служить $GM$, $a$, $G\left( C - A \right)$ (или $J_2$), 
$\omega$ или $GM$, $U_0$, $G\left( C -
A\right)$ (или $J_2$), $\omega$. Второй набор параметров имеет то принципиальное преимущество, что
все параметры, входящие в него, имеют вполне конкретный физический смысл, если в качестве значения
потенциала $U_0$ на поверхности эллипсоида принять значения реального потенциала силы тяжести $W_0$ на некоторой
уровенной поверхности, совпадающей со средним уровнем Мирового океана --- геоиде. Только результаты
измерений спутниковой альтиметрии последних лет позволили достаточно точно определить $W_0$, 
что открыло перспективу к выводу нового общеземного эллипсоида по параметрам, каждый из которых имеет 
физический смысл.

Тем временем, пока основным набором параметров остается $GM$, $a$, $J_2$, $\omega$. 
Так, например, задается эллипсоид GRS 80. Для удобства часто за исходный выдают 
параметр геометрического сжатия $\alpha$ вместо $J_2$. Однако это не имеет под собой научных
оснований, лишь практические, поскольку большая часть пользователей нуждается в геометрических
приложениях. Таким набором исходных параметров обладают эллипсоиды государственных систем координат 
ГСК--2011 и ПЗ--90.11. Здесь же стоит отметить, что эллипсоид Красовского не является уровенным и
не используется для физических приложений.

Производные (геометрические и физические) фундаментальные постоянные получают для Нормальной Земли,
исходя из соотношений между параметрами уровенного эллипсоида. Далее будем рассматривать случай,
когда в качестве исходных заданы $GM$, $a$, $J_2$, $\omega$.


\subsubsection*{Соотношения между фундаментальными постоянными}

Потенциал притяжения планеты может быть представлен в виде ряда шаровых функций
\begin{equation}
\label{eq:pot-shexp-1}
V\left( r, \theta, \lambda \right) = \sum\limits_{n=0}^{\infty} \dfrac{1}{r^{n+1}}
\sum\limits_{k=0}^{n}\left( A_{nk}\cos{k\lambda} + B_{nk}\sin{k\lambda} \right) P_n^k \left(
\cos\theta \right)
\end{equation}
где $r, \theta, \lambda$ --- сферические координаты, $n$ и $k$ --- степень и порядок,
$P_{nk}\left( \cos{\theta} \right)$ --- присоединённые функции
Лежандра, $A_{nk}, B_{nk}$ --- гармонические коэффициенты, иначе --- \textbf{стоксовы постоянные}.

Уровенный эллипсоид обладает осевой симметрией, следовательно, в разложении (\ref{eq:pot-shexp-1}) 
останутся только зональные гармоники, то есть потенциал не будет зависеть от
долготы. Кроме того, принимая во внимание, что плоскость экватора является плоскостью симметрии, 
то останутся только чётные зональные гармоники. Действительно, поскольку эллипсоид --- симметричное
относительно экватора тело, то и массы его должны быть распределены симметрично. В противном случае,
возникнет <<грушевидность>> и уровенная поверхность перестанет быть эллипсоидом. Таким образом,
записываем разложение потенциала притяжения эллипсоида в ряд по шаровым функциям
\begin{equation}
        \label{eq:pot-shexp-2b}
    V = 
        \sum\limits_{n=0}^{\infty} \dfrac{1}{r^{2n+1}} A_{2\cdot n} 
        P_{2\cdot n} \left(\cos\theta \right).
\end{equation}

Но ранее было записано выражение (\ref{eq:V}) для $V$ в эллипсоидально--гармонических координатах.
Перепишем его еще раз
\begin{equation}
    V = \dfrac{GM}{E} \arctg{\dfrac{E}{b}} + \dfrac{\omega^2}{3} \left( b_0^2 + E^2 \right)
    \dfrac{q}{q_0} P_2 \left( \sin{u} \right).
    \label{eq:V2}
\end{equation}
Теперь можно получить связь выражений (\ref{eq:pot-shexp-2b}) и (\ref{eq:V2}). Для этого разложим
$\arctg{\frac{E}{b}}$ по степеням $E/b$, получим

\begin{equation}
    \arctg{\dfrac{E}{b}} = 
    \dfrac{E}{b} - \dfrac{1}{3}\left( \dfrac{E}{b} \right)^3 + \dfrac{1}{5} \left(\dfrac{E}{b}
    \right)^5 - \dots =
    \sum\limits_{n=0}^{\infty}\dfrac{\left( -1 \right)^n}{2n + 1} \left(\dfrac{E}{b}\right)^{2n+1}.
    \label{eq:arctg-series}
\end{equation}
Также разложим $q$ (см. (\ref{eq:q})) по степеням $\frac{E}{b}$:
\begin{equation}
    q = 2\left[ \dfrac{1}{15} \dfrac{E^3}{b^3} - \dfrac{2}{35}\dfrac{E^5}{b^5} + \dfrac{3}{63}\dfrac{E^7}{b^7} - \dots \right] = 
    2\sum\limits_{n=0}^{\infty} \dfrac{n \left( -1 \right)^{n+1}}{\left( 2n+1 \right)\left( 2n+3
    \right)}\left(\dfrac{E}{b}\right)^{2n+1}.
    \label{eq:q-series}
\end{equation}
Теперь подставим оба разложения в выражение (\ref{eq:V2}), запишем
\begin{equation}
    V = \dfrac{GM}{b} \left( 1 - \dfrac{E^2}{3b^2} + \dots + \dfrac{\omega^2 \left( b_0^2 + E^2
    \right)}{3GMq_0} \left( \dfrac{4}{15}\dfrac{E^3}{b^2} + \dots \right) P_2 \left( \sin{u} \right)
\right).
\label{eq:Vexp}
\end{equation}
Наконец, преобразуем разложение (\ref{eq:pot-shexp-2b}), перейдя к безразмерным коэффициентам и имея ввиду физический смысл стоксовых
постоянных первых степеней: 
\begin{equation}
    V = \dfrac{GM}{r} \left( 1 - \dfrac{G\left( C - A_m \right)}{GM r^2} P_2\left( \cos{\theta}\right) +
        \sum\limits_{n = 2}^{\infty} \dfrac{A_{2\cdot n}}{GMr^{2n}} P_{2\cdot n}\left( \cos{\theta}
    \right) \right).
\label{eq:pot-shexp-new}
\end{equation}

Рассмотрим произвольную точку на оси вращения эллипсоида. В этой точке приведенная широта $u$ равна
геоцентрической $\phi = \theta - 90^\circ$, причем обе они равны $\pi/2$, а радиус--вектор
$r$ совпадает с координатой $b$. Из выражений (\ref{eq:Vexp}) и (\ref{eq:pot-shexp-new}) записываем
\begin{equation}
    \dfrac{GM}{b} \left( 1 - \dfrac{E^2}{3b^2} + \dfrac{\omega^2 \left( b_0^2 + E^2
    \right)}{3GMq_0} \dfrac{4}{15}\dfrac{E^3}{b^2} + \dots \right)
    = \dfrac{GM}{b} \left( 1 - \dfrac{G\left( C - A_m \right)}{GM b^2} + \dots\right)
\end{equation}
Равенство будет выполняться только если будут равны коэффициенты при одинаковых степенях~$b$.
Приравняем коэффициенты при $1/b^3$:
\begin{equation}
   -\dfrac{1}{3}E^2 + \dfrac{\omega^2 \left( b_0^2 + E^2
    \right)E^3}{3GMq_0} \dfrac{4}{15}
    =  - \dfrac{G\left( C - A_m \right)}{GM}
    \label{eq:pizzetti-1}
\end{equation}
или, умножая обе части последнего равенства на $-GM$ и раскрывая $q_0$, получаем знаменитую формулу Пиццетти
\begin{equation}
    G\left( C - A_m \right) = \dfrac{1}{3} GM E^2 - \dfrac{4}{45} \dfrac{\omega^2 \left( b_0^2 +
    E^2\right) E^3}{
    \left( 1 + 3\dfrac{b^2}{E^2} \right)\arctg\dfrac{E}{b} - 3\dfrac{b}{E}},
    \label{eq:pizzetti}
\end{equation}
которая связывает физические и геометрические параметры уровенного эллипсоида.

Вспомним, что $E = ae$ и подставим это выражение в $\ref{eq:pizzetti-1}$, тогда

\begin{equation*}
    e_0^2 - \dfrac{4}{15}\dfrac{\omega^2 a_0^2 e_0^3}{GM q_0} = 3\dfrac{G\left( C - A_m
    \right)}{GM a_0^2}.
\end{equation*}
Введём обозначение $m = \dfrac{\omega^2 a_o^3}{GM}$, $J_2 = \dfrac{G\left( C - A_m
\right)}{GM a_0^2}$, тогда
\begin{equation}
    e_0^2 - \dfrac{4}{15}m\dfrac{e_0^3}{q_0} = 3J_2\quad\Longrightarrow\quad
    J_2 = \dfrac{e_0^2}{3} - \dfrac{4}{45}m\dfrac{e_0^3}{q_0}.
    \label{eq:j2e0}
\end{equation}
Мы получили еще одну формулу, связывающую физические и геометрические параметры. Все входящие
в неё величины безразмерны. Её можно было бы использовать, если бы исходными параметрами были
$a$ и $\alpha$. Тогда, получив коэффициент $J_2$, можно получить коэффициенты $J_{2n}$
(см. \ref{eq:jn}).
Если в качестве исходных для уровенного эллипсоида заданы $J_2$ и $a$, то выражение
\begin{equation}
    e_0^2 = 3J_2  + \dfrac{4}{15}m\dfrac{e_0^3}{q_0},
    \label{eq:j2e0-2}
\end{equation}
вытекающее из ($\ref{eq:j2e0}$), используется для определения первого эксцентриситета и далее ---
геометрического сжатия. Поскольку $e_0$ здесь стоит в обеих частях уравнения, то его решают
итерациями относительно $e_0^2$
\begin{equation}
    \left( e_0^2 \right)_{i} = 3J_2  + \dfrac{4}{15}m\left( \dfrac{e_0^3}{q_0} \right)_{i-1},
    \label{eq:j2e0-3}
\end{equation}
принимая в первом приближении $e_0^2 = 3J_2 + m$ (обратите внимание, что в $q_0$ также
входит эксцентриситет). Итерационное решение можно построить встроенными 
функциями электронных таблиц MS Excel и его аналогами.

\subsubsection*{Разложение потенциала эллипсоида в ряд по шаровым функциям}
Если, в соответствии с рекомендацией Международного астрономического союза, для зональных гармоник
$A_{n0}$ ввести следующее обозначение 
\begin{equation*}
    J_n = -C_{n0} = -\dfrac{A_{n0}}{GMa^n},
\end{equation*}
то из выражения (\ref{eq:pot-shexp-1}) можно получить
\begin{equation}
    V  = \dfrac{GM}{r}   
        \left[ 1 - \sum\limits_{n=1}^{\infty} \left( \dfrac{a}{r} \right)^{2\cdot n}
        J_{2\cdot n} P_{2\cdot n} \left( \cos\theta \right) \right].
        \label{eq:pot-shexp-2c}
\end{equation}

Здесь бесконечное число членов, но, как уже было выяснено ранее, потенциал эллипсоида может быть
однозначно задан четырьмя фундаментальными постоянными, поэтому в (\ref{eq:pot-shexp-2c}) остальные
члены не могут быть независимыми и их можно выразить через $J_2$ (подробности в \cite{Moritz2007, Pellinen1978,
Ogorodova2006}):
\begin{equation}
    J_{2\cdot n} = \left( -1 \right)^{n + 1} \dfrac{3e^{2n-2}}{\left( 2n+1 \right)\left( 2n+3 \right)}
    \left[ 5nJ_2 - \left( n - 1 \right) e^2 \right].
    \label{eq:jn}
\end{equation}
Ряд ($\ref{eq:pot-shexp-2c}$) особенно полезен, когда действительный потенциал притяжения представлен в виде
ряда по шаровым функциям. Тогда, вычитая из него притяжение уровенного эллипсоида, можно получить
коэффициенты разложения аномального потенциала в ряд по шаровым функциям.

\subsubsection*{Сила тяжести на поверхности уровеннного эллипсоида}
В геодезии и геофизике наиболее часто используемой величиной, связанной с нормальным полем, является
нормальная сила тяжести $\gamma$. Согласно определению потенциала, сила является его производной.
Для того, чтобы найти нормальную силу тяжести $\gamma_0$ на эллипсоиде, необходимо найти производную
потенциала $U$ по внешней нормали $n$ к уровенной поверхности:
\begin{equation*}
    \gamma_0 = \left. -\dfrac{\partial U}{\partial n} \right|_{b = b_0}.
\end{equation*}
В системе координат $b, u, L$ элемент нормали к эллипсоиду имеет вид $dn = h_1 db$, поэтому,
используя выражении для $h_1$ из (\ref{eq:lame}), можно записать
\begin{equation}
    \gamma_0 =  \left. -\dfrac{1}{h_1}\dfrac{\partial U}{\partial b} \right|_{b = b_0} =
        \left. \dfrac{a_0}{\sqrt{b_0^2 + E^2\sin^2{u}}} \dpd{U}{b} \right|_{b=b_0}.
    \label{eq:gamma-0}
\end{equation}

После дифференцирования (\ref{eq:U}), получим
\begin{equation}
    \gamma_0 = \dfrac{1}{\sqrt{b_0^2 + E^2\sin^2{u}}}
    \left( 
        \dfrac{GM}{a_0} - \dfrac{2}{3}\omega^2 a_0 b_0 +
        \dfrac{4}{3}\omega^2 a_0 E
        \dfrac{1 - \dfrac{b_0}{E}\arctg{\dfrac{E}{b_0}}}{
        \left( 1 + 3\dfrac{b_0^2}{E^2} \right)\arctg{\dfrac{E}{b_0}} - 3\dfrac{b_0}{E}}
        P_2 \left( \sin{u} \right)
    \right).
    \label{eq:gamma0}
\end{equation}

По этой формуле при $u = 0$ получаем значение силы тяжести на экваторе уровенного эллипсоида
\begin{equation}
    \gamma_e = 
        \dfrac{GM}{a_0 b_0} - \dfrac{2}{3}\omega^2 a_0 -
        \dfrac{2}{3}\dfrac{\omega^2 a_0 E}{b_0}
        \dfrac{1 - \dfrac{b_0}{E}\arctg{\dfrac{E}{b_0}}}{
        \underbrace{\left( 1 + 3\dfrac{b_0^2}{E^2} \right)\arctg{\dfrac{E}{b_0}} -
    3\dfrac{b_0}{E}}_{\textrm{Обратите внимание: } q_0}},
    \label{eq:gamma_e}
\end{equation}
при $u = \pi/2$ получаем значение силы тяжести на полюсе
\begin{equation}
    \gamma_p = 
        \dfrac{GM}{a_0^2} - \dfrac{2}{3}\omega^2 b_0 +
        \dfrac{4}{3}\omega^2 E
        \dfrac{1 - \dfrac{b_0}{E}\arctg{\dfrac{E}{b_0}}}{
        \left( 1 + 3\dfrac{b_0^2}{E^2} \right)\arctg{\dfrac{E}{b_0}} - 3\dfrac{b_0}{E}}.
    \label{eq:gamma_p}
\end{equation}

Выражение (\ref{eq:gamma0}) можно привести к значительно более удобному для вычислений виду, запишем
без вывода
\begin{equation}
    \gamma_0 = \dfrac{\gamma_e a_0 \cos^2{B} + \gamma_p b_0 \sin^2{B}}{
    \sqrt{a_0^2\cos^2{B} + b_0^2\sin^2{B}}}.
    \label{eq:somgliana}
\end{equation}
Это известная формула Сомильяны, позволяющая вычислить значение нормальной силы тяжести в
любой точке на поверхности эллипсоида по геодезической широте $B$. Формулу Сомильяны часто записывают в виде
\begin{equation}
    \gamma_0 = \gamma_e\dfrac{1 + k\sin^2{B}}{\sqrt{1 - e^2\sin^2{B}}}, \qquad\textrm{где}\qquad
    k = \dfrac{b_0\gamma_p - a_0\gamma_e}{a_0\gamma_e}.
    \label{eq:somgliana-2}
\end{equation}

В не очень точных приложениях часто используют приближённую формулу, которую можно получить из
формулы Сомильяны, разложив её в ряд. Опуская вывод, запишем
\begin{equation}
    \gamma_0 = \gamma_e \left( 1 + \beta\sin^2{B} - \beta_1\sin^2{2B}  + \dots \right),
    \label{eq:gamma-series}
\end{equation}
где $\beta_1 = \frac{1}{8} \alpha^2 + \frac{1}{4}\alpha\beta$, а
\begin{equation}
    \beta = \dfrac{\gamma_p - \gamma_e}{\gamma_e}
    \label{eq:gravity-f}
\end{equation}
--- гравитационное сжатие.

Такой вид, например, имеет нормальная формула Гельмерта
\begin{equation}
    \gamma_0 = 978030 \left( 1 + 0,005302\sin^2{B} - 0,000007\sin^2{2B} \right)\quad
    \textrm{мГал},
    \label{eq:Helmert}
\end{equation}
до сих пор часто используемая для
вычисления аномалий силы тяжести и составления гравиметрических карт у нас в стране.

\printbibliography

\end{document}
