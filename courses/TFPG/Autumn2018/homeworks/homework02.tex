\documentclass[11pt, a4paper,addpoints]{exam}

% Languages and fonts
\usepackage{cmap} 
\usepackage[T2A]{fontenc}
\usepackage[utf8]{inputenc} 
\usepackage[english, russian]{babel}
\usepackage{microtype}
\usepackage{indentfirst}
\frenchspacing

% Mathematics
\usepackage{amsmath, amssymb, amsfonts, amsthm, mathtools, fixmath}
\mathtoolsset{showonlyrefs=true}
\usepackage{esint, esvect} % integrals and vectors
\usepackage{systeme} % equation system
\usepackage{commath} % partials and differentials
\usepackage{icomma} % smart comma ($0,2$ is a number)

% Floats
\usepackage{float}

% Tables
\usepackage{array,tabularx,tabulary,booktabs} % better tables
\usepackage{longtable}
\usepackage{multirow}

% Graphics
\usepackage[pdftex]{graphicx}
\usepackage{wrapfig}

% Theorems
\renewcommand{\proofname}{Доказательство}

%\theoremstyle{plain}
\newtheorem{theorem}{Теорема}[section]

%\theoremstyle{definition}
\newtheorem{definition}{Определение}
\newtheorem{corollary}{Следствие}[theorem]

\theoremstyle{remark}
\newtheorem{remark}{Замечание}

\usepackage[top=20mm,bottom=20mm,left=20mm,right=20mm]{geometry}

\usepackage{soul}
\usepackage{enumerate} % better numbered lists
\usepackage{hyperref}
\usepackage{xcolor}
\usepackage{tikz} % drawing

%\usepackage{csquotes}
%\usepackage[style=authoryear,maxcitenames=2,backend=biber,sorting=nty]{biblatex}
%\bibliography{}

\renewcommand{\epsilon}{\ensuremath{\varepsilon}}
\renewcommand{\phi}{\ensuremath{\varphi}}
\renewcommand{\theta}{\vartheta}
\renewcommand{\kappa}{\ensuremath{\varkappa}}
\renewcommand{\le}{\ensuremath{\leqslant}}
\renewcommand{\leq}{\ensuremath{\leqslant}}
\renewcommand{\ge}{\ensuremath{\geqslant}}
\renewcommand{\geq}{\ensuremath{\geqslant}}

\usepackage[useregional]{datetime2}

% exam
\pointsinrightmargin
\marginpointname{ б.}

% custom maketitle
\usepackage{titling}
\setlength{\droptitle}{-4em}
\posttitle{\end{center}\vspace{-4em}}

\title{{\Large Теория фигур планет и гравиметрия 2018}\\ 
    {\bf\Large Домашнее задание № 2}}
\author{}
\DTMsavedate{deadline}{2018-10-05}

\date{\normalsize\bf Крайний срок сдачи: \DTMusedate{deadline}}

\begin{document}
\maketitle
\thispagestyle{empty}
\begin{questions}
    \question[2] Составить полиномы Лежандра $P_n \left( x \right)$ указанных степеней $n$.
    Нормировать эти полиномы. Убедиться в их нормированности и ортогональности на отрезке $\left[
    -1, 1 \right]$.
    \begin{enumerate}
        \item $n = 4$, $n = 5$.
        \item $n = 4$, $n = 6$.
        \item $n = 5$, $n = 6$.
        \item $n = 5$, $n = 7$.
        \item $n = 6$, $n = 7$.
        \item $n = 6$, $n = 8$.
        \item $n = 6$, $n = 9$.
        \item $n = 4$, $n = 7$.
    \end{enumerate}
    \question[1] Нарисовать графики полиномов Лежандра степеней $n = 0, 1, 2, 3$, а также указанных
    степеней из предыдущей задачи.
    \question[2] Разложить в ряд по полиномам Лежандра на отрезке $\left[ -1, 1 \right]$ указанную
    функцию (получить первые три коэффициента ряда).
    \begin{enumerate}
        \item $y = \arcisn{\frac{x}{2}}$.
        \item $y = |x|$.
        \item $y = 2^x$.
        \item $y = \ln{\left( x + 2 \right)}$.
        \item $y = |2x + 3|$.
        \item $y = \arcctg{x}$.
        \item $y = \arccos{\frac{x}{2}}$.
        \item $y = \arctg{x}$.
    \end{enumerate}
    \end{questions}
%\printbibliography
\end{document}
