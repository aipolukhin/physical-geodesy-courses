\documentclass[11pt, a4paper,addpoints]{exam}

% Languages and fonts
\usepackage{cmap} 
\usepackage[T2A]{fontenc}
\usepackage[utf8]{inputenc} 
\usepackage[english, russian]{babel}
\usepackage{microtype}
\usepackage{indentfirst}
\frenchspacing

% Mathematics
\usepackage{amsmath, amssymb, amsfonts, amsthm, mathtools, fixmath}
%\mathtoolsset{showonlyrefs=true}
\usepackage{esint, esvect} % integrals and vectors
\usepackage{systeme} % equation system
\usepackage{commath} % partials and differentials
\usepackage{icomma} % smart comma ($0,2$ is a number)

% Floats
\usepackage{float}

% Tables
\usepackage{array,tabularx,tabulary,booktabs} % better tables
\usepackage{longtable}
\usepackage{multirow}

% Graphics
\usepackage[pdftex]{graphicx}
\usepackage{wrapfig}

% Theorems
\renewcommand{\proofname}{Доказательство}

%\theoremstyle{plain}
\newtheorem{theorem}{Теорема}[section]
\newtheorem*{theorem*}{Теорема}

%\theoremstyle{definition}
\newtheorem{definition}{Определение}
\newtheorem{corollary}{Следствие}[theorem]

\theoremstyle{remark}
\newtheorem{remark}{Замечание}

\usepackage[top=20mm,bottom=20mm,left=20mm,right=20mm]{geometry}

\usepackage{soul}
\usepackage{enumerate} % better numbered lists
\usepackage{hyperref}
\usepackage{xcolor}
\usepackage{tikz} % drawing

\usepackage{csquotes}
\usepackage[style=numeric,maxcitenames=2,backend=biber,sorting=nty]{biblatex}
\bibliography{../../../../bibliography}

\footer{}{\thepage}{}
{}

\renewcommand{\epsilon}{\ensuremath{\varepsilon}}
\renewcommand{\phi}{\ensuremath{\varphi}}
\renewcommand{\theta}{\vartheta}
\renewcommand{\kappa}{\ensuremath{\varkappa}}
\renewcommand{\le}{\ensuremath{\leqslant}}
\renewcommand{\leq}{\ensuremath{\leqslant}}
\renewcommand{\ge}{\ensuremath{\geqslant}}
\renewcommand{\geq}{\ensuremath{\geqslant}}

\usepackage[useregional]{datetime2}

% exam
\pointsinrightmargin
\marginpointname{ б.}

% custom maketitle
\usepackage{titling}
\setlength{\droptitle}{-4em}
\posttitle{\end{center}\vspace{-4em}}

\title{{\Large Теория фигур планет и гравиметрия 2018}\\ 
    {\bf\Large Домашнее задание № 7}}
\author{}
\DTMsavedate{deadline}{2018-11-16}

\date{\normalsize\bf Крайний срок сдачи: \DTMusedate{deadline}}

\begin{document}
\maketitle
\begin{questions}
    \question[3] Принять в качестве исходных данных глобальную модель гравитационного поля Земли и
    координаты точки из домашнего задания №5, а также параметры уровенного эллипсоида~GRS~80.
    Выполнить следующее.
    \begin{parts}
        \part Написать в явном виде разложение нормального потенциала притяжения
        в ряд по шаровым функциям до $n = 4$ и вычислить его значение 
        в точке на территории МИИГаиК.
        \part Вычислить значения коэффициентов разложения аномального потенциала в ряд по шаровым
        функциям до 4--го порядка и вычислить значение аномального потенциала в точке на территории
        МИИГАиК по полученным коэффициентам.
        \part Вычислить значение аномалии высоты для точки на территории МИИГаиК. Считать, что
        значение нормального потенциала силы тяжести на эллипсоиде равно действительному
        значению потенциала силы тяжести в начале счёта высот. Нормальную силу тяжести в точке на
        высоте $H$ над поверхностью эллипсоида с достаточной точностью можно вычислить так
        \begin{equation*}
            \gamma = \gamma_0 + \dpd{\gamma}{H} H = \gamma_0 - 0,3086 H,
        \end{equation*}
        где $\gamma_0$ --- значение нормальной силы тяжести на эллипсоиде, 
        $\pd{\gamma}{H} = -0,3086\,\textrm{мГал/м}$ ---
        вертикальный градиент нормальной силы тяжести на эллипсоиде.
    \end{parts}
    \question[2] Вычислить аномалию высоты в пяти равномерно расположенных точках над эллипсоидом на
    высотах от 0 до 10 км и построить график изменения аномалии высоты с высотой. 
    \begin{parts}
        \part Почему аномалия высоты ведет себя таким образом вблизи поверхности Земли? 
        \part Изменится ли поведение аномалии высоты при дальнейшем увеличении расстояния? Почему?
        \part Одинаково ли будут меняться геодезические и нормальные высоты при удалении от Земли? 
        \part Как меняется высота квазигеоида и будет ли меняться геоид?   
    \end{parts}
    
\end{questions}

%\printbibliography

\end{document}
