\documentclass[11pt, a4paper,addpoints]{exam}

% Languages and fonts
\usepackage{cmap} 
\usepackage[T2A]{fontenc}
\usepackage[utf8]{inputenc} 
\usepackage[english, russian]{babel}
\usepackage{microtype}
\usepackage{indentfirst}
\frenchspacing

% Mathematics
\usepackage{amsmath, amssymb, amsfonts, amsthm, mathtools, fixmath}
\mathtoolsset{showonlyrefs=true}
\usepackage{esint, esvect} % integrals and vectors
\usepackage{systeme} % equation system
\usepackage{commath} % partials and differentials
\usepackage{icomma} % smart comma ($0,2$ is a number)

% Floats
\usepackage{float}

% Tables
\usepackage{array,tabularx,tabulary,booktabs} % better tables
\usepackage{longtable}
\usepackage{multirow}

% Graphics
\usepackage[pdftex]{graphicx}
\usepackage{wrapfig}

% Theorems
\renewcommand{\proofname}{Доказательство}

%\theoremstyle{plain}
\newtheorem{theorem}{Теорема}[section]

%\theoremstyle{definition}
\newtheorem{definition}{Определение}
\newtheorem{corollary}{Следствие}[theorem]

\theoremstyle{remark}
\newtheorem{remark}{Замечание}

\usepackage[top=20mm,bottom=20mm,left=20mm,right=20mm]{geometry}

\usepackage{soul}
\usepackage{enumerate} % better numbered lists
\usepackage{hyperref}
\usepackage{xcolor}
\usepackage{tikz} % drawing

%\usepackage{csquotes}
%\usepackage[style=authoryear,maxcitenames=2,backend=biber,sorting=nty]{biblatex}
%\bibliography{}

\renewcommand{\epsilon}{\ensuremath{\varepsilon}}
\renewcommand{\phi}{\ensuremath{\varphi}}
\renewcommand{\theta}{\vartheta}
\renewcommand{\kappa}{\ensuremath{\varkappa}}
\renewcommand{\le}{\ensuremath{\leqslant}}
\renewcommand{\leq}{\ensuremath{\leqslant}}
\renewcommand{\ge}{\ensuremath{\geqslant}}
\renewcommand{\geq}{\ensuremath{\geqslant}}

\usepackage[useregional]{datetime2}

% exam
\pointsinrightmargin
\marginpointname{ б.}

% custom maketitle
\usepackage{titling}
\setlength{\droptitle}{-4em}
\posttitle{\end{center}\vspace{-4em}}

\title{{\Large Теория фигур планет и гравиметрия 2018}\\ 
    {\bf\Large Домашнее задание № 1}}
\author{}
\DTMsavedate{lessondate}{2018-02-11}
\DTMsavedate{deadline}{2018-02-25}

\date{\normalsize\bf Крайний срок сдачи: \DTMusedate{deadline}}

\begin{document}
\maketitle
\thispagestyle{empty}
\begin{questions}
    \question[1] Ответьте на вопросы.
    \begin{parts}
        \part Какой диапазон изменения высот физической поверхности Земли и глубин дна Мирового
        океана? Где находятся максимум и минимум?
        \part Как относятся эти величины (диапазон, минимум и максимум) к среднему радиусу Земли? Выразите численно.
        \part Вы стоите во дворе университета. Назовите как минимум три любых независимых способа
        определения своего местоположения (можно пользоваться любыми средствами и инструментами). В
        какой системе координат будет результат и с какой точностью (примерно)?
        \part Как по--русски называются эти буквы греческого алфавита:\\
        $\alpha$, $\beta$, $\gamma$, $\delta\Delta$, $\zeta$, $\eta$, $\theta$, $\lambda\Lambda$, 
        $\mu$, $\nu$, $\xi$, $\pi$, $\rho$, $\phi$, $\chi$, $\psi$, $\omega$?
    \end{parts}
    \question[1] Постройте график зависимости геоцентрической широты и приведённой широты
    от геодезической широты, если последняя изменятеся от $-90^\circ$ до $90^\circ$ на
    поверхности эллипсоида. Найдите максимум и минимумы этих разностей. За исходные принять параметры общеземного эллипсоида ГСК--2011 
    ($a = 6378136,5\ \text{м}$, $\alpha~=~1/298,2564151$).

    \question[1] Высота Эвереста (Гималаи) 8848 м над уровнем моря ($B = 28^\circ, L = 87^\circ$), 
    высота вулкана Чимборасо (Анды) 6267 м ($B = -1,5^\circ, L = -79^\circ$). Чья вершина находится дальше от
    центра масс Земли? 
    Сделайте необходимые расчёты, приняв за исходные параметры общеземного эллипсоида из предыдущей
    задачи. Различиями в системах высот пренебречь. Объясните полученный результат.
    \question[2] Составьте (или найдите) для себя таблицу простейших производных и интегралов (первообразных). Вспомните основные
    правила дифференцирования и интегрирования. Решите примеры.
    \begin{parts}
        \part Найти производную функций ($a$ и $n$ -- числа)
        \begin{subparts}
            \subpart $y = x + \sqrt{x} + \sqrt[3]{x}$,
            \subpart $y = \dfrac{1}{x} + \dfrac{1}{\sqrt{x}} + \dfrac{1}{\sqrt[3]{x}}$,
            \subpart $y = \sin^n{x}\cdot\cos{nx}$,
            \subpart $y = \ln\tg{\dfrac{x}{2}}$,
            \subpart $y = \dfrac{a}{x^n}$, найти $y'''$.
        \end{subparts}
        \part Найти все частные производные первого и второго порядков для функции
        $f\left( x, y \right) = \dfrac{x}{y}$.
        \part Разложить в ряд Тейлора функцию из предыдущего примера в окрестности точки $M(1, 1)$.
        \part Найти интегралы
        \begin{subparts}
            \subpart $\int\left( \dfrac{a}{x} + \dfrac{a^2}{x^2} + \dfrac{a^3}{x^3} \right)\dif x$,
            \subpart $\int\left( e^{3x} + 2\sin{2x} \right)\dif x$,
            \subpart $\int\limits_{0}^{1}\sqrt{x^3}\dif x$.
        \end{subparts}
    \end{parts}
\end{questions}
%\printbibliography
\end{document}
