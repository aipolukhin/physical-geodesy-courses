\documentclass[11pt, a4paper,addpoints]{exam}

% Languages and fonts
\usepackage{cmap} 
\usepackage[T2A]{fontenc}
\usepackage[utf8]{inputenc} 
\usepackage[english, russian]{babel}
\usepackage{microtype}
\usepackage{indentfirst}
\frenchspacing

% Mathematics
\usepackage{amsmath, amssymb, amsfonts, amsthm, mathtools, fixmath}
\mathtoolsset{showonlyrefs=true}
\usepackage{esint, esvect} % integrals and vectors
\usepackage{systeme} % equation system
\usepackage{commath} % partials and differentials
\usepackage{icomma} % smart comma ($0,2$ is a number)

% Floats
\usepackage{float}

% Tables
\usepackage{array,tabularx,tabulary,booktabs} % better tables
\usepackage{longtable}
\usepackage{multirow}

% Graphics
\usepackage[pdftex]{graphicx}
\usepackage{wrapfig}

% Theorems
\renewcommand{\proofname}{Доказательство}

%\theoremstyle{plain}
\newtheorem{theorem}{Теорема}[section]

%\theoremstyle{definition}
\newtheorem{definition}{Определение}
\newtheorem{corollary}{Следствие}[theorem]

\theoremstyle{remark}
\newtheorem{remark}{Замечание}

\usepackage[top=20mm,bottom=20mm,left=20mm,right=20mm]{geometry}

\usepackage{soul}
\usepackage{enumerate} % better numbered lists
\usepackage{hyperref}
\usepackage{xcolor}
\usepackage{tikz} % drawing

%\usepackage{csquotes}
%\usepackage[style=authoryear,maxcitenames=2,backend=biber,sorting=nty]{biblatex}
%\bibliography{}

\renewcommand{\epsilon}{\ensuremath{\varepsilon}}
\renewcommand{\phi}{\ensuremath{\varphi}}
\renewcommand{\theta}{\vartheta}
\renewcommand{\kappa}{\ensuremath{\varkappa}}
\renewcommand{\le}{\ensuremath{\leqslant}}
\renewcommand{\leq}{\ensuremath{\leqslant}}
\renewcommand{\ge}{\ensuremath{\geqslant}}
\renewcommand{\geq}{\ensuremath{\geqslant}}

\usepackage[useregional]{datetime2}

% exam
\pointsinrightmargin
\marginpointname{ б.}

% custom maketitle
\usepackage{titling}
\setlength{\droptitle}{-4em}
\posttitle{\end{center}\vspace{-4em}}

\title{{\Large Геодезическая гравиметрия 2018}\\ 
    {\bf\Large Домашнее задание № 6}}
\author{}
\DTMsavedate{deadline}{2018-12-01}

\date{\normalsize\bf Крайний срок сдачи: \DTMusedate{deadline}}

\begin{document}
\maketitle
\thispagestyle{empty}
\begin{questions}
        \question Пусть потенциал притяжения представлен в видя ряда по шаровым функциям
        \begin{equation}
        V \left( r, \phi, \lambda \right) = \dfrac{GM}{r}   
            \sum\limits_{n=0}^{\infty} \left( \dfrac{a}{r} \right)^n
        \sum\limits_{k=0}^{n} \left(  
        \bar{C}_{nk}\cos{k\lambda} + \bar{S}_{nk}\sin{k\lambda}
        \right) \bar{P}_{nk} \left( \sin{\phi} \right),
    \end{equation}
    где $r, \phi, \lambda$ --- сферические координаты, $n$ и $k$ --- степень и порядок,
    $\bar{C}_{nk},\bar{S}_{nk}$ --- полностью нормированные стоксовы постоянные, 
    $\bar{P}_{nk}\left( \sin{\phi} \right)$ --- полностью нормированные присоединённые функции
    Лежандра, $GM$ --- планетоцентрическая гравитационная постоянная.
        \begin{parts}
        \part Написать в общем виде выражение для частной производной $\dpd{V}{r}$.
        \part Написать в явном виде разложение силы тяжести в ряд по шаровым функциям до
        4--го порядка в радиальном приближении ($|\vec{g}| = |\nabla W | \approx |\partial W /
        \partial r$|). 
        \part Вычислить значение потенциала притяжения, потенциала силы тяжести и силы тяжести
        (в радиальном приближении) по глобальной модели гравитационного поля Земли, ограничившись первыми четырьмя степенями
        разложения ($n_{max} = 4$), для точки на территории МИИГАиК с координатами (эллипсоид $WGS84$):
        \begin{equation*}
            B = 55,764058^\circ,\quad L = 37,661425^\circ,\quad H = 158,064\,\textrm{м}.
        \end{equation*}
    \end{parts}
\end{questions}
   \section*{\centering Получение модели}
    Глобальные модели гравитационного поля можно получить с сайта Международного центра для глобальных
    моделей (ICGEM). Для этого необходимо сделать следующее.
    \begin{enumerate}
        \item Зайти в раздел Static Models (статические модели) по адресу:\\
            \url{http://icgem.gfz-potsdam.de/tom_longtime}.
        \item Из списка найти модель с номером $Nr = 167 - i$, где $i$ --- номер варианта.
        \item Скачать файл модели с расширением *.gfc.
        \item В заголовке файла: \\
            earth\_gravity\_constant -- геоцентрическая гравитационная постоянная, \\
            radius --- значение $a$, которое использовалось для получения безразмерных
            коэффициентов, \\
            norm --- норма коэффициентов (необходимо убедиться, что здесь стоит fully\_normalized
            --- полностью нормированные).
    \end{enumerate}

%\printbibliography
\end{document}
