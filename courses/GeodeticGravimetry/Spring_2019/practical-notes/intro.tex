\documentclass[11pt, a4paper]{article}

% For scalable fonts - install cm-super package

% Languages and fonts
\usepackage{cmap} 
\usepackage[T2A]{fontenc}
\usepackage[utf8]{inputenc} 
\usepackage[english, russian]{babel}
\usepackage{microtype}
\usepackage{indentfirst}
\frenchspacing

% Mathematics
\usepackage{amsmath, amssymb, amsfonts, amsthm, mathtools, fixmath}
\mathtoolsset{showonlyrefs=true}
\usepackage{esint, esvect} % integrals and vectors
\usepackage{systeme} % equation system
\usepackage{commath} % partials and differentials
\usepackage{icomma} % smart comma ($0,2$ is a number)

% Floats
\usepackage{float}

% Tables
\usepackage{array,tabularx,tabulary,booktabs} % better tables
\usepackage{longtable}
\usepackage{multirow}

% Graphics
\usepackage[pdftex]{graphicx}
\usepackage{wrapfig}

% Theorems
\renewcommand{\proofname}{Доказательство}

\theoremstyle{plain}
\newtheorem{theorem}{Теорема}[section]

\theoremstyle{definition}
\newtheorem{definition}{Определение}
\newtheorem{corollary}{Следствие}[theorem]
\newtheorem{problem}{Задача}[section]

\theoremstyle{remark}
\newtheorem{remark}{Замечание}
\newtheorem*{solution}{Решение}

\usepackage[top=20mm,bottom=20mm,left=20mm,right=20mm]{geometry}

\usepackage{lastpage} % how many pages

\usepackage{soul}

\usepackage{framed} % easy frames
\usepackage{enumerate} % better numbered lists

\usepackage{hyperref}
\usepackage{xcolor}

\usepackage{tikz} % drawing

\usepackage{csquotes}
% If a bibliography is not shown and you use TexMaker, please change in Options > Configure TeXstudio > Build > Bib(la)tex value from "bibtex %" to "biber %"
\usepackage[style=numeric,backend=biber,sorting=none]{biblatex}
\addbibresource{../../../../bibliography.bib}

\renewcommand{\epsilon}{\ensuremath{\varepsilon}}
\renewcommand{\phi}{\ensuremath{\varphi}}
\renewcommand{\theta}{\vartheta}
\renewcommand{\kappa}{\ensuremath{\varkappa}}
\renewcommand{\le}{\ensuremath{\leqslant}}
\renewcommand{\leq}{\ensuremath{\leqslant}}
\renewcommand{\ge}{\ensuremath{\geqslant}}
\renewcommand{\geq}{\ensuremath{\geqslant}}

\usepackage[useregional]{datetime2}

% custom maketitle
\usepackage{titling}
\setlength{\droptitle}{-4em}
\posttitle{\end{center}\vspace{-3em}}

\title{{Геодезическая гравиметрия 2019}\\ 
{\Large Весенний семестр}}
\author{}
\DTMsavedate{lessondate}{2019-02-11}
\date{\DTMusedate{lessondate}}

\begin{document}
\maketitle

Курс <<Геодезическая гравиметрия>> читается на третьем (весенний семестр) и четвёртом (осенний семестр) курсах для студентов геодезического факультета МИИГАиК направления «Геодезия и дистанционное зондирование» (бакалавриат).

Весенний семестр посвящен изучению поля тяготения и его свойств, а также измерениям силы тяжести на поверхности Земли. 
Осенний семестр будет включать в себя вопросы моделирования гравитационного поля Земли и решение геодезических задач с использованием информации о гравитационном поле Земли.

\section{Контакты}
Преподаватель: Алексей Игоревич Полухин

Telegram-канал:
\fcolorbox{cyan}{white}{\href{https://t.me/gravimetry_course_2019}{gravimetry\_course\_2019}}

Почта: \href{mailto:aip@geod.ru}{aip@geod.ru}

\section{Примерная программа практических занятий}
\begin{enumerate}
    \item Введение. Краткие сведения из математики и высшей геодезии.
    \item Притяжение. Основные понятия и свойства.
    \item Притяжение тел простой формы I.
    \item Притяжение тел простой формы II.
    \item Притяжение тел сложной формы. Гармонические функции.
    \item Гравитационное поле Земли и планет. Общая характеристика.
    \item Гравитационное поле Земли и планет. Изменение гравитационного поля во времени.
    \item Наземные методы и средства измерений. Абсолютные измерения силы тяжести.
    \item Статический метод измерения силы тяжести.
    \item Исследования статических гравиметров I.
    \item Исследования статических гравиметров II.
    \item Метрология. Сравнения и эталонирование гравиметров.
    \item Гравиметрический рейс.
    \item Обработка гравиметрического рейса. Гравиметрические сети.
\end{enumerate}

\section{Контроль знаний и выставление оценок}
В курсе (весенний семестр) предусмотрены следующие формы контроля знаний: 
\begin{itemize}
    \item лабораторные работы,
    \item домашние задания,
    \item самостоятельные работы,
    \item контрольные работы,
    \item зачёт.
\end{itemize}

На практических занятиях будут разбираться основные понятия для закрепления теоретического материала
лекций, а также будут решаться и разбираться простейшие и/или типовые примеры и задачи.
\textbf{Только} на
практических занятиях будут выполняться лабораторные работы с гравиметрами и разбираться отдельные
темы по разделу курса <<гравиметрия>>. Пропуски занятий с гравиметрами не допускаются, ибо в связи с
большим числом студентов и ограниченным числом преподавателей, у нас нет возможности заниматься с
вами вне аудиторных часов. Лабораторные работы с гравиметрами должны быть аккуратно оформлены и
защищены. Оцениваются работы по двоичной системе (зачёт/незачёт).

\subsection{Домашние задания}
Домашние задания (ДЗ) будут выдаваться после (почти) каждого практического занятия и будут
состоять из контрольных вопросов, обязательных типовых задач, а также дополнительных задач
повышенной сложности. Каждый вопрос и каждая задача в задании будут иметь свою «стоимость» в баллах.
Общая оценка за одно домашнее задание равна сумме баллов за все вопросы, примеры и задачи.
Максимальное число баллов за каждое домашнее задание --- $5,0$, минимальное --- $0,0$. Баллы, заработанные за решение задач
повышенной сложности могут быть зачтены в другие задания и формы контроля. Все домашние задания
должны быть защищены, что включает в себя несколько контрольных вопросов по теме и/или ходу решения.
Незащищённые задания не могут быть зачтены. 

Крайний срок сдачи --- две недели с момента выдачи
задания. После дедлайна домашние задания не принимаются и могут полностью войти в программу зачёта
для несдавшего студента. 

Задания крайне рекомендуется высылать в электронном виде вне занятий, но также остаётся возможность сдавать их в рукописном/печатном виде в часы занятий. При сдаче в электронном виде работа
высылается в формате PDF и редактируемом формате (\LaTeX («латех»), ODT, DOC, DOCX)
вместе со всеми вычислительными материалами (ODS, XLS, XLSX или иное), что позволит
проверить ход вычислений. Работы следует оформлять в соответствии с Правилами оформления работ.

Дополнительный балл может быть заработан, если домашнее задание (или его часть) будет сдано
в виде работающей программы на одном из следующих языков программирования: Python, C/C++, Fortran. Код программы должен иметь комментарии
и работать не только на машине студента, но и у преподавателя.

\subsection{Самостоятельные работы}
Самостоятельные работы (СР) будут проводиться в течение 10 -- 15 минут в начале (почти) каждого
практического занятия по пройденному материалу, включая лекции. Они будут состоять из 2--3 вопросов
и/или простых задач. Максимальное число баллов за одну самостоятельную~---~$5,0$, минимальное --- $0,0$.

\subsection{Контрольные работы}
В середине семестра будет проведена контрольная работа (КР) продолжительностью в один академический час (половина пары). Программа
контрольной работы будет включать в себя теоретические и практические вопросы по пройденному материалу, в
том числе лекционному. Максимальное число баллов за контрольную --- $5,0$, минимальное --- $0,0$.\par
Сдача контрольной работы на положительную оценку ($\geq 3,0$) является условием допуска 
студента к зачёту.

\subsection{Зачёт}
При условии всех сданных и защищённых лабораторных работах, итоговая оценка за семестр выставляется
следующим образом. По всем видам контроля выводятся средние оценки, которые затем подставляются в
выражение

\begin{equation*}
    \text{О\_И = 0,5*О\_ДЗ + 0,2*О\_СР + 0,3*О\_КР},
\end{equation*}

где О\_И – итоговая оценка, О\_ДЗ – средняя оценка по домашним заданиям, О\_СР – средняя оценка по
самостоятельным работам, О\_КР – средняя оценка по двум контрольным.

Если итоговая оценка на конец семестра получается не менее 4,0 ($\text{О\_И} >= 4,0$), то студент получает
\textbf{зачёт автоматом}. 

К зачету студент собирает портфолио, то есть всё, что он сделал за семестр:
\begin{itemize}
    \item лабораторные работы;
    \item домашние задания;
    \item дополнительные задания, если имеются;
    \item конспекты практических занятий и лекций;
    \item контрольная работа, независимо от оценки.
\end{itemize}

Зачёт будет проходить в устной форме, и будет включать в себя вопросы по практическим и лекционным занятиям.
Особое внимание уделяется темам, по которым за семестр студент получил неудовлетворительную ($<3,0$)
оценку.

\subsection{Бонусы}

При желании, студенту могут быть даны дополнительные <<творческие>> задания, которые могут быть
выполнены как индивидуально, так и коллективно (2--3 человека). За выполнение таких заданий будут
начисляться бонусные баллы. Штрафных санкций не предусмотрено.

\subsection{Списывание, плагиат и коллективные работы}
Студент должен выполнять все домашние задания самостоятельно. Списанные и «похожие»
работы будут расцениваться как плагиат и зачтены не будут. Это означает, что за все неори-
гинальные работы, независимо от того, кто дал списать, а кто списывал, выставляется оценка
«0» без возможности её пересдачи или исправления. Высшее образование — это добровольный
выбор каждого из вас и некоторыми из условий его получения являются самостоятельная работа,
соблюдение академической этики и норм, уважение к своему и чужому труду.
Часть заданий в течение семестра предполагает совместную работу нескольких студентов. В
этом случае на всех сдаваемых материалах, как обычно, указываются фамилии всех исполнителей.
Части домашних заданий, которые, вообще говоря, не предполагают коллективной работы, могут
в исключительных случаях при возникновении трудностей быть сделаны совместно. Каждый
студент тогда сдаёт работу самостоятельно, указывая при этом, какие конкретно части и с кем
выполнялись вместе. Защита таких работ всё равно проходит индивидуально и на ней может быть
задан вопрос о конкретном вкладе каждого студента в решение.

\subsection{Пропуски занятий}
Отметка о посещении занятия равносильна написанию самостоятельной работы, которая
выдается в самом начале занятия, и/или выполнению лабораторной работы. Опоздание или
пропуск занятия автоматически влечёт за собой оценку «0» за самостоятельную работу с отметкой
о том, что она не сдавалась. Исключения составляют только пропуски по уважительной причине.
В конце семестра 1—2 неудовлетворительные оценки, независимо от причины их получения, могут
быть прощены.

\section{Правила оформления работ}
Все работы (домашние задания) должны иметь титульный лист, на котором указывается номер домашнего задания, номер варианта, а также фамилия и инициалы студента, его факультет, курс и группа.

Текст печатается на одной стороне листа формата А4 белого цвета 12 кеглем через 1,5 интервала с полями слева 3,0 см, справа 1,0 см, сверху и снизу по 2,0 см.

Нумерация страниц сквозная, начиная с титульного листа, однако номер страницы на нем не ставится.

Каждая задача в работе должна начинаться с новой страницы, где приводится полный текст задачи и исходные данные, если последние не занимают больше половины листа A4. Решение
задачи и результаты вычислений приводятся далее в тексте работы. На все сторонние файлы или
web–ресурсы, относящиеся к данной задаче, в тексте должны быть ссылки.

На все рисунки и таблицы в тексте должны быть даны ссылки: «в соответствии с рисунком 1», «в таблице 1». Рисунки и таблицы следует нумеровать арабскими цифрами сквозной
нумерацией. Рисунки подписываются снизу: Рисунок 1 — Притяжение шара и сферы. Наименование таблиц следует помещать над таблицей слева, без абзацного отступа: Таблица 1 — Результаты вычисления значений потенциалов притяжения шара и сферы.

Графики вставляются в работу в виде рисунка и оформляются либо на компьютере, либо на
миллиметровой бумаге. Каждый график должен иметь заголовок. График должен занимать всю
площадь, отведённую под него. На осях графика должны быть обозначены физические величины
с единицами измерения в принятых сокращениях.

\section{Литература}
Литература по курсу доступна для скачивания на \fcolorbox{cyan}{white}{\href{https://yadi.sk/d/T-Eq4MEZ4GlcQg}{Yandex.Disk}}.
\begin{refsection}
    \nocite{Shimbirev1975, Ogorodova2013,Yuzefovich1980, Serkerov1999, Yuzefovich2014, Torge1999}
    \printbibliography[heading=none]
\end{refsection}

\end{document}
