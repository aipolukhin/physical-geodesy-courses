\documentclass[11pt, a4paper]{article}

% For scalable fonts - install cm-super package

% Languages and fonts
\usepackage{cmap} 
\usepackage[T2A]{fontenc}
\usepackage[utf8]{inputenc} 
\usepackage[english, russian]{babel}
\usepackage{microtype}
\usepackage{indentfirst}
\frenchspacing

% Mathematics
\usepackage{amsmath, amssymb, amsfonts, amsthm, mathtools, fixmath}
\mathtoolsset{showonlyrefs=true}
\usepackage{esint, esvect} % integrals and vectors
\usepackage{systeme} % equation system
\usepackage{commath} % partials and differentials
\usepackage{icomma} % smart comma ($0,2$ is a number)

% Floats
\usepackage{float}

% Tables
\usepackage{array,tabularx,tabulary,booktabs} % better tables
\usepackage{longtable}
\usepackage{multirow}

% Graphics
\usepackage[pdftex]{graphicx}
\usepackage{wrapfig}

% Theorems
\renewcommand{\proofname}{Доказательство}

\theoremstyle{plain}
\newtheorem{theorem}{Теорема}[section]

\theoremstyle{definition}
\newtheorem{definition}{Определение}
\newtheorem{corollary}{Следствие}[theorem]
\newtheorem{problem}{Задача}[section]

\theoremstyle{remark}
\newtheorem{remark}{Замечание}
\newtheorem*{solution}{Решение}

\usepackage[top=20mm,bottom=20mm,left=20mm,right=20mm]{geometry}

\usepackage{lastpage} % how many pages

\usepackage{soul}

\usepackage{framed} % easy frames
\usepackage{enumerate} % better numbered lists

\usepackage{hyperref}
\usepackage{xcolor}

\usepackage{tikz} % drawing

\usepackage{csquotes}
\usepackage[style=numeric,backend=biber,sorting=none]{biblatex}
\addbibresource{../../../../bibliography.bib}

\renewcommand{\epsilon}{\ensuremath{\varepsilon}}
\renewcommand{\phi}{\ensuremath{\varphi}}
\renewcommand{\theta}{\vartheta}
\renewcommand{\kappa}{\ensuremath{\varkappa}}
\renewcommand{\le}{\ensuremath{\leqslant}}
\renewcommand{\leq}{\ensuremath{\leqslant}}
\renewcommand{\ge}{\ensuremath{\geqslant}}
\renewcommand{\geq}{\ensuremath{\geqslant}}

\usepackage[useregional]{datetime2}

% custom maketitle
\usepackage{titling}
\setlength{\droptitle}{-4em}
\posttitle{\end{center}\vspace{-3em}}

\title{{\Large Геодезическая гравиметрия 2019}\\ 
    {\bf\Large Практическое занятие № 1} \\
{\Large Введение. Краткие сведения из математики и высшей геодезии}}
\author{}
\DTMsavedate{lessondate}{2019-02-11}
\date{\DTMusedate{lessondate}}

\begin{document}
\maketitle

\section{Предмет и задачи курса}
Вспомним, что основной научной задачей геодезии является определение фигуры и внешнего
гравитационного поля Земли и их изменений во времени. 

Геодезическая гравиметрия решает эту задачу
преимущественно на основе гравиметрических данных (то есть по измерениям величин,
характеризующих гравитационное поле Земли), изучая взаимосвязь фигуры Земли и её гравитационного
поля на поверхности. 
Геодезическую гравиметрию можно рассматривать как теоретический фундамент геодезии.
\footnote{Огородова, Л. В. Геодезическая гравиметрия // 
    Большая российская энциклопедия. Том~6. Москва, 2006, стр.~595}
Синонимами являются дисциплины <<физическая геодезия>> и <<теория фигуры Земли>>.
Решением той же задачи, но с использованием всей
совокупности существующих исходных данных (например, спутниковых) занимается теоретическая
геодезия, которая преподается обычно на последних курсах геодезических специальностей.

Другое определение можно сформулировать так: геодезическая гравиметрия ---
раздел геодезии, в котором рассматриваются теории и методы использования гравиметрических данных 
для решения научных и практических задач геодезии.
\footnote{Юркина, М. И. Геодезическая гравиметрия // Большая советская энциклопедия. --- М.:
Советская энциклопедия. 1969--1978. (в оригинале вместо <<гравиметрических данных>> ---
<<результатов измерения силы тяжести>>)}

Задача получения гравиметрических данных с необходимой плотностью и точностью стоит 
перед другой наукой, которая называется <<гравиметрия>> (или <<экспериментальная гравиметрия>>).

Вообще говоря, задача изучения внешнего гравитационного поля Земли в сущности является задачей
гефизики также, как и изучение магнитного поля (теория которого очень близка к гравитационному) 
и других физических полей. Но, как мы увидим по ходу курса, внешнее гравитационное поле и фигура
Земли на самом деле определяются одновременно из обработки одних и тех же исходных данных. Более
того, эти задачи неотделимы друг от друга, а потому и входят в основную задачу геодезии и её подразделов\cite{Pellinen1978}.

Действительно, ведь абсолютно все (<<геометрические>>) геодезические измерения выполняются в гравитационном поле 
Земли и связаны с ним. В этом легко убедиться, ответив на вопросы:
\begin{enumerate}
    \item Назовите основные геометрические условия в нивелирах и угломерных приборах.
    \item Что происходит с геодезическими приборами, когда мы выставляем их по уровням? 
    \item В какой системе координат выполняются измерения на поверхности Земли?
    \item Чему равна сумма измеренных углов в треугольнике на поверхности Земли, если измерения считать безошибочными?
    \item Как расположена визирная ось поверенного и выставленного по уровням нивелира?
\end{enumerate}
Оказывается, в теории во все наземные и спутниковые измерения, даже выполненные исправными
инструментами и оборудованием, необходимо вводить те или иные поправки, связанные с гравитационным полем Земли.

На практике же необходимость учёта неоднородности гравитационного поля Земли всегда определяется требованиями к
точности результатов измерений. Например, при нивелировании I и II классов вводить поправки в измеренные
превышения за переход к разностям нормальных высот необходимо (в том числе, по нормативным документам), а в 
нивелировании низших классов (III, IV, техническое) --- нет.

Перейдём теперь к более тонкому понятию --- фигуре Земли. Что это такое? 
Понятие фигуры Земли неоднозначно и может подразумевать под собой
\begin{itemize}
    \item геометрическую фигуру простой и правильной формы (сфера, эллипсоид);
    \item фигуру конкретной эквипотенциальной (уровенной) поверхности (Земля --- геоид, Луна --- селеноид, Марс --- ареоид);
    \item фигуру её физической поверхности.
\end{itemize}
Исторически дисциплина развивалась точно также, от простого к сложному (см.
\cite{Ogorodova2013,Yuzefovich2014}), но мы начнём с конца, с современных взглядов. 
Что такое физическая поверхность Земли?

В областях суши физическая поверхность ограничена физической твёрдой оболочкой Земли. Эта же
поверхность изображается на картах, аэро-- и космо-- снимках. На ней проходит большая часть
деятельности человека. 

Мировой океан занимает около 71\% земной поверхности и находится в постоянном движении и
возмущении, кторые вызваны разностями температуры, атмосферного давления, солёности,
ветровыми нагонами и т.д. Поэтому за физическую поверхность здесь принимается невозмущенная поверхность
воды, называемая морской топографической поверхностью. 

Итак, в настоящее время, под фигурой Земли понимают форму её физической поверхности, которая
образуется в областях суши поверхностью твёрдой оболочки Земли, а на территории океанов и
морей -- их невозмущенной поверхностью. 

Физическая поверхность Земли является очень сложной, не всегда однозначно определена и не имеет 
строгого математического описания.
Вместо неё, а также для решения ряда научных и практических задач за приближённую фигуру 
Земли может быть принята одна из уровенных поверхностей потенциала силы тяжести, которая близка (но не совпадает)
к невозмущенной поверхности океана.

При решении целого ряда научных и практических задач можно использовать еще более простую фигуру
Земли, эллипсоид или сферу. Эллипсоид вращения является основой для геодезической системы координат. 
Определение параметров (геометрических и физических) такого эллипсоида, близкого к геоиду, является одной из
современных задач теории фигуры Земли.

Что значит определить поверхность Земли? Что вообще значит определить и задать геометрическую
поверхность? В геодезии в настоящее время под определением физической поверхности Земли подразумевается 
определение положения её точек в единой системе координат. 

\section{Системы координат}

Вообще говоря, сама задача установления системы координат в настоящее время не входит в задачи
геодезической гравиметрии, хотя и тесно с ней связана. В геодезической гравиметрии мы будем
пользоваться различными системами координат прежде всего как теоретическим инструментом (system, а
не frame, если пользоваться англоязычной терминологией), если не оговорено иное.

\subsection{Прямоугольная система координат}
В геодезии используют прямоугольную систему координат, начало $O$ которой находится в центре масс
Земли, ось $Z$ направлена по оси вращения Земли, ось $X$ совмещена с линией пересечения плоскостей
экватора и начального меридиана, ось $Y$ дополняет систему до правой\cite{Ogorodova2006}. 
Это геоцентрическая или общеземная система координат. В ней положение точек определяется по всей
Земле. Если начало системы координат по той или иной причине смещено относительно центра масс, то
система называется референцной.

\subsection{Сферическая система координат}

Сферические (полярные) координаты определяются геоцентрической широтой $\overline{\phi}$ (или полярным
расстоянием $\theta$), долготой $\lambda$ и полярным радиус--вектором $r$.
Геоцентрической широтой $\overline{\phi}$ называется угол между радиусом--вектором заданной точки и плоскостью
экватора.
Долгота $\lambda$ есть угол между
плоскостью меридиана заданной точки и плоскостью меридиана, принятого в качестве начального.
Полярное расстояние $\theta$ является дополнением широты $\overline{\phi}$ до $90^\circ$:
\begin{equation*}
    \theta = 90^\circ - \overline{\phi}.
\end{equation*}
Сферические координаты связаны с прямоугольными следующими соотношениями
\begin{align*}
    &X = r\cos\overline{\phi}\cos\lambda,\\
    &Y = r\cos\overline{\phi}\sin\lambda,\\
    &Z = r\sin\overline{\phi}.
\end{align*}
\begin{problem}
Получите формулы связи для случая, когда вместо широты $\Phi$ задано полярное расстояние $\theta$.
\end{problem}
\begin{solution}
\begin{align*}
    &X = r\sin\theta\cos\lambda,\\
    &Y = r\sin\theta\sin\lambda,\\
    &Z = r\cos\theta.
\end{align*}
\end{solution}
\begin{problem}
    Получите обратные формулы для перехода от геоцентрических прямоугольных координат к сферическим.
\end{problem}
\begin{solution}
\begin{align*}
    &r = \sqrt{X^2 + Y^2 + Z^2},\\
    &\overline{\phi} = \arctg\dfrac{Z}{\sqrt{X^2 + Y^2}},\\
    &\lambda = \arctg\dfrac{Y}{X}.
\end{align*}
\end{solution}

\subsection{Астрономические  координаты}
Астрономичесие координат естественным образом возникают при измерениях в гравитационном поле и
определяют направление силовой линии поля силы тяжести.
Астрономическая широта $\Phi$ -- это дополнение до $90^\circ$ угла между линией, параллельной оси
вращения Земли, и отвесной линией. Долгота равна двугранному углу между плоскостями начального
астрономического меридиана и астрономического меридиана данной точки.
\begin{problem}
    Объясните, чем неудобна астрономическая система координат?
\end{problem}

\subsection{Эллипсоид. Геодезическая система координат}
Во многих геодезических приложениях применяют системы геодезических координат $B, L, H$, 
связанных с выбранным
эллипсоидом вращения. Эллипсоид обычно задается его большой полуосью $a$ и сжатием $\alpha$.
Вспомним, что 
\begin{equation*}
    \alpha = \frac{a - b}{a},\quad e^2 = \frac{a^2 - b^2}{a^2},
\end{equation*}
где $b$ --- малая полуось эллипсоида, $e$ -- его первый эксцентриситет. 

Геоодезическая широта $B$ для некоторой точки $P$ есть угол между опущенной из $P$ нормалью к эллипсоиду и плоскостью
экватора. Геодезическая долгота $L$ --- угол между плоскостью начального меридиана и плоскостью
меридиана точки $P$ (равна сферической долготе $\lambda$). Геодезическая высота~$H$ --- кратчайшее
расстояние от точки $P$ по нормали до поверхности эллипсоида.

Важно отметить, что именно геодезическая система координат подразумевается, когда мы говорим
об определении физической поверхности Земли в единой системе координат. 
\begin{problem}
Подумайте, всегда ли в этой
системе координат поверхность Земли может быть определена однозначно? Какие недостатки у этой
системы координат?
\end{problem}

Геодезические координаты связаны с прямоугольными следующими соотношениями
\begin{align*}
    &X = \left( N + H \right)\cos{B}\cos{L},\\
    &Y = \left( N + H \right)\cos{B}\sin{L},\\
    &Z = \left( N (1 - e^2) + H \right)\sin{B},
\end{align*}
где $N$ --- радиус кривизны первого вертикала, который, как известно из курса сфероидической
геодезии, вычисляется так
\begin{equation*}
    N = \frac{a}{\sqrt{1 - e^2\sin^2{B}}}.
\end{equation*}

\begin{problem}
    Вспомните, что такое первый вертикал и что такое плоскость меридиана? Что такое главные радиусы
    кривизны эллипсоида?
\end{problem}

Геодезическая долгота $L$ совпадает со сферической
долготой $\lambda$, если начала и ориентация координатных осей систем совпадают. 
Геоцентрическая широта отличается от геодезической. Опуская вывод (см. лекции), приведем здесь окончательное выражение для точки на поверхности
эллипсоида
\begin{equation*}
    \tg{\overline{\phi}} = \left( 1 - e^2 \right)\tg{B}.
\end{equation*}

В некоторых геодезических выводах также полезно использовать приведенную широту $u$ --- геоцентрическую широту точки $P'$, которая является проекцией точки $P_0$ (пересечение нормали точки $P$ с эллипсоидом) на вспомогательную сферу радиуса $a$ (размер большой полуоси) нормальной к плоскости экватора. Приведенная широта связана с геодезической следующим выражением (вывод см. в лекциях):
\begin{equation*}
    \tg{u} = \sqrt{1 - e^2}\tg{B}.
\end{equation*}

Геодезические широта и долгота отличаются от соответствующих астрономических координат, поскольку
направление отвесной линии отличается от направления нормали к эллипсоиду. Угол между направлением
отвесной линии и нормалью к эллипсоиду называется астрономо--геодезическим уклонением отвеса.
Удобно этот угол разложить на две составляющие --- проекции угла в плоскости первого вертикала
$\eta$  и в плоскости меридиана $\xi$, тогда
\begin{align*}
    &\xi = \Phi - B,\\
    &\eta = \left( \Lambda - L \right)\cos\phi.
\end{align*}
В дальнейшем мы познакомимся и с другими видами уклонения отвеса.

\section{Связь с другими науками}
\paragraph{Математика.}
Изучение гравитационного поля и фигуры Земли --- сложная задача. В ходе курса мы будет пользоваться
различными разделами математики, с некоторыми из которых вам придется познакомиться впервые:
\begin{itemize}
    \item векторный анализ,
    \item теория поля,
    \item теория ньютоновского потенциала,
    \item специальные функции,
    \item дифференциальные уравнения, обыкновенные и в частных производных,
    \item краевые задачи.
\end{itemize}
Исторически так сложилось, как и в случае теории математической обработки геодезических
измерений, обогатившей теорию вероятностей, теория фигуры Земли обогатила многие разделы математики,
которые теперь прочно служат её основой.

\paragraph{Геофизика и геология.} Гравитационное поле на поверхности Земли отражает распределение масс внутри
неё. И хотя, как мы очень скоро убедимся, одних только гравиметрических данных недостаточно для изучения 
внутреннего строения, они, наряду с другими геоифизическими методами, служат важным источником
информации.\par
Гравиметрический метод является одним из основных при поиске и разведке полезных ископаемых.
Высокоточные регулярные измерения используются для монторинга месторождений в процессе добычи нефти и газа.
\paragraph{Археология и строительство.} Локальная информация о гравитационном поле может быть
полезна для поиска пустот (карст), провалов, древних подземных ходов и тоннелей, объектов археологического
наследия.
\paragraph{Гляциология и уровень моря. Океанология.} Таяние ледников, вызванное изменением климата, уменьшает их
массу, следовательно, меняется и гравитационное поле. По спутниковым гравиметрическим данным (миссия GRACE) получены
важнейшие данные о ледниках Гренландии и Антарктиды. Таяние льдов вызывает рост среднего
уровня Мирового океана, следовательно, изменение высоты морской топографической поверхности, то есть
физической поверхности Земли. Эти процессы изучаются методом спутниковой альтиметрии.
\paragraph{Гидрология.} Перераспределение водных масс на всей поверхности Земли вызвано не только
таянием льдов, но и другими климатическими явлениями. Локальные измерения слы тяжести позволяют
изучать местный гидрологический режим, а спутниковые гравиметрические миссии --- региональный и даже
глобальный. 
\paragraph{Орбиты ИСЗ.} Для вычисления орбит искусственных спутников для определения его положения относительно центра
масс Земли необходимо знание гравитационного поля вне поверхности Земли (на высоте полета спутника). Этот нюанс свидетельствует
о том, что, казалось бы, чисто геометрический метод определения координат при помощи глобальных
навигационных спутниковых систем, на самом деле также связан с гравитационным полем.

Кроме всего вышеперечисленного, высокоточные измерения силы тяжести используются в метрологии и при
изучении геодинамических процессов, а также в других областях науки и техники.

\section{Задачи для решения на занятии}
Найдите в Google таблицу простейших производных и интегралов (первообразных). Вспомните основные правила дифференцирования и интегрирования. Решите примеры:

\begin{enumerate} 
	\item Найти производную функций ($a$ и $n$ -- числа)
		\begin{enumerate} 
			\item $y = x + \sqrt{x} + \sqrt[3]{x}$,
			\item $y = \dfrac{1}{x} + \dfrac{1}{\sqrt{x}} + \dfrac{1}{\sqrt[3]{x}}$,
			\item $y = \sin^n{x}\cdot\cos{nx}$,
			\item $y = \dfrac{a}{x^n}$, найти $y'''$.
		\end{enumerate} 
	\item Найти все частные производные первого и второго порядков для функции
	$f\left( x, y \right) = \dfrac{x}{y}$.
	\item Разложить в ряд Тейлора функцию из предыдущего примера в окрестности точки $M(1, 1)$.
	\textit{Подсказка: примените формулу Тейлора для функции двух переменных.}
\end{enumerate} 
	
\printbibliography

\end{document}
