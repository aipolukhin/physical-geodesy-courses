\documentclass[11pt, a4paper]{article}

% Languages and fonts
\usepackage{cmap} 
\usepackage[T2A]{fontenc}
\usepackage[utf8]{inputenc} 
\usepackage[english, russian]{babel}
\usepackage{microtype}
\usepackage{indentfirst}
\frenchspacing

% Mathematics
\usepackage{amsmath, amssymb, amsfonts, amsthm, mathtools, fixmath}
\usepackage{esint, esvect} % integrals and vectors
\usepackage{systeme} % equation system
\usepackage{commath} % partials and differentials
\usepackage{icomma} % smart comma ($0,2$ is a number)

% Floats
\usepackage{float}

% Tables
\usepackage{array,tabularx,tabulary,booktabs} % better tables
\usepackage{longtable}
\usepackage{multirow}

% Graphics
\usepackage[pdftex]{graphicx}
\usepackage{wrapfig}

% Theorems
\renewcommand{\proofname}{Доказательство}

\theoremstyle{plain}
\newtheorem{theorem}{Теорема}[section]

\theoremstyle{definition}
\newtheorem{definition}{Определение}
\newtheorem{corollary}{Следствие}[theorem]
\newtheorem{problem}{Задача}[section]

\theoremstyle{remark}
\newtheorem{remark}{Замечание}
\newtheorem*{solution}{Решение}

\usepackage[top=20mm,bottom=20mm,left=20mm,right=20mm]{geometry}

\usepackage{lastpage} % how many pages

\usepackage{soul}

\usepackage{framed} % easy frames
\usepackage{enumerate} % better numbered lists

\usepackage{hyperref}
\usepackage{xcolor}

\usepackage{tikz} % drawing

\usepackage{csquotes}
\usepackage[style=numeric,backend=biber,sorting=none]{biblatex}
\addbibresource{../../../../bibliography.bib}

\renewcommand{\epsilon}{\ensuremath{\varepsilon}}
\renewcommand{\phi}{\ensuremath{\varphi}}
%\renewcommand{\theta}{\vartheta}
\renewcommand{\kappa}{\ensuremath{\varkappa}}
\renewcommand{\le}{\ensuremath{\leqslant}}
\renewcommand{\leq}{\ensuremath{\leqslant}}
\renewcommand{\ge}{\ensuremath{\geqslant}}
\renewcommand{\geq}{\ensuremath{\geqslant}}

\usepackage[useregional]{datetime2}

% custom maketitle
\usepackage{titling}
\setlength{\droptitle}{-4em}
\posttitle{\end{center}\vspace{-3em}}

\title{{\Large Геодезическая гравиметрия 2018}\\ 
    {\bf\Large Практическое занятие № 5} \\
{\Large Притяжение тел сложной формы. Гармонические функции}}
\author{}
\DTMsavedate{lessondate}{2018-03-13}
\date{\DTMusedate{lessondate}}

\begin{document}
\maketitle

Потенциал объёмных масс
\begin{equation}
    V = G\iiint\limits_\tau \dfrac{\delta}{r}\dif \tau,
\end{equation}
а внутри притягивающих масс потенциал удовлетворяет уравнению Пуассона
\begin{equation}
    \label{eq:poisson}
    \Delta V = -4\pi G\delta.
\end{equation}
Из последних двух выражений нетрудно получить
\begin{equation}
    \label{eq:poissonvol}
    V = -\dfrac{1}{4\pi}\iiint\limits_\tau \dfrac{\Delta V}{r} \dif\tau,
\end{equation}
то есть зная вторые производные потенциала притяжения во всех внутренних точках объёма $\tau$ можно
определить потенциал в любой точке пространства.

\section{Первая и вторая формулы Грина}
Пусть в области $\tau$ с границей $\sigma$ заданы две правильные функции $U\left( x, y, z \right)$ и
$V \left( x, y, z \right)$. Функция называется правильной в области $\tau$, если она внутри этой
области и на её границе, конечна, однозначна и непрерывна вместе со всеми своими частными
производными первых двух порядков. \par

Если правильная функция $V \left( x, y, z \right)$ удовлетворяет во всех точках области $\sigma$
уравнению Лапласа $\Delta V = 0$, то она называется гармонической в области $\tau$. 

Опуская вывод, запишем для этих функций первую формулу Грина
\begin{equation*}
    \iiint\limits_\tau U\Delta V\dif\tau + \iiint\limits_\tau D\left( U, V \right)\dif\tau =
    \iint\limits_\sigma U\od{V}{n}\dif\sigma,
\end{equation*}
где
\begin{equation*}
    D\left( U, V \right) = \dpd{U}{x}\dpd{V}{x} + \dpd{U}{y}\dpd{V}{y} + \dpd{U}{z}\dpd{V}{z}
\end{equation*}
--- оператор Дирихле, $\dod{V}{n}$ --- производная по внешней нормали к поверхности $\sigma$.
Для краткости далее не будем писать знаки тройных и двойных интегралов, то есть
\begin{equation}
    \label{eq:firstGreen}
    \int\limits_\tau U\Delta V\dif\tau + \int\limits_\tau D\left( U, V \right)\dif\tau =
    \int\limits_\sigma U\od{V}{n}\dif\sigma.
\end{equation}
Поменяем в (\ref{eq:firstGreen}) местами функции $U$ и $V$, получим
\begin{equation*}
    \int\limits_\tau V\Delta U\dif\tau + \int\limits_\tau D\left( V, U \right)\dif\tau =
    \int\limits_\sigma V\od{U}{n}\dif\sigma.
\end{equation*}
Вычитая это выражение почленно из (\ref{eq:firstGreen}), получаем вторую формулу Грина
\begin{equation}
    \label{eq:secondGreen}
    \int\limits_\tau \left[ U\Delta V - V\Delta U \right] \dif\tau  =
    \int\limits_\sigma U\od{V}{n} - V\od{U}{n} \dif\sigma.
\end{equation}
Пусть $U = 1$, тогда из первой формулы Грина (\ref{eq:firstGreen})
\begin{equation}
    \label{eq:firstGreenU1}
    \int\limits_\tau \Delta V\dif\tau = \int\limits_\sigma\od{V}{n}\dif\sigma.
\end{equation}
Если $V$ --- гармоническая функция ($\Delta V = 0$), то
\begin{equation}
    \label{eq:harmchar}
    \int\limits_\sigma\od{V}{n}\dif\sigma = 0,
\end{equation}
то есть двойной интеграл от нормальной производной гармонической функции по замкнутой поверхности
равен нулю (или поток вектора градиента гармонической функции через замкнутую поверхность равен
нулю). Это \textbf{характеристическое свойство гармонических функций}. 
Причем, двойной интеграл может быть взять по любой (не только по $\sigma$ в смысле границы
области $\tau$) замкнутой поверхности, принадлежащей
области правильности функции. 
Из характеристического свойства можно доказать одну фундаментальную для гармонических функций теорему. 
\begin{theorem}[О максимуме и минимуме]
    Гармоническая функция $V$, заданная в области $\tau$, не может иметь внутри этой области ни
    макcимумов, ни минимумов.
    \label{theorem:maxmin}
\end{theorem}
\begin{proof}
    Будем рассуждать от противного. Предположим, что гармоническая функция $V$ достигает в некоторой
    точке $P$, лежащей внутри $\tau$ своего максимума (минимума). В силу этого, значения функции
    $V$ вблизи точки $P$ будут меньше (больше), чем значения функции в самой точке $P$. Поэтому мы
    можем найти сферу $\sigma$ столь малого радиуса, что во всех точка поверхности этой сферы
    производная по внешней нормали $\od{V}{n}$ будет отрицательной (положительной). Поэтому левая
    часть (\ref{eq:harmchar}) будет существенно положительной (отрицательной), что противоречит самой формуле
    (\ref{eq:harmchar}).
\end{proof}

\section{Стоксовы постоянные}
Вернёмся к формуле (\ref{eq:firstGreenU1}). Если $V$ --- потенциал объёмных масс, а точка находится внутри притягиваемого тела 
$\tau$, то $\Delta V = -4\pi G\delta$, поэтому
\begin{equation*}
    -4\pi G\underbrace{\int\limits_\tau\delta\dif\tau}_{\text{масса}} =
    \int\limits_\sigma\od{V}{n}\dif\sigma,
\end{equation*}
откуда получаем формулу Гаусса
\begin{equation}
    \label{eq:gm}
    GM = -\dfrac{1}{4\pi}\int\limits_\sigma\od{V}{n}\dif\sigma,
\end{equation}
где $M$ --- полная масса тела. Если поверхность $\sigma$ ограничивает некоторую внутреннюю
область для $\tau$, скажем, $\tau'$, то $M$ --- будет массой, заключённой в объёме $\tau'$. Таким
образом, если на поверхности $\sigma$ тела $\tau$ известны значения производной потенциала по нормали к
этой поверхности, то можно вычислить массу $M$ (или $GM$ --- гравитационный параметр,
планетоцентрическая гравитационная постоянная), заключенную в $\tau$. Если $\sigma$ является
уровенной для потенциала $V$, то производная по нормали есть сила $F = -\od{V}{n}$.

Обратимся теперь ко второй формуле Грина (\ref{eq:secondGreen}). Пусть $V$ --- потенциал
объёмных масс $\tau$, $U$ --- произвольная гармоническая функция. Применим вторую формулу Грина
\begin{equation*}
    -4\pi G\delta \int\limits_{\tau} U \delta \dif\tau = \int\limits_\sigma 
    \left( U \od{V}{n} - V \od{U}{n} \right). 
\end{equation*}
Объёмные интегралы 
\begin{equation*}
\int\limits_{\tau} U \delta \dif\tau
\end{equation*}
от произведения плотности на произвольную гармоническую функцию называются
\textbf{стоксовыми постоянными}. Они могут быть однозначно определены, если на поверхности
$\sigma$ тела заданы производные потенциала $\od{V}{n}$ притяжения по нормали и значения самого потенциала $V$.

Гармоническую функцию $U$ можно выбирать в виде многочлена степени $n$ от координат $x$, $y$,
$z$. В общем виде можно записать однородный многочлен степени $n$
\begin{equation*}
    U_n \left( x, y, z \right) = \sum\limits_{n=1}^{N} a_{pqr} x^p y^q z^r,\quad p + q + r = n,
    \quad \Delta U = 0.
\end{equation*}
Разные степени $n$ соответствуют стоксовым постоянным разных порядков. Поскольку $U$ ---
гармоническая функция $\Delta U = 0$, то коэффициенты не могут быть выбраны произвольно.

При $n = 0$ $U\left( x, y, z \right) = const$. Пусть  $U \left( x, y, z \right) = 1$, тогда
получим стоксову постоянную нулевого порядка (\ref{eq:gm}), которая равна массе тела.

Пусть $n = 1$,
тогда $U \left( x, y, z \right) = x + y + z$, поэтому стоксовы постоянные первого
порядка будут равны
\begin{equation*}
    \int\limits_{\tau} x \delta \dif\tau,\quad
    \int\limits_{\tau} y \delta \dif\tau,\quad 
    \int\limits_{\tau} z \delta \dif\tau.
\end{equation*}
Из механики известно, что положение центра масс тела --- геометрической точки,
характеризующей движение тела или системы частиц как целого --- определяется так
\begin{equation*}
    \vv{r_c} = \dfrac{1}{M} \int\limits_\tau \delta \vv{r}\dif\tau,
\end{equation*}
следовательно, стоксовы постоянные первого порядка определяют положение центра масс тела
\begin{equation*}
    x_0 = \dfrac{1}{M}\int\limits_{\tau} x \delta \dif\tau,\quad
    y_0 = \dfrac{1}{M}\int\limits_{\tau} y \delta \dif\tau,\quad 
    z_0 = \dfrac{1}{M}\int\limits_{\tau} z \delta \dif\tau.
\end{equation*}
Вообще говоря, интегралы $\int_M r^n \dif m$ от произведения элемента массы $\dif m = \delta\dif\tau$ массы тела на
$n$--ую степень от расстояния до оси или плоскости называются моментами тела относительно оси или
плоскости соответственно. Моменты первого порядка ($n = 1$) называются статическими. Таким образом, стоксовы
постоянные первого порядка это статические моменты первого порядка относительно координатных осей.\\
Моменты второго порядка ($n = 2$) называются моментами инерции
\begin{equation*}
    J = \int\limits_M r^2 \dif m,
\end{equation*}
которые являются мерой инертности тела во вращательном движении вокруг оси.\\
Если положить
\begin{equation*}
    U = xy,\quad U = xz, \quad U = yz,
\end{equation*}
то получим стоксовы постоянные
\begin{equation*}
    J_{xy} = \int\limits_{\tau} xy \delta \dif\tau,\quad
    J_{xz} = \int\limits_{\tau} xz \delta \dif\tau,\quad
    J_{yz} = \int\limits_{\tau} yz \delta \dif\tau,
\end{equation*}
которые называются центробежными моментами инерции или произведениями инерции относительно осей
прямоугольной системы координат.\\
Рассмотрим еще две стоксовы постоянные второго порядка для
\begin{equation*}
    U = z^2 - \dfrac{x^2 + y^2}{2},\quad U =  x^2 - y^2,
\end{equation*}
которые имеют вид
\begin{align*}
    \int\limits_{\tau} \left[ z^2 - \dfrac{x^2 + y^2}{2} \right] \delta \dif\tau =
    \int\limits_{\tau} \dfrac{\delta}{2} \left[ 2z^2 - x^2 + y^2 \right] \dif\tau =
    \dfrac{A + B}{2} - C,\\
    \int\limits_{\tau} \left[ x^2 - y^2 \right] \delta \dif\tau = B - A,
\end{align*}
где 
\begin{align*}
    &A = J_{xx} = \int\limits_{\tau} \left( y^2 + z^2 \right) \delta \dif\tau,\quad\\
    &B = J_{yy} = \int\limits_{\tau} \left( x^2 + z^2 \right) \delta \dif\tau,\quad\\
    &C = J_{zz} = \int\limits_{\tau} \left( x^2 + y^2 \right) \delta \dif\tau
\end{align*}
--- моменты инерции тела относительно осей $X$, $Y$, $Z$ соответственно.

Таким образом, массу тела, координаты центра масс, разности моментов инерции, произведения инерции
можно рассматривать как стоксовы постоянные различных порядков.

Определение стоксовых постоянных является одной из основных задач современной геодезии. Для Земли она
решается методами космической геодезии. Определением стоксовых постоянных аномальных тел составляет
основную задачу гравитационной разведки.

\section{Фундаментальная формула Грина}
Во второй формуле Грина (\ref{eq:secondGreen}) будем полагать $U = \frac{1}{r}$, тогда
\begin{equation}
    \int\limits_\tau \dfrac{\Delta V}{r} \dif\tau  = \int\limits_\sigma \left[
    \dfrac{1}{r}\dod{V}{n} - V\dod{}{n}\left( \frac{1}{r} \right)\right] \dif\sigma.
\end{equation}
Если $V$ --- потенциал объемных масс, то учитывая выражение ($\ref{eq:poissonvol}$), получаем
\begin{equation}
    \label{eq:fundamentalGreen}
    V  = -\dfrac{1}{4\pi}\int\limits_\sigma \left[
    \dfrac{1}{r}\dod{V}{n} - V\dod{}{n}\left( \frac{1}{r} \right)\right] \dif\sigma,
\end{equation}
которая называется \textbf{основной формулой теории гармонических функций} или \textbf{фундаментальной
формулой Грина} для внешней точки.
Первое слагаемое правой части 
\begin{equation*}
    -\dfrac{1}{4\pi}\int\limits_\sigma  \dfrac{1}{r}\dod{V}{n}  \dif\sigma
\end{equation*}
есть не что иное, как потенциал притяжения простого слоя поверхностной плотности $\mu =
-\dfrac{1}{4\pi}\dod{V}{n}$. Второе слагаемое
\begin{equation*}
    \dfrac{1}{4\pi}\int\limits_\sigma V\dod{}{n}\left( \frac{1}{r} \right)\dif\sigma
\end{equation*}
представляет собой потенциал двойного слоя плотности $\dfrac{V}{4\pi}$. 

Таким образом, потенциал объёмных масс во внешнем пространстве можно представить суммой
потенциалов двойного и прстого слоя, распределённых на поверхности $\sigma$ притягиваемого тела
$\tau$. Отметим и то, что для определения внешнего потенциала не нужна информация о распределении
плотности внутри тела.

%\printbibliography
\end{document}
