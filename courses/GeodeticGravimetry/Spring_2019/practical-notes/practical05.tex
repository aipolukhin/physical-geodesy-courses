\documentclass[11pt, a4paper]{article}

% Languages and fonts
\usepackage{cmap} 
\usepackage[T2A]{fontenc}
\usepackage[utf8]{inputenc} 
\usepackage[english, russian]{babel}
\usepackage{microtype}
\usepackage{indentfirst}
\frenchspacing

% Mathematics
\usepackage{amsmath, amssymb, amsfonts, amsthm, mathtools, fixmath}
\mathtoolsset{showonlyrefs=true}
\usepackage{esint, esvect} % integrals and vectors
\usepackage{systeme} % equation system
\usepackage{commath} % partials and differentials
\usepackage{icomma} % smart comma ($0,2$ is a number)

% Floats
\usepackage{float}

% Tables
\usepackage{array,tabularx,tabulary,booktabs} % better tables
\usepackage{longtable}
\usepackage{multirow}

% Graphics
\usepackage[pdftex]{graphicx}
\usepackage{wrapfig}

% Theorems
\renewcommand{\proofname}{Доказательство}

\theoremstyle{plain}
\newtheorem{theorem}{Теорема}[section]

\theoremstyle{definition}
\newtheorem{definition}{Определение}
\newtheorem{corollary}{Следствие}[theorem]
\newtheorem{problem}{Задача}[section]

\theoremstyle{remark}
\newtheorem{remark}{Замечание}
\newtheorem*{solution}{Решение}

\usepackage[top=20mm,bottom=20mm,left=20mm,right=20mm]{geometry}

\usepackage{lastpage} % how many pages

\usepackage{soul}

\usepackage{framed} % easy frames
\usepackage{enumerate} % better numbered lists

\usepackage{hyperref}
\usepackage{xcolor}

\usepackage{tikz} % drawing

\usepackage{csquotes}
\usepackage[style=numeric,backend=biber,sorting=none]{biblatex}
\addbibresource{../../../../bibliography.bib}

\renewcommand{\epsilon}{\ensuremath{\varepsilon}}
\renewcommand{\phi}{\ensuremath{\varphi}}
%\renewcommand{\theta}{\vartheta}
\renewcommand{\kappa}{\ensuremath{\varkappa}}
\renewcommand{\le}{\ensuremath{\leqslant}}
\renewcommand{\leq}{\ensuremath{\leqslant}}
\renewcommand{\ge}{\ensuremath{\geqslant}}
\renewcommand{\geq}{\ensuremath{\geqslant}}

\usepackage[useregional]{datetime2}

% custom maketitle
\usepackage{titling}
\setlength{\droptitle}{-4em}
\posttitle{\end{center}\vspace{-3em}}

\title{{\Large Геодезическая гравиметрия 2019}\\ 
    {\bf\Large Практическое занятие № 5} \\
{\Large Элементы теории поля. Свойства потенциала притяжения}}
\author{}
\DTMsavedate{lessondate}{2019-03-11}
\date{\DTMusedate{lessondate}}

\begin{document}
\maketitle

\section{Уравнения Лапласа и Пуассона}
Потенциал объёмных масс является функцией непрерывной, однозначной и конечной во всём проcтранстве.
Этими же свойствами обладают и первые производные потенциала.

Найдём вторые производные потенциала объёмных масс в прямоугольных
координатах
\begin{align*}
    &\dpd[2]{V}{x} = G\iiint\limits_{\tau} \delta \left[ \dfrac{3 \left( x - x_1 \right)^2}{r^5} -
    \dfrac{1}{r^3} \right]\dif\tau, \\
    &\dpd[2]{V}{y} = G\iiint\limits_{\tau} \delta \left[ \dfrac{3 \left( y - y_1 \right)^2}{r^5} -
    \dfrac{1}{r^3} \right]\dif\tau, \\
    &\dpd[2]{V}{z} = G\iiint\limits_{\tau} \delta \left[ \dfrac{3 \left( z - z_1 \right)^2}{r^5} -
    \dfrac{1}{r^3} \right]\dif\tau.
\end{align*}
Складываем все три равенства и получаем
\begin{equation*}
    \dpd[2]{V}{x} + \dpd[2]{V}{y} + \dpd[2]{V}{y} = 
    G\iiint\limits_{\tau} \delta \left[ \dfrac{3 \left( x - x_1 \right)^2 +
        3 \left( y - y_1 \right)^2 +3 \left( z - z_1 \right)^2}{r^5} - \dfrac{3}{r^3} 
    \right]\dif\tau =
    G\iiint\limits_{\tau} \delta \left[ \dfrac{3}{r^3} - \dfrac{3}{r^3} \right]\dif\tau = 0.
\end{equation*}
То есть
\begin{equation*}
    \Delta V = \dpd[2]{V}{x} + \dpd[2]{V}{y} + \dpd[2]{V}{y} = 0.
\end{equation*}
Это \textbf{уравнение Лапласа} --- дифференциальное уравнение в частных производных второго порядка. Здесь
$\Delta = \dpd[2]{}{x} + \dpd[2]{}{y} + \dpd[2]{}{y}$ --- оператор Лапласа или <<лапласиан>>.
Функция, непрерывная в некоторой области вместе со своими частными производными первого и второго
порядков и удовлетворяющая уравнению Лапласа, называется \textbf{гармонической}. Таким образом,
потенциал притяжения во внешнем пространстве является гармонической функцией.

\begin{problem}
Доказать, что функция $f (x, y) = \ln (x^2 + y^2)$ является гармонической везде, кроме начала координат $(x = 0, y = 0)$.
\end{problem}

Рассмотрим случай, когда притягиваемая точка $P$ находится внутри притягивающего тела объёма $\tau$, ограниченного поверхностью $\Sigma$ произвольной формы. Окружим точку $P$ малой замкнутой поверхностью $S$ (в общем случае --- произвольной формы). Рассмотрим частный случай, когда $S$ имеет форму сферы радиуса $R$. Тогда потенциал в точке $P$ можно разделить на две части $V = V_1 + V_2$:
\begin{enumerate}
	\item на потенциал $V_1$ шара радиуса $R$, внутри которого находится точка $P$;
	\item на потенциал $V_2$ всех остальных масс вне шара.
\end{enumerate}
Применим оператор Лапласа на левую и правую части суммы: $\Delta V = \Delta (V_1 + V_2) = \Delta V_1 + \Delta V_2$.
Потенциал $V_1$ будет определяться выражением:
\begin{equation*}
	V_1 = \dfrac{2}{3} \pi G \delta \left(3R^3 - r^2 \right), 
	r = \sqrt{x^2 + y^2 + z^2}.
\end{equation*}
Найдём вторые производные по $x, y, z$. Для производной по $x$ получаем:
\begin{equation*}
	\dpd[2]{V_1}{x} = \dpd{}{x} \left(- \dfrac{4}{3} \pi G \delta x \right) = - \dfrac{4}{3} \pi G \delta.
\end{equation*}
Аналогично для $y$ и $z$:
\begin{equation*}
	\dpd[2]{V_1}{y} = - \dfrac{4}{3} \pi G \delta,    
	\dpd[2]{V_1}{z} = - \dfrac{4}{3} \pi G \delta.
\end{equation*}
Для потенциала $V_2$ точка $P$ является лежащей вне притягивающих масс, а значит удовлетворяет уравнению Лапаласа $\Delta V_2 = 0$. Окончательно получаем:
\begin{equation*}
	\Delta V = \dpd[2]{V_1}{x} + \dpd[2]{V_1}{y} + \dpd[2]{V_1}{z} + \dpd[2]{V_2}{x} + \dpd[2]{V_2}{y} + \dpd[2]{V_2}{z} = -4 \pi G \delta + 0 = -4 \pi G \delta,
\end{equation*}
или
\begin{equation*}
	\Delta V = \dpd[2]{V_1}{x} + \dpd[2]{V_1}{y} + \dpd[2]{V_1}{z} = -4 \pi G \delta.
\end{equation*}
Это \textbf{уравнение Пуассона}, которому удовлетворяет потенциал притяжения внутри притягивающих масс. Таким образом, потенциал притяжения внутри притягивающих масс \textbf{не является гармонической функцией}. Легко заметить, что вне притягивающих масс $\delta = 0$ уравнение Пуассона переходит в уравнение Лапласа.

\section{Регулярность на бесконечности}
Исследуем поведение потенциала на бесконечности. Напишем неравенство
\begin{equation*}
    G\int\limits_\tau\dfrac{\delta\dif\tau}{r_{min}} >
    G\int\limits_\tau\dfrac{\delta\dif\tau}{r} >
    G\int\limits_\tau\dfrac{\delta\dif\tau}{r_{max}},
\end{equation*}
где $r_{min}$, $r_{max}$ --- минимальное и максимальное расстояние от притягиваемой точки до
притягиваемого тела. Поскольку $\int\limits_\tau\delta\dif\tau = M$, то
\begin{equation*}
    G\dfrac{M}{r_{min}} >
    G\dfrac{M}{r} >
    G\dfrac{M}{r_{max}}.
\end{equation*}
Образуем производную $\pd{V}{r} = -G\int\limits_\tau\dfrac{\delta\dif\tau}{r^2}$ и неравенство
\begin{equation*}
    G\dfrac{M}{r^2_{min}} >
     \left|\dpd{V}{r}\right| >
    G\dfrac{M}{r^2_{max}}.
\end{equation*}
Теперь умножаем предпоследнее неравенство на $r$, а последнее --- на $r^2$, тогда
\begin{equation*}
    G\dfrac{Mr}{r_{min}} >
    Vr >
    G\dfrac{Mr}{r_{max}}, \qquad
    G\dfrac{Mr^2}{r^2_{min}} >
     \left|\dpd{V}{r}\right| r^2 >
    G\dfrac{M r^2}{r^2_{max}}.
\end{equation*}
Пусть $r\to\infty$, тогда
\begin{equation*}
    \lim\limits_{r\to\infty}\dfrac{r}{r_{min}} = 1,\qquad 
    \lim\limits_{r\to\infty}\dfrac{r}{r_{max}} = 1,
\end{equation*}
и
\begin{equation*}
    \lim\limits_{r\to\infty} V = 0,\qquad 
    \lim\limits_{r\to\infty} rV = GM,\qquad
    \lim\limits_{r\to\infty} \left|\dpd{V}{r}\right| r^2 = GM.
\end{equation*}
Функция, удовлетворяющая всем трём последним пределам называется \textbf{регулярной на
бесконечности}. Следовательно, потенциал является функцией, регулярной на бесконечности. Иными словами, в бесконечно удалённой точке потенциал обращается в нуль.

\section{Скалярное поле, градиент и производная по направлению}
Если каждой точке пространства или его части однозначно сопоставлена некоторая скалярная (векторная) величина, то говорят, что задано скалярное (векторное) поле этой величины.

Геометрической характеристикой скалярного поля $\phi = \phi \left(x, y, z\right)$ является поверхность уровня или уровенная поверхность
\begin{equation*}
    \phi \left(x, y, z\right) = const.
\end{equation*}
Если $\phi$ --- потенциальное поле, то поверхность уровня называется эквипотенциальной.

Градиентом поля называется вектор
\begin{equation*}
    \nabla \phi = \vv{i} \dfrac{\dif \phi}{\dif x} + \vv{j} \dfrac{\dif \phi}{\dif y} + \vv{k} \dfrac{\dif \phi}{\dif z} = \left(\dfrac{\dif \phi}{\dif x}, \dfrac{\dif \phi}{\dif y}, \dfrac{\dif \phi}{\dif z} \right).
\end{equation*}

Вектор $\nabla \phi$ даёт в каждой точке величину $|\nabla \phi|$ и направление $\vv{n}$ наибольшей пространственной скорости изменения функции $\phi$:
\begin{equation*}
    \vv{n} = \dfrac{\nabla \phi}{|\nabla \phi|}.
\end{equation*}

Скорость изменения поля $\phi$ в направлении, задаваемом единичным вектором $\vv{l}$ называется производной по направлению и определяется скалярным произведением
\begin{equation*}
    \dfrac{\dif \phi}{\dif l} = \vv{l} \cdot \nabla \phi.
\end{equation*}

Градиент поля $\phi$ в каждой точке пространства направлен по нормали к поверхности уровня.

Угол между поверхностями $\phi_1 \left(x, y, z\right) = const$ и $\phi_2 \left(x, y, z\right) = const$, определяется как угол $\alpha$ между нормалями $\vv{n_1}$ и $\vv{n_2}$ к поверхностям в точке их пересечения
\begin{equation*}
    \cos \alpha = \vv{n_1} \cdot \vv{n_2} = \dfrac{\nabla \phi_1 \cdot \nabla \phi_2}{|\nabla \phi_1| \cdot |\nabla \phi_2|}.
\end{equation*}


\section{Локальные свойства потенциала}
Пусть точка $P\left( x,y,z \right)$ переместилась в точку $P_1\left( x + \dif x, y + \dif y, z +
\dif z \right)$, находящуюся на бесконечно малом расстоянии $dl$ от $P$, тогда полный дифференциал
\begin{equation*}
    \dif V = \pd{V}{x}\dif x + \pd{V}{y}\dif y + \pd{V}{z}\dif z.
\end{equation*}
Для производных и направления $\dif l$ можно записать
\begin{align*}
    \dpd{V}{x} = |\vv{F}|\cos\left( \vv{F}, x \right),\qquad \dif x = \dif l\cos\left( x, \dif l
    \right),\\
    \dpd{V}{y} = |\vv{F}|\cos\left( \vv{F}, y \right),\qquad \dif y = \dif l\cos\left( y, \dif l
    \right),\\
    \dpd{V}{z} = |\vv{F}|\cos\left( \vv{F}, z \right),\qquad \dif z = \dif l\cos\left( z, \dif l
    \right),
\end{align*}
тогда
\begin{equation*}
    \dif V = |\vv{F}|\left[
        \cos\left( \vv{F}, x \right)\cos\left( x, \dif l \right) +
        \cos\left( \vv{F}, y \right)\cos\left( y, \dif l \right) +
    \cos\left( \vv{F}, z \right)\cos\left( z, \dif l \right) \right]\dif l,
\end{equation*}
или
\begin{equation*}
    \dif V = |\vv{F}|cos\left( \vv{F}, \dif l \right)\dif l.
\end{equation*}
Это же легко получить из определения прозводной по направлению, действительно
\begin{equation*}
    \dfrac{\dif V}{\dif l} = \vv{l} \cdot \nabla \phi = \vv{l} \cdot \vv{F} = |\vv{F}|cos\left( \vv{F}, \dif l \right).
\end{equation*}
Из этого выражения силедуют несколько важных и полезных свойств.
\begin{enumerate}
    \item Производная потенциала $\dif V/ \dif l$ по любому направлению $l$ равна проекции силы на
        это направление
        \begin{equation*}
            \od{V}{l} =  |\vv{F}|cos\left( \vv{F}, \dif l \right).
        \end{equation*}
    \item Если $\cos\left( \vv{F}, \dif l \right) = 1$, то изменение потенциала максимально и
        $dl$ направлено по линии действия силы. Тогда $\vv{F}$ является вектор--градиентом
        потенциала
        \begin{equation*}
            \vv{F} = \textrm{grad}\,V.
        \end{equation*}
    \item Если $\cos\left( \vv{F}, \dif l \right) = 0$, то $\dif V = 0$ и 
        \begin{equation*}
            V\left( x, y, z \right) = const.
        \end{equation*}
        Уравнение поверхности, в каждой точке которой сила направлена по нормали, а потенциал ---
        постоянен. Это уровенная или эквипотенциальная поверхность. Заметим, что на силу не
        накладываются никакие ограничения и, вообще говоря, она может меняться на уровенной
        поверхности.
    \item Если $\cos\left( \vv{F}, \dif l \right) = -1$ и $\dif l = \dif h$ --- расстояние между
        бесконечно близкими уровенными поверхностями, то
        \begin{equation*}
            \dif V = - |\vv{F}| \dif h, \qquad \dif h = -\dfrac{\dif V}{|\vv{F}|}.     
        \end{equation*}
    \item Пусть $\dif h\to 0$, тогда получим кривую, перпендикулярную ко всем уровенным
        поверхностям, в каждой точке которой касательная совпадает с направлением силы. Эта кривая
        называется \textbf{силовой линией}. Найдём длину отрезка $OM$ силовой линии между двумя уровенным
        поверхностями
        \begin{equation*}
            h = -\int\limits_{O}^{M} \dfrac{\dif V}{|\vv{F}|} = - \dfrac{V_M - V_O}{F_m} =
            \dfrac{V_O - V_M}{F_m},
        \end{equation*}
        где $F_m$ --- среднее интегральное значение силы на отрезке $OM$. Таким образом, для
        определения высоты в гравитационном поле необходимо знать разность потенциалов и среднее
        значение силы между уровенными поверхностями. Эта формула имеет важное значение в теории
        высот.
\end{enumerate}

%\printbibliography
\end{document}
