\documentclass[11pt, a4paper]{article}

% Languages and fonts
\usepackage{cmap} 
\usepackage[T2A]{fontenc}
\usepackage[utf8]{inputenc} 
\usepackage[english, russian]{babel}
\usepackage{microtype}
\usepackage{indentfirst}
\frenchspacing

% Mathematics
\usepackage{amsmath, amssymb, amsfonts, amsthm, mathtools, fixmath}
\mathtoolsset{showonlyrefs=true}
\usepackage{esint, esvect} % integrals and vectors
\usepackage{systeme} % equation system
\usepackage{commath} % partials and differentials
\usepackage{icomma} % smart comma ($0,2$ is a number)

% Floats
\usepackage{float}

% Tables
\usepackage{array,tabularx,tabulary,booktabs} % better tables
\usepackage{longtable}
\usepackage{multirow}

% Graphics
\usepackage[pdftex]{graphicx}
\usepackage{wrapfig}

% Theorems
\renewcommand{\proofname}{Доказательство}

\theoremstyle{plain}
\newtheorem{theorem}{Теорема}[section]

\theoremstyle{definition}
\newtheorem{definition}{Определение}
\newtheorem{corollary}{Следствие}[theorem]
\newtheorem{problem}{Задача}[section]

\theoremstyle{remark}
\newtheorem{remark}{Замечание}
\newtheorem*{solution}{Решение}

\usepackage[top=20mm,bottom=20mm,left=20mm,right=20mm]{geometry}

\usepackage{lastpage} % how many pages

\usepackage{soul}

\usepackage{framed} % easy frames
\usepackage{enumerate} % better numbered lists

\usepackage{hyperref}
\usepackage{xcolor}

\usepackage{tikz} % drawing

\usepackage{csquotes}
\usepackage[style=numeric,backend=biber,sorting=none]{biblatex}
\addbibresource{../../../../bibliography.bib}

\renewcommand{\epsilon}{\ensuremath{\varepsilon}}
\renewcommand{\phi}{\ensuremath{\varphi}}
%\renewcommand{\theta}{\vartheta}
\renewcommand{\kappa}{\ensuremath{\varkappa}}
\renewcommand{\le}{\ensuremath{\leqslant}}
\renewcommand{\leq}{\ensuremath{\leqslant}}
\renewcommand{\ge}{\ensuremath{\geqslant}}
\renewcommand{\geq}{\ensuremath{\geqslant}}

\usepackage[useregional]{datetime2}

% custom maketitle
\usepackage{titling}
\setlength{\droptitle}{-4em}
\posttitle{\end{center}\vspace{-3em}}

\title{{\Large Геодезическая гравиметрия 2019}\\ 
    {\bf\Large Практическое занятие № 5} \\
{\Large Элементы теории поля}}
\author{}
\DTMsavedate{lessondate}{2019-03-11}
\date{\DTMusedate{lessondate}}

\begin{document}
\maketitle

\section{Уравнения Лапласа и Пуассона}
Потенциал объёмных масс является функцией непрерывной и конечной во всём проcтранстве, т.к. вне притягивающих масс, величина $r$ нигде не может обратиться в нуль, а внутри притягивающих масс $r \to 0$, потенциал не обращается в бесконечность и сохраняет конечное значение. Этими же свойствами обладают и первые производные потенциала.

Потенциал объёмных масс --- функция гармоническая во внешнем пространстве. Для того, чтобы убедиться в этом, найдём вторые производные потенциала объёмных масс в прямоугольных координатах во внешнем пространстве (для внешней точки). Запишем выражение потенциала объёмных масс для \textit{внешней точки} в общем виде
\begin{equation*}
	V = G\iiint\limits_{\tau} \dfrac{\delta \dif \tau}{r},
\end{equation*}
где 
\begin{equation*}
	r = \sqrt{(x-x_1)^2 + (y-y_1)^2 + (z-z_1)^2}.
\end{equation*}
Найдём частные производные $r$
\begin{equation*}
	\dpd{r}{x} = \dfrac{x-x_1}{r}, \\
	\dpd{r}{x} = \dfrac{y-y_1}{r}, \\
	\dpd{r}{x} = \dfrac{z-z_1}{r}. 
\end{equation*}
Найдём первые производные потенциала объёмных масс. Для производной по $x$ получаем
\begin{equation*}
	\dpd{V}{x} = G\iiint\limits_{\tau} \delta \left[\dpd{(r^{-1})}{x}\right] \dif \tau,
\end{equation*}
где
\begin{equation*}
	\dpd{(r^{-1})}{x} = - \dfrac{1}{r^2} \dpd{r}{x} = - \dfrac{x-x_1}{r^3},
\end{equation*}
в итоге
\begin{align*}	
	&\dpd{V}{x} = G\iiint\limits_{\tau} \delta \left[ - \dfrac{x-x_1}{r^3} \right] \dif \tau.
\end{align*}
Аналогично для $y$ и $z$
\begin{align*}	
	&\dpd{V}{y} = G\iiint\limits_{\tau} \delta \left[ - \dfrac{y-y_1}{r^3} \right] \dif \tau, \\
	&\dpd{V}{z} = G\iiint\limits_{\tau} \delta \left[ - \dfrac{z-z_1}{r^3} \right] \dif \tau.
\end{align*}
Найдём его вторые производные. Для производной по $x$ получаем
\begin{equation*}
	 \dpd[2]{V}{x} = G\iiint\limits_{\tau} \delta \left[ - \dpd{}{x} \left(\dfrac{x-x_1}{r^3} \right)\right] \dif \tau,
\end{equation*}
где
\begin{equation*}
	\dpd{}{x} \left(\dfrac{x-x_1}{r^3} \right) = \dpd{(x-x_1)}{x} r^{-3} + \dpd{(r^{-3})}{x} (x-x_1) = \dfrac{1}{r^3} - \dfrac{3}{r^4} \dfrac{x-x_1}{r} (x-x_1) = \dfrac{1}{r^3} - \dfrac{3 \left(x-x_1 \right)^2}{r^5},
\end{equation*}
в итоге
\begin{align*}
    &\dpd[2]{V}{x} = G\iiint\limits_{\tau} \delta \left[ \dfrac{3 \left( x - x_1 \right)^2}{r^5} - \dfrac{1}{r^3} \right]\dif\tau
\end{align*}
Аналогично для $y$ и $z$
\begin{align*}
    &\dpd[2]{V}{y} = G\iiint\limits_{\tau} \delta \left[ \dfrac{3 \left( y - y_1 \right)^2}{r^5} -
    \dfrac{1}{r^3} \right]\dif\tau, \\
    &\dpd[2]{V}{z} = G\iiint\limits_{\tau} \delta \left[ \dfrac{3 \left( z - z_1 \right)^2}{r^5} -
    \dfrac{1}{r^3} \right]\dif\tau.
\end{align*}
Складываем все три равенства и получаем
\begin{equation*}
    \dpd[2]{V}{x} + \dpd[2]{V}{y} + \dpd[2]{V}{y} = 
    G\iiint\limits_{\tau} \delta \left[ \dfrac{3 \left( x - x_1 \right)^2 +
        3 \left( y - y_1 \right)^2 +3 \left( z - z_1 \right)^2}{r^5} - \dfrac{3}{r^3} 
    \right]\dif\tau =
    G\iiint\limits_{\tau} \delta \left[ \dfrac{3}{r^3} - \dfrac{3}{r^3} \right]\dif\tau = 0.
\end{equation*}
То есть
\begin{equation*}
    \Delta V = \dpd[2]{V}{x} + \dpd[2]{V}{y} + \dpd[2]{V}{y} = 0.
\end{equation*}
Это \textbf{уравнение Лапласа} --- дифференциальное уравнение в частных производных второго порядка. Здесь
$\Delta = \dpd[2]{}{x} + \dpd[2]{}{y} + \dpd[2]{}{y}$ --- оператор Лапласа или <<лапласиан>>.
Функция называется \textbf{гармонической}, если:
\begin{enumerate}
	\item существуют частные производные второго порядка этой функции;
	\item все они непрерывны;
	\item удовлетворяют уравнению Лапласа $\Delta V = 0$.
\end{enumerate}
Таким образом, \textbf{потенциал притяжения во внешнем пространстве является гармонической функцией}.

\begin{problem}
Доказать, что функция $f (x, y) = \ln (x^2 + y^2)$ является гармонической везде, кроме начала координат $(x = 0, y = 0)$.
\end{problem}

Найдём оператор Лапласа потенциала объёмных масс $\Delta V$ для внешней точки. Рассмотрим случай, когда притягиваемая точка $P$ находится внутри притягивающего тела объёма $\tau$, ограниченного поверхностью $\Sigma$ произвольной формы. Окружим точку $P$ сферой настолько малого радиуса $R$, что плотность внутри этой сферы можно считать постоянной. Тогда потенциал тела в точке $P$ можно разделить на две части $V = V_1 + V_2$:
\begin{enumerate}
	\item потенциал $V_1$ однородного шара радиуса $R$, внутри которого находится точка $P$;
	\item на потенциал $V_2$ всех остальных масс вне шара.
\end{enumerate}
Применим оператор Лапласа на левую и правую части суммы: $\Delta V = \Delta (V_1 + V_2) = \Delta V_1 + \Delta V_2$.
Потенциал $V_1$ будет определяться выражением:
\begin{equation*}
	V_1 = \dfrac{2}{3} \pi G \delta \left(3R^3 - r^2 \right), 
	r = \sqrt{x^2 + y^2 + z^2}.
\end{equation*}
Найдём частные производные $r$
\begin{equation*}
	\dpd{r}{x} = \dfrac{x}{r}, \\
	\dpd{r}{x} = \dfrac{y}{r}, \\
	\dpd{r}{x} = \dfrac{z}{r}. 
\end{equation*}
Найдём первые производные потенциала однородного шара для внутренней точки $P$. Для производной по $x$ получаем
\begin{equation*}
	\dpd{V_1}{x} = \dfrac{2}{3} \pi G \delta \left(3R^3 - r^2 \right) = - \dfrac{2}{3} \pi G \delta \left[\dpd{(r^{2})}{x} \right],
\end{equation*}
где
\begin{equation*}
	\dpd{(r^{2})}{x} = 2 r \dpd{r}{x} = 2x,
\end{equation*}
в итоге
\begin{equation*}
    \dpd{V_1}{x} = - \dfrac{4}{3} \pi G \delta x.
\end{equation*}
Аналогично для $y$ и $z$
\begin{equation*}	
    \dpd{V_1}{y} = - \dfrac{4}{3} \pi G \delta y, 
    \dpd{V_1}{z} = - \dfrac{4}{3} \pi G \delta z.
\end{equation*}
Найдём вторые производные по $x, y, z$. Для производной по $x$ получаем:
\begin{equation*}
	\dpd[2]{V_1}{x} =  - \dfrac{4}{3} \pi G \delta \dpd{x}{x} = -\dfrac{4}{3} \pi G \delta.
\end{equation*}
Аналогично для $y$ и $z$:
\begin{equation*}
	\dpd[2]{V_1}{y} = - \dfrac{4}{3} \pi G \delta,    
	\dpd[2]{V_1}{z} = - \dfrac{4}{3} \pi G \delta.
\end{equation*}
Для потенциала $V_2$ точка $P$ является лежащей вне притягивающих масс, а значит удовлетворяет уравнению Лапаласа $\Delta V_2 = 0$. Окончательно получаем:
\begin{equation*}
	\Delta V = \dpd[2]{V_1}{x} + \dpd[2]{V_1}{y} + \dpd[2]{V_1}{z} + \dpd[2]{V_2}{x} + \dpd[2]{V_2}{y} + \dpd[2]{V_2}{z} = -4 \pi G \delta + 0 = -4 \pi G \delta,
\end{equation*}
или
\begin{equation*}
	\Delta V = \dpd[2]{V_1}{x} + \dpd[2]{V_1}{y} + \dpd[2]{V_1}{z} = -4 \pi G \delta.
\end{equation*}
Это \textbf{уравнение Пуассона}, которому удовлетворяет потенциал притяжения внутри притягивающих масс. Таким образом, потенциал притяжения внутри притягивающих масс \textbf{не является гармонической функцией}. Легко заметить, что вне притягивающих масс уравнение Пуассона переходит в уравнение Лапласа.

\section{Регулярность на бесконечности}
Исследуем поведение потенциала на бесконечности. Напишем неравенство
\begin{equation*}
    G\int\limits_\tau\dfrac{\delta\dif\tau}{r_{min}} >
    G\int\limits_\tau\dfrac{\delta\dif\tau}{r} >
    G\int\limits_\tau\dfrac{\delta\dif\tau}{r_{max}},
\end{equation*}
где $r_{min}$, $r_{max}$ --- минимальное и максимальное расстояние от притягиваемой точки до
притягиваемого тела. Поскольку $\int\limits_\tau\delta\dif\tau = M$, то
\begin{equation*}
    G\dfrac{M}{r_{min}} >
    G\dfrac{M}{r} >
    G\dfrac{M}{r_{max}}.
\end{equation*}
Образуем производную $\pd{V}{r} = -G\int\limits_\tau\dfrac{\delta\dif\tau}{r^2}$ и неравенство
\begin{equation*}
    G\dfrac{M}{r^2_{min}} >
     \left|\dpd{V}{r}\right| >
    G\dfrac{M}{r^2_{max}}.
\end{equation*}
Теперь умножаем предпоследнее неравенство на $r$, а последнее --- на $r^2$, тогда
\begin{equation*}
    G\dfrac{Mr}{r_{min}} >
    Vr >
    G\dfrac{Mr}{r_{max}}, \qquad
    G\dfrac{Mr^2}{r^2_{min}} >
     \left|\dpd{V}{r}\right| r^2 >
    G\dfrac{M r^2}{r^2_{max}}.
\end{equation*}
Пусть $r\to\infty$, тогда
\begin{equation*}
    \lim\limits_{r\to\infty}\dfrac{r}{r_{min}} = 1,\qquad 
    \lim\limits_{r\to\infty}\dfrac{r}{r_{max}} = 1,
\end{equation*}
и
\begin{equation*}
    \lim\limits_{r\to\infty} V = 0,\qquad 
    \lim\limits_{r\to\infty} rV = GM,\qquad
    \lim\limits_{r\to\infty} \left|\dpd{V}{r}\right| r^2 = GM.
\end{equation*}
Функция, удовлетворяющая всем трём последним пределам называется \textbf{регулярной на
бесконечности}. Следовательно, потенциал является функцией, регулярной на бесконечности. Иными словами, в бесконечно удалённой точке потенциал обращается в нуль.

\section{Скалярное поле, градиент и производная по направлению}
Если каждой точке пространства или его части однозначно сопоставлена некоторая скалярная (векторная) величина, то говорят, что задано скалярное (векторное) поле этой величины. 

Поверхностью уровня скалярного поля $\phi = \phi \left(x, y, z\right)$ называется множество точек пространства, в которых функция $\phi$ принимает одно и то же значение $c$, то есть поверхность уровня определяется уравнением 
\begin{equation*}
    \phi \left(x, y, z\right) = c.
\end{equation*}
Набор поверхностей уровня для разных $c$ дает наглядное представление о конкретном скалярном поле, для которого они построены (изображены).
Если скалярное поле $\phi$ является \textit{потенциальным полем}, то поверхность уровня такого поля называется \textit{эквипотенциальной}.

Рассмотрим изменение скалярного поля $\dif \phi\left(x, y, z\right)$ при смещении точки $P\left( x, y,z \right)$ на бесконечно малый вектор $\vv{\dif l}$ в точку $P_1\left( x + \dif x, y + \dif y, z +
\dif z \right)$.
Изменение функции $\phi$ можно представить в виде
\begin{equation*}
    \dif \phi = \pd{\phi}{x}\dif x + \pd{\phi}{y}\dif y + \pd{\phi}{z}\dif z.
\end{equation*}
Выражение в правой части этого равенства можно интерпретировать как скалярное произведение векторов, одним из которых является вектор смещения $\vv{\dif l} = \{\dif x, \dif y, \dif z\}$, а другим – вектор $\{\pd{\phi}{x}, \pd{\phi}{y}, \pd{\phi}{z}\}$, называемый \textbf{градиентом скалярного поля}: 
\begin{equation*}
    \text{grad } \phi = \nabla \phi = \vv{i} \dfrac{\dif \phi}{\dif x} + \vv{j} \dfrac{\dif \phi}{\dif y} + \vv{k} \dfrac{\dif \phi}{\dif z} = \left(\dfrac{\dif \phi}{\dif x}, \dfrac{\dif \phi}{\dif y}, \dfrac{\dif \phi}{\dif z} \right).
\end{equation*}
Таким образом, изменение поля
\begin{equation*}
    \dif \phi = \text{grad } \phi \cdot \vv{\dif l} = |\text{grad} \phi|  |\vv{\dif l}|  \cos \theta,
\end{equation*}
где $\theta$ --- угол между векторами $\text{grad } \phi$ и $\vv{\dif l}$.

Из этого равенства следует, что изменение $\dif \phi$ принимает наибольшее значение, если направление вектора смещения $\vv{\dif l}$ совпадает с направлением $\text{grad } \phi$. Другими словами, направление вектора  $\text{grad } \phi$ соответствует направлению вектора  $\vv{n}$ наиболее быстрого возрастания функции $\phi$
\begin{equation*}
    \vv{n} = \dfrac{\text{grad } \phi}{|\text{grad } \phi|}.
\end{equation*}

Производная по направлению показывает быстроту изменения функции в этом направлении. Её можно рассматривать как проекцию градиента функции на это направление, или иначе, как скалярное произведение градиента на орт $\vv{l} = \dfrac{\vv{\dif l}}{|\vv{\dif l}|} = \dfrac{\vv{\dif l}}{\dif l}$ направления 
\begin{equation*}
     \dfrac{\dif \phi}{\dif l} = \text{grad } \phi \cdot \vv{l}.
\end{equation*}

Градиент поля $\phi$ в каждой точке пространства направлен по нормали к поверхности уровня.

Угол между поверхностями $\phi_1 \left(x, y, z\right) = const$ и $\phi_2 \left(x, y, z\right) = const$, определяется как угол $\alpha$ между нормалями $\vv{n_1}$ и $\vv{n_2}$ к поверхностям в точке их пересечения
\begin{equation*}
    \cos \alpha = \vv{n_1} \cdot \vv{n_2} = \dfrac{\text{grad } \phi_1 \cdot \text{grad } \phi_2}{|\text{grad } \phi_1| \cdot |\text{grad } \phi_2|}.
\end{equation*}

%\printbibliography
\end{document}
