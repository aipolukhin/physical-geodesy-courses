\documentclass[11pt, a4paper]{article}

% Languages and fonts
\usepackage{cmap} 
\usepackage[T2A]{fontenc}
\usepackage[utf8]{inputenc} 
\usepackage[english, russian]{babel}
\usepackage{microtype}
\usepackage{indentfirst}
\frenchspacing

% Mathematics
\usepackage{amsmath, amssymb, amsfonts, amsthm, mathtools, fixmath}
\mathtoolsset{showonlyrefs=true}
\usepackage{esint, esvect} % integrals and vectors
\usepackage{systeme} % equation system
\usepackage{commath} % partials and differentials
\usepackage{icomma} % smart comma ($0,2$ is a number)

% Floats
\usepackage{float}

% Tables
\usepackage{array,tabularx,tabulary,booktabs} % better tables
\usepackage{longtable}
\usepackage{multirow}

% Graphics
\usepackage[pdftex]{graphicx}
\usepackage{wrapfig}

% Theorems
\renewcommand{\proofname}{Доказательство}

\theoremstyle{plain}
\newtheorem{theorem}{Теорема}[section]

\theoremstyle{definition}
\newtheorem{definition}{Определение}
\newtheorem{corollary}{Следствие}[theorem]
\newtheorem{problem}{Задача}[section]

\theoremstyle{remark}
\newtheorem{remark}{Замечание}
\newtheorem*{solution}{Решение}

\usepackage[top=20mm,bottom=20mm,left=20mm,right=20mm]{geometry}

\usepackage{lastpage} % how many pages

\usepackage{soul}

\usepackage{framed} % easy frames
\usepackage{enumerate} % better numbered lists

\usepackage{hyperref}
\usepackage{xcolor}

\usepackage{tikz} % drawing

\usepackage{csquotes}
\usepackage[style=numeric,backend=biber,sorting=none]{biblatex}
\addbibresource{../../../../bibliography.bib}

\renewcommand{\epsilon}{\ensuremath{\varepsilon}}
\renewcommand{\phi}{\ensuremath{\varphi}}
%\renewcommand{\theta}{\vartheta}
\renewcommand{\kappa}{\ensuremath{\varkappa}}
\renewcommand{\le}{\ensuremath{\leqslant}}
\renewcommand{\leq}{\ensuremath{\leqslant}}
\renewcommand{\ge}{\ensuremath{\geqslant}}
\renewcommand{\geq}{\ensuremath{\geqslant}}

\usepackage[useregional]{datetime2}

% custom maketitle
\usepackage{titling}
\setlength{\droptitle}{-4em}
\posttitle{\end{center}\vspace{-3em}}

\title{{\Large Геодезическая гравиметрия 2019}\\ 
    {\bf\Large Практическое занятие № 3} \\
{\Large Притяжение тел простой формы $I$}}
\author{}
\DTMsavedate{lessondate}{2019-02-25}
\date{\DTMusedate{lessondate}}

\begin{document}
\maketitle

\section{Притяжение простого слоя}

Потенциал объемного тела может быть представлен в виде потенциалов простого слоя, двойного слоя или
их комбинации. Это полезное свойство, которое часто используется в теории фигур планет при
решении теоретических задач. Такая замена позволяет перейти от интегрирования по объёму к
интегрированию по поверхности (то есть от тройного интеграла к двойному). Рассмотрим потенциал
простого слоя.

Пусть в объеме $\tau$, заключенном между двумя очень близкими, поверхностями $\sigma$ и $\sigma'$
находится притягивающая масса с переменной объёмной плотностью $\delta$, тогда потенциал объёмных
масс будет равен
\begin{equation*}
    V = G\iiint\limits_{\tau}\dfrac{\delta \left( x, y, z \right)\dif\tau}{r},
\end{equation*}
где интегрирование ведётся по всему объёму $\tau$, а $r$ --- расстояние от текущей 
точки до притягиваемой $P$.

\textbf{Элементарный объём}
\begin{equation*}
    \dif\tau = \dif h\dif\sigma,
\end{equation*}
где $dh$ --- кратчайшее расстояние между $\sigma$ и $\sigma'$,
$\dif\sigma$ --- элемент площади поверхности. \\

\textbf{Элементарная масса}
\begin{equation*}
    \dif m = \delta\dif\tau = \delta\dif h\dif\sigma.
\end{equation*}

Пусть $\sigma$ и $\sigma'$ неограниченно приближаются друг к другу, тогда вся элементарная масса
$\dif m$ элементарного объёма $\dif\tau$ будет сконденсирована на бесконечно тонком слое площадью
$\dif\sigma$. Такой слой называется простым слоем, а его поверхностная плотность (или плотность
простого слоя) равна
\begin{equation*}
    \mu = \dfrac{\dif m }{\dif \sigma},
\end{equation*}
откуда
\begin{equation*}
    \dif m = \delta\dif\tau = \delta\dif h\dif\sigma = \mu\dif\sigma. 
\end{equation*}
Элементарный потенциал, создаваемый массой $\dif m$ будет равен
\begin{equation*}
    \dif V = G\dfrac{\mu\dif\sigma}{r},
\end{equation*}
откуда, интегрируя по поверхности $\sigma$, получаем потенциал притяжения простого слоя
\begin{equation*}
    V = G\iint\limits_{\sigma}\dfrac{\mu\dif\sigma}{r}.
\end{equation*}

\section{Притяжение однородной сферы (сферического слоя)}
Пусть простой слой постоянной плотности $\mu$ распределён на сфере радиуса $R$. Найдём притяжение
такой однородной сферы в точке $P$, находящейся на расстоянии $r$ от центра сферы
$O$. $\rho$ --- переменное расстояние от элементарной площади $\dif\sigma$ поверхности сферы до точки
$P$. 

Воспользуемся сферической системой координат ($\theta$, $\lambda$). Пусть элемент $\dif\sigma$
представляет собой криволинейную трапецию, ограниченную меридианами $\lambda$, $\lambda +
\dif\lambda$ и параллелями $\theta$, $\theta + \dif\theta$. Тогда стороны трапеции будут равны
\begin{align*}
    &R\dif\theta\ \text{--- длина дуги меридиана}, \\
    &R\sin\theta\dif\lambda\ \text{--- длина дуги параллели},
\end{align*}
а элементарная площадь
\begin{equation*}
    \dif\sigma = R\sin\theta\dif\lambda\cdot R\dif\theta = R^2\sin\theta\dif\theta\dif\lambda.
\end{equation*}
По теореме косинусов
\begin{equation}
    \label{eq:thcos}
    \rho^2 = R^2 + r^2 - 2Rr\cos\theta.
\end{equation}
Тогда для простого слоя, распределённого на поверхности сферы, можно записать
\begin{align*}
    V  =  
    G\mu\int\limits_{0}^{\pi}\int\limits_{0}^{2\pi}
    \dfrac{R^2\sin\theta\dif\theta\dif\lambda}{\sqrt{R^2 + r^2 - 2Rr\cos\theta}} 
    & =
    G\mu R^2\int\limits_{0}^{\pi}
    \dfrac{\sin\theta\dif\theta}{\sqrt{R^2 + r^2 - 2Rr\cos\theta}}
    \int\limits_{0}^{2\pi}\dif\lambda = \\
    & =  2\pi G\mu R^2\int\limits_{0}^{\pi}
    \dfrac{\sin\theta\dif\theta}{\sqrt{R^2 + r^2 - 2Rr\cos\theta}}.
\end{align*}
Выполним замену переменных. Дифференцируя $\rho^2 = R^2 + r^2 - 2Rr\cos\theta$, 
получаем
\begin{equation*}
    \dfrac{\dif \rho}{\dif \theta} = \dfrac{Rr\sin\theta}{\rho},\quad \sin\theta\dif\theta = \dfrac{\rho\dif \rho}{Rr},
\end{equation*}
тогда
\begin{equation*}
    V =  2\pi G\mu R^2\int\limits_{0}^{\pi}
    \dfrac{\sin\theta\dif\theta}{\sqrt{R^2 + r^2 - 2Rr\cos\theta}} =
    2\pi G\mu \dfrac{R}{r}\int\limits_{\rho_1}^{\rho_2}
    \dif \rho.
\end{equation*}
Если точка $P$ --- внешняя, то $\rho_1 = r - R$, $\rho_2 = r + R$, тогда
\begin{equation*}
    V_e =
    2\pi G\mu \dfrac{R}{r}\int\limits_{r - R}^{r + R} \dif\rho =
    2\pi G\mu \dfrac{R}{r}\left[ \left( r + R \right) - \left( r - R \right) \right] =
    \underline{4\pi G\mu \dfrac{R^2}{r}}.
\end{equation*}
Поскольку масса всего сферического слоя равна $M = 4\pi R^2\mu$, то окончательно получаем
\begin{equation*}
    V_e = \dfrac{GM}{r}.
\end{equation*}
Если точка $P$ --- внутренняя, то $\rho_1 = R - r$, $\rho_2 = R + r$, тогда
\begin{equation*}
    V_i =
    2\pi G\mu \dfrac{R}{r}\int\limits_{R - r}^{R + r} \dif\rho =
    2\pi G\mu \dfrac{R}{r}\left[ \left( R + r \right) - \left( R - r \right) \right] =
    \underline{4\pi G\mu R = const}.
\end{equation*}

При неограниченном приближении точки $P$ к поверхности сферы $r\to R$, поэтому значение внешнего потенциала будет равно
\begin{equation*}
    V_e = \lim_{r\to R} 4\pi G\mu \dfrac{R^2}{r} = 4\pi G\mu R.
\end{equation*}
Таким образом, на поверхности сферы $V_e = V_i$. Следовательно, потенциал сферы --- функция, непрерывная во всём пространстве.

Поскольку потенциал притяжения зависит только от расстояния от притягивающей точки до центра сферы,
то для нахождения силы для внешней точки достаточно вычислить радиальную производную
\begin{equation*}
    |\vv{F_e}| = F_r = -\pd{V}{r} = 4\pi G\mu \dfrac{R^2}{r^2} = \dfrac{GM}{r^2}.
\end{equation*}
Для внутренней точки 
\begin{equation*}
    |\vv{F_i}| = F_r = -\pd{V}{r} = 0.
\end{equation*}
При неограниченном приближении точки $P$ к поверхности сферы $P_0$ с внешней и с внутренней стороны, получим
\begin{align*}
    &F_{e_0} = \lim_{r\to R} -\dpd{V}{r} = \lim_{r\to R} 4\pi G\mu \dfrac{R^2}{r^2} = 4\pi G\mu, \\
    &F_{i_0} = \lim_{r\to R} -\dpd{V}{r} = 0.
\end{align*}

Прямое значение силы на самой поверхности сферы $P_0$ равно среднему из предельных значений, то есть
\begin{equation*}
    F_0 = 2\pi G\mu.
\end{equation*}

\begin{problem}
    Как ведут себя потенциал и сила притяжения сферы, если притягиваемая точка премещается из центра
    сферы в бесконечность?
    Постройте графики зависимости потенциала и силы притяжения сферы от расстояния.
\end{problem}
%\printbibliography
\end{document}
