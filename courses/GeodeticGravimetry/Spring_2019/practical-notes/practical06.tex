\documentclass[11pt, a4paper]{article}

% Languages and fonts
\usepackage{cmap} 
\usepackage[T2A]{fontenc}
\usepackage[utf8]{inputenc} 
\usepackage[english, russian]{babel}
\usepackage{microtype}
\usepackage{indentfirst}
\frenchspacing

% Mathematics
\usepackage{amsmath, amssymb, amsfonts, amsthm, mathtools, fixmath}
\usepackage{esint, esvect} % integrals and vectors
\usepackage{systeme} % equation system
\usepackage{commath} % partials and differentials
\usepackage{icomma} % smart comma ($0,2$ is a number)

% Floats
\usepackage{float}

% Tables
\usepackage{array,tabularx,tabulary,booktabs} % better tables
\usepackage{longtable}
\usepackage{multirow}

% Graphics
\usepackage[pdftex]{graphicx}
\usepackage{wrapfig}

% Theorems
\renewcommand{\proofname}{Доказательство}

\theoremstyle{plain}
\newtheorem{theorem}{Теорема}[section]

\theoremstyle{definition}
\newtheorem{definition}{Определение}
\newtheorem{corollary}{Следствие}[theorem]
\newtheorem{problem}{Задача}[section]

\theoremstyle{remark}
\newtheorem{remark}{Замечание}
\newtheorem*{solution}{Решение}

\usepackage[top=20mm,bottom=20mm,left=20mm,right=20mm]{geometry}

\usepackage{lastpage} % how many pages

\usepackage{soul}

\usepackage{framed} % easy frames
\usepackage{enumerate} % better numbered lists

\usepackage{hyperref}
\usepackage{xcolor}

\usepackage{tikz} % drawing

\usepackage{csquotes}
\usepackage[style=numeric,backend=biber,sorting=none]{biblatex}
\addbibresource{../../../../bibliography.bib}

\renewcommand{\epsilon}{\ensuremath{\varepsilon}}
\renewcommand{\phi}{\ensuremath{\varphi}}
%\renewcommand{\theta}{\vartheta}
\renewcommand{\kappa}{\ensuremath{\varkappa}}
\renewcommand{\le}{\ensuremath{\leqslant}}
\renewcommand{\leq}{\ensuremath{\leqslant}}
\renewcommand{\ge}{\ensuremath{\geqslant}}
\renewcommand{\geq}{\ensuremath{\geqslant}}

\usepackage[useregional]{datetime2}

% custom maketitle
\usepackage{titling}
\setlength{\droptitle}{-4em}
\posttitle{\end{center}\vspace{-3em}}

\title{{\Large Геодезическая гравиметрия 2019}\\ 
    {\bf\Large Практическое занятие № 6} \\
{\Large Гравитационное поле Земли}}
\author{}
\DTMsavedate{lessondate}{2019-03-18}
\date{\DTMusedate{lessondate}}

\begin{document}
\maketitle
\section{Локальные свойства потенциала}
Рассмотрим изменение потенциала $\dif V \left(x, y, z\right)$ при смещении точки $P\left( x, y,z \right)$ на бесконечно малый вектор $\vv{\dif l}$ в точку $P_1\left( x + \dif x, y + \dif y, z + \dif z \right)$.
Изменение потенциала $\phi$ можно представить в виде
\begin{equation*}
    \dif V = \pd{V}{x}\dif x + \pd{V}{y}\dif y + \pd{V}{z}\dif z, 
\end{equation*}
иначе - как скалярное произведение векторов, один из которых является вектором смещения $\vv{\dif l} = \{\dif x, \dif y, \dif z\}$, а другой – градиентом потенциала $\{\pd{V}{x}, \pd{V}{y}, \pd{V}{z}\}$, который, насколько мы уже успели узнать, равен вектору силы притяжения $\vv{F}$
\begin{equation*}
    \dif V = \text{grad } V \cdot \vv{\dif l} = \vv{F} \cdot \vv{\dif l} = |\vv{F}| |\vv{\dif l}| \cos \left(\vv{F}, \vv{\dif l}\right).
\end{equation*}
Производную потенциала по направлению получим как скалярное произведение градиента потенциала на орт $\vv{l} = \dfrac{\vv{\dif l}}{|\vv{\dif l}|} = \dfrac{\vv{\dif l}}{\dif l}$ данного направления 
\begin{equation}
    \dfrac{\dif V}{\dif l} = \text{grad } V \cdot \vv{l} = \vv{F} \cdot \vv{l} = | \vv{F}| \cos \left(\vv{F}, \vv{l}\right).
\end{equation}

Из этого выражения следуют несколько важных и полезных свойств.
\begin{enumerate}
    \item Производная потенциала $\dif V/ \dif l$ по любому направлению $\vv{l}$ равна проекции силы на
        это направление
        \begin{equation*}
            \od{V}{l} =  |\vv{F}|\cos \left(\vv{F}, \vv{l}\right).
        \end{equation*}
    \item Если направление перемещения $\vv{l}$ совпадает с направлением действия силы $\vv{F}$: $\cos\left( \vv{F}, \vv{l} \right) = 1$, то изменение потенциала будет максимальным.
    \item Если направление перемещения $\vv{l}$ перпендикулярно направлению действия силы $\vv{F}$: $\cos\left( \vv{F}, \vv{l} \right) = 1$, то, то $\dif V = 0$ и 
        \begin{equation*}
            V\left( x, y, z \right) = c.
        \end{equation*}
        Таким образом, мы получили уравнение поверхности уровня, в каждой точке которой потенциал $V$ принимает одно и тоже значение $c$, a вектор силы $\vv{F}$ направлен по нормали к ней. Поверхность уровня потенциала $V$ называется \textit{эквипотенциальной}. Заметим, что на силу не накладываются никакие ограничения и, вообще говоря, она может меняться на уровенной поверхности.
    \item Если направление перемещения $\vv{l}$ совпадает с противоложным направлением действия силы $\vv{F}$: $\cos\left( \vv{F}, \vv{l} \right) = -1$, то $\dif V = - |\vv{F}| \dif l$. Положим, что $\dif l = \dif h$ --- расстояние между бесконечно близкими уровенными поверхностями, то
        \begin{equation*}
            \dif V = - |\vv{F}| \dif h, \qquad \dif h = -\dfrac{\dif V}{|\vv{F}|}.     
        \end{equation*}
        Согласно этому выражению, расстояние $\dif h$ между уровенными поверхностями обратно пропорционально величине силы и поэтому уровенные поверхности непараллельны.

	\item Представим себе семейство уровенных поверхностей и отрезки $\dif h$, перпендикулярные к соседним поверхностям. Эти отрезки образуют ломаную линию. Если отрезки $\dif h$ устремить к нулю, то, в пределе, ломаная линия превратится в кривую, пересекающую все уровенные поверхности по нормали. Эта кривая называется \textbf{силовой линией}. Найдём длину отрезка $OM$ силовой линии между двумя уровенным
        поверхностями
        \begin{equation*}
            h = -\int\limits_{O}^{M} \dfrac{\dif V}{|\vv{F}|} = - \dfrac{V_M - V_O}{F_m} =
            \dfrac{V_O - V_M}{F_m},
        \end{equation*}
        где $F_m$ --- среднее интегральное значение силы на отрезке $OM$. Таким образом, для
        определения высоты в гравитационном поле необходимо знать разность потенциалов и среднее
        значение силы между уровенными поверхностями. Эта формула имеет важное значение в теории
        высот.
\end{enumerate}

%\printbibliography
\end{document}
