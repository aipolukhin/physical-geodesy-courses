\documentclass[11pt, a4paper]{article}

% Languages and fonts
\usepackage{cmap} 
\usepackage[T2A]{fontenc}
\usepackage[utf8]{inputenc} 
\usepackage[english, russian]{babel}
\usepackage{microtype}
\usepackage{indentfirst}
\frenchspacing

% Mathematics
\usepackage{amsmath, amssymb, amsfonts, amsthm, mathtools, fixmath}
\usepackage{esint, esvect} % integrals and vectors
\usepackage{systeme} % equation system
\usepackage{commath} % partials and differentials
\usepackage{icomma} % smart comma ($0,2$ is a number)

% Floats
\usepackage{float}

% Tables
\usepackage{array,tabularx,tabulary,booktabs} % better tables
\usepackage{longtable}
\usepackage{multirow}

% Graphics
\usepackage[pdftex]{graphicx}
\usepackage{wrapfig}

% Theorems
\renewcommand{\proofname}{Доказательство}

\theoremstyle{plain}
\newtheorem{theorem}{Теорема}[section]

\theoremstyle{definition}
\newtheorem{definition}{Определение}
\newtheorem{corollary}{Следствие}[theorem]
\newtheorem{problem}{Задача}[section]

\theoremstyle{remark}
\newtheorem{remark}{Замечание}
\newtheorem*{solution}{Решение}

\usepackage[top=20mm,bottom=20mm,left=20mm,right=20mm]{geometry}

\usepackage{lastpage} % how many pages

\usepackage{soul}

\usepackage{framed} % easy frames
\usepackage{enumerate} % better numbered lists

\usepackage{hyperref}
\usepackage{xcolor}

\usepackage{tikz} % drawing

\usepackage{csquotes}
\usepackage[style=numeric,backend=biber,sorting=none]{biblatex}
\addbibresource{../../../../bibliography.bib}

\renewcommand{\epsilon}{\ensuremath{\varepsilon}}
\renewcommand{\phi}{\ensuremath{\varphi}}
%\renewcommand{\theta}{\vartheta}
\renewcommand{\kappa}{\ensuremath{\varkappa}}
\renewcommand{\le}{\ensuremath{\leqslant}}
\renewcommand{\leq}{\ensuremath{\leqslant}}
\renewcommand{\ge}{\ensuremath{\geqslant}}
\renewcommand{\geq}{\ensuremath{\geqslant}}

\usepackage[useregional]{datetime2}

% custom maketitle
\usepackage{titling}
\setlength{\droptitle}{-4em}
\posttitle{\end{center}\vspace{-3em}}

\title{{\Large Геодезическая гравиметрия 2019}\\ 
    {\bf\Large Практическое занятие № 6} \\
{\Large Гравитационное поле Земли I}}
\author{}
\DTMsavedate{lessondate}{2019-03-18}
\date{\DTMusedate{lessondate}}

\begin{document}
\maketitle
\section{Сила тяжести}
Полная сила, действующая на каждую частицу Земли, является равнодействующей силы $\vv{F}$ притяжения всех масс Земли и центробежной силы $\vv{P}$ вращения Земли и называется силой тяжести $\vv{g}$
\begin{equation*}
    \vv{g} = \vv{F} + \vv{P}.
\end{equation*}

В геодезии принято считать силу тяжести неизменной во времени, полагая, что Земля - твёрдое тело, которое вращается с постоянной угловой скоростью вокруг неизменной оси. Строго говоря, силу тяжести нельзя считать абсолютно неизменной из-за переменного притяжения Луны и Солнца, тектонических процессов в Земле, движения масс атмосферы и других явлений. Однако, все эти эффекты относительно невелики (не более $0,3 \cdot 10^{-6} g$), и, при необходимости, их можно учесть с приемлемой точностью.

Сначала рассмотрим действие центробежной силы. Для этого зададим прямоугольную систему координат с началом в центре тяжести Земли, осью $z$, совпадающей с осью вращения Земли, осью $x$, cовпадающей с пересечением плоскости экватора и гринвичского меридиана и осью $y$,  дополняющей систему до правой. Величина центробежной силы $P$, действующей на единичную массу, равна
\begin{equation*}
    P = \omega^2 l,
\end{equation*}
где $\omega$ --- угловая скорость вращения Земли, $l$ --- радиус параллели
\begin{equation*}
	l = \sqrt{x^2 + y^2},
\end{equation*}
где $x,\; y$ --- координаты точки $M$.
Вектор $\vv{P}$ центробежной силы  сонаправлен с вектором $\vv{l}$
\begin{equation*}
	\vv{l} = \left\lbrace x, y, 0\right\rbrace.
\end{equation*}
В соответствии с правилом умножения вектора на скалярную величину, получим вектор центробежной силы
\begin{equation*}
	\vv{P} = \omega^2 \cdot \vv{l} = \left\lbrace \omega^2 x, \omega^2 y, 0\right\rbrace.
\end{equation*}
Потенциал $Q$ центробежной силы $\vv{P}$ равен
\begin{equation*}
	Q = \dfrac{\omega^2}{2} \left(x^2+y^2\right).
\end{equation*}
Мы можем в этом убедиться, исходя из того что вектор силы должен быть равен градиенту потенциала
\begin{equation*}
    \vv{P} = \text{grad } Q = \left\lbrace \dpd{Q}{x}, \dpd{Q}{y}, \dpd{Q}{z} \right\rbrace = \left\lbrace \omega^2 x, \omega^2 y, 0\right\rbrace.
\end{equation*}

Сила тяжести $\vv{g}$, как уже было сказано, является равнодействующей силы притяжения $\vv{F}$ и центробежной силы $\vv{P}$, тогда потенциал $W$ силы тяжести  будет равен сумме потенциалов $F$ силы притяжения  и $Q$ центробежной силы
\begin{equation*}
	W = V + Q = G \iiint \limits_{\tau} \dfrac{\delta \dif \tau}{r} +  \dfrac{\omega^2}{2} \left(x^2+y^2\right).
\end{equation*}
Вектор силы тяжести $\vv{g}$ равен
\begin{equation*}
	\vv{g} =  \text{grad } W = \left\lbrace \dpd{W}{x}, \dpd{W}{y}, \dpd{W}{z} \right\rbrace.
\end{equation*}

Считая Землю однородным вращающимся шаром, радиус параллели можно представить как
\begin{equation*}
	l = R \cos \overline{\phi}, 
\end{equation*}
где $R$ --- радиус шара, $\overline{\phi}$ --- геоцентрическая широта точки. Таким образом, радиальная составляющая центробежной силы
\begin{equation*}
	P \cos \overline{\phi} = \omega^2 R \cos^2 \overline{\phi}, 
\end{equation*}
в итоге, величина силы тяжести будет равна
\begin{equation*}
	g = \dfrac{GM}{R^2} + \omega^2 R \cos^2 \overline{\phi}. 
\end{equation*}
Для потенциала силы тяжести запишем
\begin{equation*}
	W = \dfrac{GM}{R} + \dfrac{\omega^2}{2} R^2 \cos^2 \overline{\phi}. 
\end{equation*}
\section{Локальные свойства потенциала силы тяжести}
Центробежный потенциал $Q$ на оси вращения $z$ равен нулю, а на экваторе достигает максимальной величины. Казалось бы, что при бесконечном удалении точки от поверхности Земли потенциал $Q$ будет бесконечно большим, но физически он существует только в той области, в которой точки участвуют в суточном вращении Земли. Таким образом, в пределах этой области потенциал $Q$ будет конечной и непрерывной функцией.

Найдём лапласиан центробежного потенциала $Q$, вычислив вторые производные
\begin{equation*}
	\Delta Q = \dpd[2]{Q}{x} + \dpd[2]{Q}{y} + \dpd[2]{Q}{z} = \omega^2\dpd{x}{x} + \omega^2\dpd{y}{y} + 0 = 2 \omega^2.
\end{equation*}
Из этого можно сделать вывод, что центробежный потенциал $Q$ не является гармонической функцией.

Лапалсиан потенциала $W$ силы тяжести во внешнем пространстве
\begin{equation*}
	\Delta W_e = \Delta V_e + \Delta Q = 0 + 2 \omega^2 = 2 \omega^2.
\end{equation*}
Применяя уравнение Пуассона, найдем лапласиан потенциала $W$ силы тяжести  во внутреннем пространстве
\begin{equation*}
	\Delta W_i = \Delta V_i + \Delta Q = -4 \pi G \delta + 2 \omega^2,
\end{equation*}
его ещё называют \textit{обобщённым уравнением Пуассона}.

В силу того, что полученные лапласианы потенциала $W$ имеют ненулевое значение, можно сделать вывод, что функция потенциала силы тяжести \textit{не является гармонической} во всём пространстве.

Рассмотрим изменение потенциала силы тяжести $\dif W \left(x, y, z\right)$ при перемещении из точки $P\left( x, y,z \right)$ на бесконечно малый вектор $\vv{\dif l}$ в точку $P_1\left( x + \dif x, y + \dif y, z + \dif z \right)$.
Изменение потенциала силы тяжести $W$ можно представить в виде
\begin{equation*}
    \dif W = \pd{W}{x}\dif x + \pd{W}{y}\dif y + \pd{W}{z}\dif z, 
\end{equation*}
иначе --- как скалярное произведение векторов, один из которых является градиентом потенциала $W$ силы тяжести, а другой --- вектором смещения $\vv{\dif l} = \left\lbrace \dif x, \dif y, \dif z \right\rbrace$
\begin{equation}
	\dif W = \text{grad } W \cdot \vv{\dif l}.
\end{equation}
Градиент потенциала $W$ силы тяжести
\begin{equation}
	\vv{g} = \text{grad } W = \left\lbrace \dpd{W}{x}, \dpd{W}{y}, \dpd{W}{z} \right\rbrace = \left\lbrace g_x, g_y, g_z \right\rbrace
\end{equation}
является \textbf{вектором силы тяжести} $\vv{g}$, который представляет собой полную силу (сила притяжения плюс центробежная сила), действующую на единичную массу.

Тогда изменение потенциала $W$ силы тяжести будет равно
\begin{equation*}
    \dif W = \text{grad } W \cdot \vv{\dif l} = \vv{g} \cdot \vv{\dif l} = |\vv{g}| |\vv{\dif l}| \cos \left(\vv{g}, \vv{\dif l}\right).
\end{equation*}
Производную потенциала по направлению получим как скалярное произведение градиента потенциала $W$ силы тяжести на единичный вектор направления перемещения (орт) $\vv{l} = \dfrac{\vv{\dif l}}{|\vv{\dif l}|} = \dfrac{\vv{\dif l}}{\dif l}$
\begin{equation*}
    \dfrac{\dif W}{\dif l} = \vv{g} \cdot \vv{l} = | \vv{g}| \cos \left(\vv{g}, \vv{l}\right).
\end{equation*}
Как видно,\textit{ изменение потенциала зависит от направления перемещения} $\vv{l}$. Из этого следует несколько важных и полезных свойств.
\begin{enumerate}
    \item Если направление перемещения $\vv{l}$ совпадает с направлением действия силы тяжести $\vv{g}$ (по отвесной линии): $\cos\left( \vv{g}, \vv{l} \right) = 1$, то изменение потенциала $W$ силы тяжести будет максимальным.
    \item Если направление перемещения $\vv{l}$ перпендикулярно направлению действия силы тяжести (перпендикулярно отвесной линии) $\vv{g}$: $\cos\left( \vv{g}, \vv{l} \right) = 1$, то величина потенциала останется неизменной $\dif W = 0$. Таким образом, можно получить поверхность, уравнение которой
        \begin{equation*}
            W\left( x, y, z \right) = c.
        \end{equation*}
	Эта поверхность называется уровенной \textit{эквипотенциальной} поверхностью потенциала силы тяжести, в каждой точке которой потенциал силы тяжести $W$ принимает одно и тоже значение $c$, a вектор силы тяжести (отвесная линия) $\vv{g}$ направлен по нормали к ней. При перемещении по уровенной поверхности работа силой тяжести не совершается.
	
	Заметим, что на силу тяжести не накладываются никакие ограничения и, вообще говоря, на уровенной поверхности она может меняться как по величине, так и по направлению. Следовательно, \textit{уровенная поверхность потенциала силы тяжести имеет сложную форму}.
	Наглядный пример уровенной поверхности в природе --- поверхность воды в замкнутом водоёме при отсутствии течений, т.е. уровенная поверхность --- поверхность равновесия.
    \item Если направление перемещения $\vv{l}$ совпадает с нправлением, противоположным действию силы тяжести $\vv{g}$ (по внешней нормали $\vv{n}$ к отвесной линии): $\cos\left( \vv{g}, \vv{l} \right) = -1$, то $\dif W = - |\vv{g}| \dif l$. Бесконечно малый отрезок $\dif l$ в направлении нормали $\vv{n}$, можно отождествить с элементарным превышением $\dif h$ между бесконечно близкими уровенными поверхностями. Следовательно
        \begin{equation*}
            \dif W = - |\vv{g}| \dif h, \qquad \dif h = -\dfrac{\dif W}{|\vv{g}|}.     
        \end{equation*}
        Из этого выражения следует, что:
        \begin{enumerate}
        	\item cила тяжести всегда имеет ненулевое значение, поэтому если изменение потенциала силы тяжести $\dif W \neq 0$, то превышение между уровенными поверхностями $\dif h \neq 0$;
        	\item через каждую точку может пройти лишь одна уровенная поверхность, соотвественно, уровенные поверхности не пересекаются и не касаются;
        	\item расстояние $\dif h$ между уровенными поверхностями обратно пропорционально величине силы тяжести и поэтому уровенные поверхности непараллельны. Так как сила тяжести на полюсах больше чем на экваторе, то уровенные поверхности на полюсах будут проходить ближе друг к другу, чем на экваторе.
        \end{enumerate}

	\item Представим себе семейство уровенных поверхностей и отрезки $\dif h$, перпендикулярные к соседним поверхностям. Эти отрезки образуют ломаную линию. Если отрезки $\dif h$ устремить к нулю, то, в пределе, ломаная линия превратится в кривую, пересекающую все уровенные поверхности по нормали. Эта кривая называется \textit{силовой линией}. Найдём длину отрезка $AB$ силовой линии между двумя уровенным
        поверхностями
        \begin{equation*}
            h = \int\limits_{A}^{B}\dif h =  -\int\limits_{A}^{B} \dfrac{\dif W}{|\vv{g}|} = \dfrac{1}{g_m}\int\limits_{B}^{A} \dif W = \dfrac{W_A - W_B}{g_m},
        \end{equation*}
        где $g_m$ --- среднее интегральное значение силы тяжести на отрезке кривой $AB$. Таким образом, для определения превышения в гравитационном поле Земли, необходимо знать разность потенциалов  и среднее значение силы между уровенными поверхностями. Эта формула имеет важное значение в теории высот.
\end{enumerate}

%\printbibliography
\end{document}
