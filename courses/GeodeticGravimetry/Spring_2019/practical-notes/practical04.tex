\documentclass[11pt, a4paper]{article}

% Languages and fonts
\usepackage{cmap} 
\usepackage[T2A]{fontenc}
\usepackage[utf8]{inputenc} 
\usepackage[english, russian]{babel}
\usepackage{microtype}
\usepackage{indentfirst}
\frenchspacing

% Mathematics
\usepackage{amsmath, amssymb, amsfonts, amsthm, mathtools, fixmath}
\mathtoolsset{showonlyrefs=true}
\usepackage{esint, esvect} % integrals and vectors
\usepackage{systeme} % equation system
\usepackage{commath} % partials and differentials
\usepackage{icomma} % smart comma ($0,2$ is a number)

% Floats
\usepackage{float}

% Tables
\usepackage{array,tabularx,tabulary,booktabs} % better tables
\usepackage{longtable}
\usepackage{multirow}

% Graphics
\usepackage[pdftex]{graphicx}
\usepackage{wrapfig}

% Theorems
\renewcommand{\proofname}{Доказательство}

\theoremstyle{plain}
\newtheorem{theorem}{Теорема}[section]

\theoremstyle{definition}
\newtheorem{definition}{Определение}
\newtheorem{corollary}{Следствие}[theorem]
\newtheorem{problem}{Задача}[section]

\theoremstyle{remark}
\newtheorem{remark}{Замечание}
\newtheorem*{solution}{Решение}

\usepackage[top=20mm,bottom=20mm,left=20mm,right=20mm]{geometry}

\usepackage{lastpage} % how many pages

\usepackage{soul}

\usepackage{framed} % easy frames
\usepackage{enumerate} % better numbered lists

\usepackage{hyperref}
\usepackage{xcolor}

\usepackage{tikz} % drawing

\usepackage{csquotes}
\usepackage[style=numeric,backend=biber,sorting=none]{biblatex}
\addbibresource{../../../../bibliography.bib}

\renewcommand{\epsilon}{\ensuremath{\varepsilon}}
\renewcommand{\phi}{\ensuremath{\varphi}}
%\renewcommand{\theta}{\vartheta}
\renewcommand{\kappa}{\ensuremath{\varkappa}}
\renewcommand{\le}{\ensuremath{\leqslant}}
\renewcommand{\leq}{\ensuremath{\leqslant}}
\renewcommand{\ge}{\ensuremath{\geqslant}}
\renewcommand{\geq}{\ensuremath{\geqslant}}

\usepackage[useregional]{datetime2}

% custom maketitle
\usepackage{titling}
\setlength{\droptitle}{-4em}
\posttitle{\end{center}\vspace{-3em}}

\title{{\Large Геодезическая гравиметрия 2019}\\ 
    {\bf\Large Практическое занятие № 4} \\
{\Large Притяжение тел простой формы II}}
\author{}
\DTMsavedate{lessondate}{2019-03-04}
\date{\DTMusedate{lessondate}}

\begin{document}
\maketitle

\section{Притяжение шара}
Вспомним формулу потенциала притяжения сферического слоя с \textit{поверхностной} плотностью $\mu=const$ и радиусом $R'$ внешней точки $P$
\begin{equation*}
    V_e = 4\pi G\mu\dfrac{{R'}^2}{r}.
\end{equation*}

Однородный шар с \textit{объёмной} плотностью $\delta=const$ можно представить состоящим из бесконечного числа
сферических слоёв. Пусть $R$ --- радиус шара,
$R'$ --- радиус сферического слоя толщиной $\dif R'$. Перейдем от поверхностной плотности $\mu$ к объёмной $\delta$
 \begin{equation*}
    \mu = \delta \dif R'.
\end{equation*}
Тогда элементарный потенциал притяжения сферического слоя с \textit{объёмной} плотностью $\delta$ внешней точки $P$ равен
\begin{equation*}
    \dif V_e = 4\pi G\delta\dfrac{{R'}^2}{r}\dif R'.
\end{equation*}
Интегрируя по всему радиусу шара, получаем
\begin{equation*}
    V_e = 4\pi G\delta \int\limits_{0}^{R}\dfrac{{R'}^2}{r}\dif R' = 
    \dfrac{4}{3}\pi G\delta\dfrac{R^3}{r}.
\end{equation*}
Вводя массу шара $M = 4/3\ \pi R^3 \delta$, снова получаем
\begin{equation*}
    V_e = \dfrac{GM}{r}.
\end{equation*}
Аналогично для силы
\begin{equation*}
    |\vv{F_e}| = F_r = -\pd{V_e}{r} = \dfrac{4}{3}\pi G\delta \dfrac{R^3}{r^2} = 
    \dfrac{GM}{r^2}.
\end{equation*}

Рассмотрим случай, когда притягиваемая точка $P$ находится внутри шара радиуса $R$ на расстоянии $r$ от его центра. Применяя свойство суперпозиции, потенциал притяжения шара $V_i$ можно представить как сумму \textit{потенциала шара} $V_1$ радиуса $r$, по отношению к которому точка $P$ будет являться внешней и \textit{потенциала шарового слоя} $V_2$, по отношению к которому точка $P$ будет являться внутренней
\begin{equation*}
    V_i = V_1 + V_2.
\end{equation*}

Для нахождения потенциала $V_1$, применим формулу \textit{потенциала шара} $V_e$ для внешней точки $P$, принимая во внимание, что $R=r$, получим
\begin{equation*}
    V_1 = \dfrac{4}{3}\pi G\delta r^2.
\end{equation*}

Для нахождения потенциала $V_2$, нужно получить формулу потенциала шарового слоя $V_i$ для точки $P$, находящейся в пределах его внутреннего радиуса $r$. Для этого, шаровой слой, ограниченный радиусами $R'_1$ и $R'_2$, можно представить состоящим из бесконечного числа сферических слоёв с объёмной плотностью $\delta$, переменным радиусом $R'$ и толщиной $\dif R'$.
Вспомним формулу потенциала притяжения сферического слоя с поверхностной плотностью $\mu=const$ и радиусом $R'$ внутренней точки $P$
\begin{equation*}
    V_i = 4\pi G\mu R'.
\end{equation*}
Выполняя переход от поверхностной плотности к объёмной $\mu = \delta \dif R'$, получим элементарный потенциал притяжения сферического слоя с объёмной плотностью $\delta$ внутренней точки $P$
\begin{equation*}
    \dif V_i = 4\pi G\delta R'\dif R'.
\end{equation*}
Интегрируя по всему шаровому слою, получим формулу \textit{потенциала шарового слоя} для точки $P$, находящейся в пределах его внутреннего радиуса $R_1'$, в общем виде
\begin{equation*}
    V_i = 4\pi G\delta \int\limits_{R_1'}^{R_2'} R'\dif R' = 2\pi G\delta \left( R_2'^2 - R_1'^2 \right).
\end{equation*}
В случае притяжения шара радиусом $R$ для внутренней точки $P$, находящейся на расстоянии $r$ от его центра, шаровой слой будет иметь пределы $[r; R]$. Тогда потенциал шарового слоя $V_2$ будет равен:
\begin{equation*}
    V_2 = 2\pi G\delta \left( R^2 - r^2 \right).
\end{equation*}

Таким образом, потенциал притяжения шара $V_i$ для внутренней точки $P$ равен
\begin{equation*}
    V_i = V_1 + V_2 = \dfrac{2}{3}\pi G\delta \left( 3R^2 - r^2 \right).
\end{equation*}
Для силы получаем
\begin{equation*}
    F_i = -\pd{V_i}{r} = \dfrac{4}{3}\pi G\delta r = \dfrac{GM}{r^2}.
\end{equation*}

\section{Притяжение шарового слоя}
Пусть точку притягивает однородный шаровой слой, заключенный между радиусами $R_1$ и  $R_2$. Тогда
потенциал притяжения внешней точки можно представить как разность потенциалов $V_2$ и $V_1$ шаров радиусом $R_2$ и $R_1$, соответственно
\begin{equation*}
    V_e = V_{2} - V_{1}.
\end{equation*}
Тогда потенциал притяжения \textit{шарового слоя} $V_e$ для внешней точки $P$ равен
\begin{equation*}
    V_e = \dfrac{4}{3}\pi G\delta \dfrac{1}{r} \left( R_2^3 - R_1^3 \right).
\end{equation*}
Для силы получаем
\begin{equation*}
    F_e = -\pd{V_i}{r} = \dfrac{4}{3} \pi G\delta\left( R_2^3 - R_1^3 \right)\dfrac{1}{r^2}.
\end{equation*}
\begin{remark}
    Потенциал шарового слоя на внешнюю точку мог бы быть вызван и притяжением точки с массой
    $M = \dfrac{4}{3}\pi \delta\left( R_2^3 - R_1^3 \right)$ или шаром радиуса $R_2$ с плотностью 
    $\delta' = \delta\left( R_2^3 - R_1^3 \right) / R_2^3$. Такая неоднозначность свидетельствует о
    том, что только по гравиметрическим данным невозможно изучать внутреннее строение Земли.
    Без сейсмических данных мы бы никогда не могли сказать, полая внутри Земля или нет.
\end{remark}

Для точки, находящейся в пределах внутреннего радиуса $r<R_1$ шарового слоя, ранее была получена формула потенциала
\begin{equation*}
    V_i = 2\pi G\delta \left( R_2^2 - R_1^2 \right),
\end{equation*}
тогда для силы справедливо
\begin{equation*}
    F_i = 0.
\end{equation*}

Если точка находится внутри шарового слоя $R_1 < r < R_2$, то в таком случае, применяя свойство суперпозиции, потенциал притяжения можно представить как сумму потенциалов  шарового слоя $V_1$, ограниченного радиусами $R_1$ и $r$, для которого точка будет внешней и шарового слоя $V_2$, ограниченного радиусами $r$ и $R_2$, для которого точка будет находиться в пределах внутреннего радиуса $r$
\begin{equation*}
    V_i = V_1 + V_2.
\end{equation*}
Для нахождения потенциала $V_1$ "внутреннего" шарового слоя, ограниченного радиусами $R_1$ и $r$, воспользуемся формулой потенциала притяжения шарового слоя $V_e$ для внешней точки
\begin{equation*}
    V_1 = \dfrac{4}{3}\pi G\delta \dfrac{1}{r} \left( r^3 - R_1^3 \right).
\end{equation*}
Для нахождения потенциала $V_2$ "внешнего" шарового слоя, ограниченного радиусами $r$ и $R_2$, воспользуемся формулой потенциала притяжения шарового слоя $V_i$ для точки, находящейся в пределах внутреннего радиуса шарового слоя
\begin{equation*}
    V_2 = 2\pi G\delta \left( R_2^2 - r^2 \right).
\end{equation*}
В итоге, потенциал притяжения шарового слоя, для случая, когда точка находится внутри его пределов  $R_1 < r < R_2$, получим:
\begin{equation*}
    V_i = V_1 + V_2 = 2\pi G\delta \left(R_2^2 - \dfrac{1}{3}r^2 - \dfrac{2}{3}\dfrac{R_1^3}{r}\right),
\end{equation*}
а для силы
\begin{equation*}
    F_i = \dfrac{4}{3}\pi G\delta \left(r - \dfrac{R_1^3}{r^2} \right).
\end{equation*}

\section{Притяжение диска}
Силу притяжения бесконечно тонкого однородного диска радиуса $a$ можно представить
через притяжение простого слоя $V = G\iint\limits_{\sigma}\frac{\mu\dif\sigma}{r}$.
Удобно воспользоваться цилиндрической системой координат ($\rho$, $\alpha$, $z$), где $\rho$ ---
полярный радиус, $\alpha$ --- полярный угол, $z$ --- аппликата точки. Таким образом, цилиндрическая
система координат является расширением полярной (плоской) системы координат.

Найдем силу притяжения диска для точки, лежащей на его оси симметрии. Можно записать
\begin{equation*}
    F = \dpd{V}{z} = - G\mu\iint\limits_{\sigma}\dfrac{\dif\sigma}{r^2}\cos\left( r, z \right),
\end{equation*}
\begin{equation*}
    \dif\sigma = \rho\dif\alpha\dif\rho,
\end{equation*}
откуда
\begin{equation*}
    F = \dpd{V}{z} = - G\mu\iint\limits_{\sigma}\dfrac{\rho\dif\rho\dif\alpha}{r^2}\cos\left( r, z \right).
\end{equation*}
Заметим, что
\begin{equation*}
    r^2 = \rho^2 + z^2,\qquad r\dif r = \rho\dif \rho,\qquad \cos\left( r, z \right) = \dfrac{z}{r},
\end{equation*}
тогда
\begin{equation*}
    F_e = - G\mu z \int\limits_{z}^{\sqrt{z^2 + a^2}}\int\limits_{0}^{2\pi}\dfrac{r\dif
    r\dif\alpha}{r^3} = -2\pi G\mu z\int\limits_{z}^{\sqrt{z^2 + a^2}}\dfrac{\dif r}{r^2} =
    -2\pi G\mu\left[ 1 - \dfrac{z}{\sqrt{z^2 + a^2}}\right].
\end{equation*}
Если $z < 0$, то $F_e = +2\pi G\mu\left[ 1 - \dfrac{z}{\sqrt{z^2 + a^2}}\right]$. \\
Теперь найдём значение силы на самом слое
\begin{equation*}
    \lim\limits_{z\to 0} F_e = \pm2\pi G\mu\lim\limits_{z\to 0} \left[ 1 - \dfrac{z}{\sqrt{z^2 +
    a^2}}\right] = \pm2\pi G\mu = const.
\end{equation*}
Прямое значение на слое, равное среднему из двух пределов, $F_0 = 0$, что можно было бы получить и
чисто опираясь на физический смысл силы. Таким образом, сила притяжения
терпит разрыв на величину $4\pi G\mu$ при переходе через слой, т.е. является не неразрывной.

\section{Притяжение плоскости}
Силу притяжения плоскости получим из силы притяжения диска, радиус которого стремится к
бесконечности
\begin{equation*}
    F = \pm2\pi G\mu\lim\limits_{a\to\infty} \left[ 1 - \dfrac{z}{\sqrt{z^2 + a^2}}\right] =
    \pm2\pi G\mu = const,
\end{equation*}
откуда следует, что сила притяжения плоскости не зависит от расстояния до неё.

\section{Притяжение цилиндра}
Будем рассматривать притяжение точки, находящейся на оси однородного цилиндра высотой $H$ и радиуса $a$. Для простоты и приложений будем считать, что точка находится на верхней плоскости его основания ($z = H$).

Однородный цилиндр с объёмной плотностью $\delta=const$ можно представить состоящим из бесконечного числа дисков толщиной $\dif z$. Перейдем от поверхностной плотности к объёмной $\mu = \delta \dif z$. Элементарная сила диска будет равна
\begin{equation*}
    \dif F_z = -2\pi G\delta\left[ 1 - \dfrac{z}{\sqrt{z^2 + a^2}}\right] \dif z,
\end{equation*}
интегрируя по всей поверхности цилиндра, получим
\begin{equation*}
    F_H = -2\pi G\delta\int\limits_{0}^{H}\left[ 1 - \dfrac{z}{\sqrt{z^2 + a^2}}\right] \dif z =
    -2\pi G\delta \left[ z - \sqrt{a^2 + z^2} \right] \bigg|_0^{H} =
    -2\pi G\delta\left( H - \sqrt{a^2 + H^2} + a \right).
\end{equation*}
В случае, когда $a >> H$, часть выражения $\sqrt{a^2 + H^2}$ можно разложить в ряд Тейлора. Для этого приведём её к виду $\sqrt{1 + x}$
\begin{equation*}
    \sqrt{a^2 + H^2} = a\sqrt{1 + \dfrac{H^2}{a^2}},
\end{equation*}
используя ряд Маклорена для функции $\sqrt{1 + x}$, получим

\begin{equation*}
    a\sqrt{1 + \dfrac{H^2}{a^2}} \approx
    a\left(1 + \dfrac{1}{2}\dfrac{H^2}{a^2} - \dfrac{1}{8} \dfrac{H^4}{a^4} + \dots\right) \approx 
    \left(a + \dfrac{H^2}{2a} - \dfrac{H^4}{8a^3} + \dots\right),
\end{equation*}
тогда
\begin{equation*}
    F_H = 
    -2\pi G\delta H\left( 1 - \dfrac{H}{2a} + \dfrac{H^3}{8a^3} + \dots \right).
\end{equation*}

\section{Притяжение плоскопараллельного слоя}
Силу притяжения плоскопараллельного (то есть заключенного между двумя плоскостями) слоя толщиной $H$
получим из силы притяжения цилиндра, радиус которого стремится к бесконечности
\begin{equation*}
    F = 
    -2\pi G\delta H\lim\limits_{a\to\infty}\left(  1 - \dfrac{H}{2a} + \dfrac{H^3}{8a^3} + \dots \right) = 
    -2\pi G\delta H = const.
\end{equation*}
Таким образом, сила притяжения плоскопараллельного слоя является величиной
постоянной. 

Редукция (поправка), вводимая в измеренные значения силы тяжести и вычисляемая по формуле $-2\pi
G\delta H$ называется редукцией Буге. Так может быть учтено притяжение топографических масс, снега,
грунтовых вод и т.д. То есть в тех случаях, когда высота притягивающих масс много меньше их радиуса.
Плотность $\delta$, конечно, будет меняться.

Поле, создаваемое плоскостью или плоскопараллельным слоем называется однородным. Сила здесь имеет одно и
то же направление --- перпендикулярное плоскости --- и одинаковую величину в любой точке. Уровенные
поверхности такого поля --- плоскости, параллельные слою. В прикладной геодезии, например, в
рядовых строительных работах, почти всегда предполагается, что работы проводятся именно в однородном поле.
%\printbibliography
\end{document}
