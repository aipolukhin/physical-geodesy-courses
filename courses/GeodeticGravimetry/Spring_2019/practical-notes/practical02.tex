\documentclass[11pt, a4paper]{article}

% Languages and fonts
\usepackage{cmap} 
\usepackage[T2A]{fontenc}
\usepackage[utf8]{inputenc} 
\usepackage[english, russian]{babel}
\usepackage{microtype}
\usepackage{indentfirst}
\frenchspacing

% Mathematics
\usepackage{amsmath, amssymb, amsfonts, amsthm, mathtools, fixmath}
\mathtoolsset{showonlyrefs=true}
\usepackage{esint, esvect} % integrals and vectors
\usepackage{systeme} % equation system
\usepackage{commath} % partials and differentials
\usepackage{icomma} % smart comma ($0,2$ is a number)

% Floats
\usepackage{float}

% Tables
\usepackage{array,tabularx,tabulary,booktabs} % better tables
\usepackage{longtable}
\usepackage{multirow}

% Graphics
\usepackage[pdftex]{graphicx}
\usepackage{wrapfig}

% Theorems
\renewcommand{\proofname}{Доказательство}

\theoremstyle{plain}
\newtheorem{theorem}{Теорема}[section]

\theoremstyle{definition}
\newtheorem{definition}{Определение}
\newtheorem{corollary}{Следствие}[theorem]
\newtheorem{problem}{Задача}[section]

\theoremstyle{remark}
\newtheorem{remark}{Замечание}
\newtheorem*{solution}{Решение}

\usepackage[top=20mm,bottom=20mm,left=20mm,right=20mm]{geometry}

\usepackage{lastpage} % how many pages

\usepackage{soul}

\usepackage{framed} % easy frames
\usepackage{enumerate} % better numbered lists

\usepackage{hyperref}
\usepackage{xcolor}

\usepackage{tikz} % drawing

\usepackage{csquotes}
\usepackage[style=numeric,backend=biber,sorting=none]{biblatex}
\addbibresource{../../../../bibliography.bib}

\renewcommand{\epsilon}{\ensuremath{\varepsilon}}
\renewcommand{\phi}{\ensuremath{\varphi}}
\renewcommand{\theta}{\vartheta}
\renewcommand{\kappa}{\ensuremath{\varkappa}}
\renewcommand{\le}{\ensuremath{\leqslant}}
\renewcommand{\leq}{\ensuremath{\leqslant}}
\renewcommand{\ge}{\ensuremath{\geqslant}}
\renewcommand{\geq}{\ensuremath{\geqslant}}

\usepackage[useregional]{datetime2}

% custom maketitle
\usepackage{titling}
\setlength{\droptitle}{-4em}
\posttitle{\end{center}\vspace{-3em}}

\title{{\Large Геодезическая гравиметрия 2019}\\ 
    {\bf\Large Практическое занятие № 2} \\
{\Large Притяжение. Основные понятия и свойства}}
\author{}
\DTMsavedate{lessondate}{2019-02-18}
\date{\DTMusedate{lessondate}}

\begin{document}
\maketitle

\section{Закон всемирного притяжения}

Закон гравитационного притяжения был сформулирован Ньютоном и опубликован в Математических началах
натуральной философии\cite{Newton1687}(лат. Philosophiae Naturalis Principia Mathematica) в 1687
году.
Звучит он следующим образом. Две частицы с массами $m_1$ и $m_2$ взаимно притягиваются с силой,
пропорциональной произведению их масс и обратно пропорционально расстоянию $r$ между ними, то есть
\begin{equation*}
    F_{12} = G\dfrac{m_1m_2}{r_{12}^2},
\end{equation*}
где $G$ (также обозначается как $f, \gamma$)--- гравитационная постоянная, 

\begin{problem}
    Чему равна скорость распространения притяжения?
\end{problem}

В системе СИ рекомендованное Комитетом данных для науки и техники (CODATA) значение в 2014
году\cite{CODATA2014} такое
\begin{equation*}
    G = (6,67408 \pm 0,00031)\times10^{-11}\,\text{м}^3\text{кг}^{-1}\text{с}^{-2},
\end{equation*}
где $\pm0,00031\times10^{-11}\text{м}^3\text{кг}^{-1}\text{с}^2$ -- стандартная неопределенность
(погрешность), а относительная стандартная неопределенность гравитационной постоянной
равна $4,7\times10^{-5}$. Точность измерений гравитационной постоянной на несколько порядков ниже
точности измерений других физических величин,  часто новые определения $G$ не совпадают со старыми
на величины, превышающие стандартную неопределённость. Впервые определена Генри Кавендишем.

Пусть теперь точка $m_1 = m$ притягивает единичную массу $m_2 = 1\,\text{кг}$, тогда
\begin{equation}
    F = G\dfrac{m}{r^2},
    \label{eq:dynamic}
\end{equation}
--- сила притяжения, \textbf{численно} (обратите внимание на единицы величин) равная ускорению.
Действительно, из второго закона Ньютона $F_{12} = m_2 a$, тогда
\begin{equation*}
    m_2 a = G\dfrac{m_1m_2}{r_{12}^2},
\end{equation*}
откуда 
\begin{equation}
    a = G\dfrac{m_1}{r_{12}^2} = G\dfrac{m}{r^2}.
    \label{eq:kinematic}
\end{equation}
Это закон Галилея, который гласит, что все тела падают на Землю под действием 
 её притяжения с одинаковым ускорением (\textbf{независимо} от массы падающего тела!):
\begin{itemize}
    \item \url{https://www.youtube.com/watch?v=E43-CfukEgs} --- эксперимент в вакуумной камере
    \item \url{https://www.youtube.com/watch?v=KDp1tiUsZw8} --- эксперимент на Луне
\end{itemize}
Массы, входящие в закон всемирного притяжения и второй закон Ньютона --- разные. В первом случае
масса выступает как количественная характеристика  способности тела к гравитационным
взаимодействиям и называется <<гравитационной>>. Во втором случае масса служит мерой инертности
тела, то есть мерой способности тела приобретать ускорение под действием силы. Такая масса
называется <<инертной>>. Из закона Галилея 
следует, что действующая  на  любое  тело  сила  тяготения  пропорциональна  его  инертной  массе.  Это
утверждение известно как пропорциональность (или эквивалентность) инертной и гравитационной масс
всех тел. Этот принцип проверялся множество раз в ходе экспериментов. Последний раз --- в 2017 году
с помощью спутниковой миссии MICROSCOPE, принцип эквивалентности подтвержден с точностью $10^{-14}$.
В силу этого, в дальнейшем мы не будем различать эти две величины, гравитационную и инертную массы.
Мы также не будет проводить различий между ускорением и силой, действующей на единичную массу.

Единица ускорения в системе СИ --- $\text{м/c}^2$. В гравиметрии, однако, мы будем
использовать внесистемную единицу --- Гал (названную в честь Галилео Галилея), который определяется
следующим образом:
\begin{align*}
    &1\,\text{Гал} = 10^{-2}\,\text{м/c}^2 = 1\,\text{см/c}^2\\
    &1\,\text{мГал} = 10^{-5}\,\text{м/c}^2 \\
    &1\,\text{мкГал} = 10^{-8}\,\text{м/c}^2. 
\end{align*}

\begin{problem}
    Оцените, с какой силой притягиваются друг к другу два человека? 
\end{problem}

Произведение гравитационной постоянной $G$ на массу притягивающего объекта $M$ называется
гравитационным параметром $\mu = GM$. Для Земли существует свой термин --- геоцентрическая гравитационная
постоянная. Согласно рекомендациям Международной службы вращения Земли\cite{iers2010}, она равна
\begin{equation*}
    GM = 3,986004418\times10^{14}\,\text{м}^3\text{с}^{-2}
\end{equation*}
со стандартной неопределенностью $8\times10^{5}\,\text{м}^3\text{с}^{-2}$. Это означает, что
геоцентрическая гравитационная постоянная определена точнее, чем гравитационная постоянная. 
Связано это с тем, что произведение $GM$ может быть определено непосредственно по измерениям, 
в сущности, из обратной задачи --- по заданной силе и расстоянию находят $GM$.

\begin{problem}
    Найдите массу Земли по известной гравитационной постоянной $G$ и геоцентрической гравитационной
    постоянной $GM$. Оцените точность.
\end{problem}
\begin{problem}
    Средний радиус Земли $R = 6371\,\text{км}$. Найдите притяжение, создаваемое Землёй
    на её сферической поверхности, предполагая, что вся её масса сосредоточена в центре.
\end{problem}

\section{Сила притяжения в векторной форме}
\begin{equation*}
    \vv{F} = -Gm\dfrac{\vv{r}}{r^3}
\end{equation*}
Радиус -- вектор $\vv{r}$ направлен от притягивающей токи $P_1 (x_1, y_1, z_1)$ к притягиваемой $P_2
(x_2, y_2, z_2)$, его длина равна
\begin{equation*}
    \left| \vv{r}\right| = r = \sqrt{\left( x_2 - x_1 \right)^2 +
    \left( y_2 - y_1 \right)^2 + \left( z_2 - z_1 \right)^2}
\end{equation*}
Производные $r$ по координатам притягиваемой точки $P_2$ (далее индексы опускаем)
\begin{equation*}
    \dpd{r}{x_2} = \dpd{r}{x} = \dfrac{x_2 - x_1}{r},\qquad
    \dpd{r}{y_2} = \dpd{r}{y} =  \dfrac{y_2 - y_1}{r},\qquad
    \dpd{r}{z_2} = \dpd{r}{z} = \dfrac{z_2 - z_1}{r}. 
\end{equation*}
Проекции $r$ на координатные оси равны
\begin{equation*}
    x_2 - x_1,\qquad y_2 - y_1, \qquad z_2 - z_1,
\end{equation*}
так что направляющие косинусы равны
\begin{equation*}
    \cos{(\vv{r}, x)} = \dfrac{x_2 - x_1}{r},\qquad \cos{(\vv{r}, y)} = \dfrac{y_2 - y_1}{r},\qquad
    \cos{(\vv{r}, z)} = \dfrac{z_2 - z_1}{r}.
\end{equation*}
Проекции силы на координатные оси будут выражаться через направляющие косинусы
\begin{align*}
    & F_x = |\vv{F}|\cos{(\vv{F}, x)}, \\
    & F_y = |\vv{F}|\cos{(\vv{F}, y)}, \\
    & F_z = |\vv{F}|\cos{(\vv{F}, z)}. 
\end{align*}
тогда, учитывая, что сила $\vv{F}$ направлена от $P_2$ к $P_1$, то есть противоположно направлению
$\vv{r}$, запишем
\begin{align*}
    & \cos{(\vv{F}, x)} = -\cos{(\vv{r}, x)} = -\dpd{r}{x} = -\dfrac{x_2 - x_1}{r}, \\
    & \cos{(\vv{F}, y)} = -\cos{(\vv{r}, y)} = -\dpd{r}{y} = -\dfrac{y_2 - y_1}{r}, \\
    & \cos{(\vv{F}, z)} = -\cos{(\vv{r}, z)} = -\dpd{r}{z} = -\dfrac{z_2 - z_1}{r}, 
\end{align*}
окончательно получаем
\begin{align*}
    & F_x = -Gm \dfrac{x_2 - x_1}{r^3}, \\
    & F_y = -Gm \dfrac{y_2 - y_1}{r^3}, \\
    & F_z = -Gm \dfrac{z_2 - z_1}{r^3}. \\
\end{align*}

\section{Потенциал материальной точки и его свойства}
\begin{equation*}
    V = \dfrac{GM}{r}
\end{equation*}
\begin{align*}
    & \dpd{V}{x} = -Gm \dfrac{x_2 - x_1}{r^3} = F_x, \\
    & \dpd{V}{y} = -Gm \dfrac{y_2 - y_1}{r^3} = F_y,\\
    & \dpd{V}{z} = -Gm \dfrac{z_2 - z_1}{r^3} = F_z.
\end{align*}
\begin{definition}
    Потенциал --- скалярная функция, частные производные которой по осям координат равны проекциям
    действующей силы на соотвествующие оси.
\end{definition}
\begin{definition}
    Потенциал --- скалярная функция $V$, градиент которой равен силе $\vv{F}$. 
\end{definition}

\begin{problem}
	Что такое градиент?
\end{problem}

\paragraph{Градиент.} 
\begin{align*}
    &\vv{F} = \nabla V = \textrm{grad} V = \begin{pmatrix}
    F_x \\ F_y \\ F_z \end{pmatrix} = \begin{pmatrix}
        \pd{V}{x}\\\pd{V}{y}\\\pd{V}{z} 
    \end{pmatrix}\\
    &\left| \vv{F} \right| = \sqrt{\left( \dpd{V}{x} \right)^2 +\left( \dpd{V}{y} \right)^2 + \left( \dpd{V}{z} \right)^2} =
    \sqrt{F_x^2 + F_y^2 + F_z^2}
\end{align*}

Таким образом, зная потенциал $V$ и вычислив его частные производные по каждой из координат, можно
определить силу притяжения $F$. 

\paragraph{Физический смысл.}
Механическая работа --- количественная мера действия силы $\vv{F}$, равная её произведению 
на пройденный путь $\vv{s}$
\begin{equation*}
    A = \vv{F}\cdot\vv{s} = Fs\cos{\left( \vv{F}, \vv{s} \right)}.
\end{equation*}
Работа совершается, только когда на тело действует сила и оно движется.

Пусть единичная масса переместилась на бесконечно малую величину $\dif r$ в направлении действия силы
притяжения, то есть совершилась работа
\begin{equation*}
    \dif A = F dr \cos{\left( \vv{F}, \vv{r} \right)} =
    F dr \cos{0^\circ} = -G\dfrac{m}{r^2}dr,
\end{equation*}
где учтено, что направление движения и направление силы совпадают, т.е. $\angle (\vv{F},\vv{s}) =
0^\circ$. А теперь переместим единичную массу из бесконечности в притягиваемую точку, тогда
\begin{equation*}
    A = \int\limits_{\infty}^{r}\dif A = \int\limits_{\infty}^{r} -G\dfrac{m}{r^2}dr =
    G\dfrac{m}{r} = V.
\end{equation*}

Пусть теперь единичная масса перемещается под действием силы притяжения  из точки $P_2$ в точку $P'_2$,
находящуюся на конечном расстоянии от $P_2$, тогда
\begin{equation*}
    A = \int\limits_{P_2}^{P'_2}\dif A =  \int\limits_{P_2}^{P'_2}\dif V = Gm\left( \dfrac{1}{r'} - \dfrac{1}{r}\right) = 
    V' - V,
\end{equation*}
где $r'$ --- расстояние от $P_1$ до $P'_2$, $V'$ --- потенциал притяжения материальной точки $P_1$ в
$P'_2$.
Таким образом, работа, совершаемая силой притяжения по перемещению материальной точки, равна
разности потенциалов в начальном и конечном положении и \textbf{не зависит от пути}. Работа по
замкнуотуму контуру равна нулю.

\paragraph{Уровенная поверхность.}
Пусть единичная масса пермещается перпендикулярно направлению силы притяжения, тогда
\begin{equation*}
    \cos{\left( \vv{F}, \vv{r} \right)} = \cos{90^\circ} = 0, \qquad \dif V = 0,
\end{equation*}
поэтому 
\begin{equation*}
    V = \int\limits_{P_2}^{P'_2}\dif V = C, 
\end{equation*}
или
\begin{equation*}
    V \left( x, y, z \right) = C.
\end{equation*}
Поверхность, на которой потенциал материальной точки постоянен, называет уровенной или
эквипотенциальной. Работа при перемещении по уровенной поверхности не совершается.
Сила всюду перпендикулярна уровенной поверхности. Можно задать бесконечное число уровенных
поверхностей, меняя константу $C$. Уровенные поверхности не пересекаются, поскольку потенциал ---
однозначная функция координат.

\begin{problem}
    Какую форму имеют уровенные поверхности поля, создаваемого материальной точкой?
\end{problem}
\begin{solution}
    Уровенные поверхности имеют форму концентрических сфер с центром в материальной точке $(x_0, y_0, z_0)$. Действительно,
    \begin{align*}
        &V = \dfrac{Gm}{r} = C,\ \text{откуда} \\
        &\sqrt{\left( x - x_0 \right)^2 +
        \left( y - y_0 \right)^2 + \left( z - z_0 \right)^2} = \dfrac{Gm}{C} \text{---уравнение
        сферы}. 
    \end{align*}
\end{solution}

\section{Принцип суперпозиции. Потенциал объёмного тела}
\begin{equation*}
    V = \sum\limits_{i=1}^{N} V_i = G\dfrac{m_1}{r_1} + G\dfrac{m_2}{r_2} + \dots +
    G\dfrac{m_N}{r_N} = G\sum\limits_{i=1}^{N} \dfrac{m_i}{r_i}
\end{equation*}
\begin{align*}
    N &\to \infty\\
    \sum\limits_{i}&\to\iiint\limits_{\Omega} \\
    m_i&\to\dif m
\end{align*}

\begin{equation*}
    \dif m\left( x, y, z \right) = \rho\left( x, y, z \right)\dif v\left( x, y, z \right) = \rho\left( x, y, z \right)\dif x\dif y\dif z
\end{equation*}

\begin{equation*}
    V = G\iiint\limits_{\Omega}\dfrac{\dif m}{r} = 
    G\iiint\limits_{\Omega}\dfrac{\rho\left( x, y, z \right)}{r}\dif x\dif y\dif z
\end{equation*}

\printbibliography
\end{document}
