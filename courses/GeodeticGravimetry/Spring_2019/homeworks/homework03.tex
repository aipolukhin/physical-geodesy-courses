\documentclass[11pt, a4paper,addpoints]{exam}

% Languages and fonts
\usepackage{cmap} 
\usepackage[T2A]{fontenc}
\usepackage[utf8]{inputenc} 
\usepackage[english, russian]{babel}
\usepackage{microtype}
\usepackage{indentfirst}
\frenchspacing

% Mathematics
\usepackage{amsmath, amssymb, amsfonts, amsthm, mathtools, fixmath}
\mathtoolsset{showonlyrefs=true}
\usepackage{esint, esvect} % integrals and vectors
\usepackage{systeme} % equation system
\usepackage{commath} % partials and differentials
\usepackage{icomma} % smart comma ($0,2$ is a number)
\usepackage{mathabx}% astronomy

% Floats
\usepackage{float}

% Tables
\usepackage{array,tabularx,tabulary,booktabs} % better tables
\usepackage{longtable}
\usepackage{multirow}

% Graphics
\usepackage[pdftex]{graphicx}
\usepackage{wrapfig}

% Theorems
\renewcommand{\proofname}{Доказательство}

%\theoremstyle{plain}
\newtheorem{theorem}{Теорема}[section]

%\theoremstyle{definition}
\newtheorem{definition}{Определение}
\newtheorem{corollary}{Следствие}[theorem]

\theoremstyle{remark}
\newtheorem{remark}{Замечание}

\usepackage[top=20mm,bottom=20mm,left=20mm,right=20mm]{geometry}

\usepackage{soul}
\usepackage{enumerate} % better numbered lists
\usepackage{hyperref}
\usepackage{xcolor}
\usepackage{tikz} % drawing

%\usepackage{csquotes}
%\usepackage[style=authoryear,maxcitenames=2,backend=biber,sorting=nty]{biblatex}
%\bibliography{}

\renewcommand{\epsilon}{\ensuremath{\varepsilon}}
\renewcommand{\phi}{\ensuremath{\varphi}}
\renewcommand{\theta}{\vartheta}
\renewcommand{\kappa}{\ensuremath{\varkappa}}
\renewcommand{\le}{\ensuremath{\leqslant}}
\renewcommand{\leq}{\ensuremath{\leqslant}}
\renewcommand{\ge}{\ensuremath{\geqslant}}
\renewcommand{\geq}{\ensuremath{\geqslant}}

\usepackage[useregional]{datetime2}

% exam
\pointsinrightmargin
\marginpointname{ б.}

% custom maketitle
\usepackage{titling}
\setlength{\droptitle}{-4em}
\posttitle{\end{center}\vspace{-4em}}

\title{{\Large Геодезическая гравиметрия 2019}\\ 
    {\bf\Large Домашнее задание № 3}}
\author{}
\DTMsavedate{deadline}{2019-03-31}

\date{\normalsize\bf Крайний срок сдачи: \DTMusedate{deadline}}

\begin{document}
\maketitle
\thispagestyle{empty}
\begin{questions}
    \question[1] Пусть Земля --- однородный шар радиусом $R = 6371\,\text{км}$. Геоцентрическая
    гравитационная постоянная $GM = 3,986\times10^{14}\,\text{м}^3\text{с}^{-2}$.
    \begin{parts}
        \part Найти значение средней плотности Земли.
        \part Вычислить потенциал и силу притяжения на заданных расстояниях от центра планеты:
        \begin{enumerate}
            \item $r_1 = 3,00\times 10^6\,\text{м}$ (внутренняя точка),
            \item $r_2 = 6,371\times 10^6\,\text{м}$ (точка на поверхности),
            \item $r_3 = 6,384\times 10^6\,\text{м}$ (вершина вулкана Чимборасо),
            \item $r_4 = 6,42\times 10^6\,\text{м}$ (50 км над поверхностью --- верхняя граница стратосферы),
            \item $r_5 = 26,4\times 10^6\,\text{м}$ (20000 км над поверхностью --- высота полета спунтиков GPS).
        \end{enumerate}
    \end{parts}
    \question[1] Построить графики зависимости силы и потенциала притяжения от расстояния до
    притягиваемой точки для притягивающих однородных сферы, шара и шарового слоя. Рассматривать случай, когда расстояние непрерывно меняется от $-4R$ до $4R$, где $R$ --- внешний радиус притягивающего тела.
    \question[1] По сейсмическим данным известно, что Землю приближённо можно представить состоящей 
    из четырёх однородных шаровых слоёв: внутреннее ядро, внешнее ядро, мантия и кора. 
    Написать выражения, необходимые для 
    вычисления потенциала и силы притяжения каждого слоя на единичную массу, находящуюся на поверхности планеты (шара). Вычислить эти величины. Найти полный потенциал и результирующую силу. Проанализировать результат: какая составляющая вносит наибольший вклад в результирующую силу притяжения, а какая --- наименьший? Данные взять из таблицы ($R$ --- внешний радиус слоя):
    \begin{table}[h]
            \centering
            \begin{tabular}{|c|c|c|c|}
                \hline
                  & Слой & $R, \text{км}$ & Средняя плотность, $\text{г/см}^3$ \\\hline
                1 & Внутреннее ядро & 1300  & 13 \\\hline
                2 & Внешнее ядро & 3500  & 11 \\\hline
                3 & Мантия  & 6350  & 4,5 \\\hline
                4 & Кора & 6371  & 2,7 \\\hline
            \end{tabular}
        \end{table}
        \question[2] Определить притяжение атмосферы на высоте $H = 500 \times i$ м ($i$ --- вариант) над Землёй. Считать, что атмосфера состоит из сферических слоёв, плотность меняется по закону $\delta = 1,3333e^{-0,13H}$ кг/м$^3$. 
        
        \question[2] Вычислить притяжение колец Сатурна на оси вращения планеты. Найти значение притяжения на высоте $z = 100 \times i$ км ($i$ --- вариант). Масса колец Сатурна $9,6 \times 10^{20}$ кг, внутренний радиус $\rho_1 = 72 000$ км, внешний $\rho_2 = 139 000$ км, толщиной колец пренебречь. Полярный радиус Сатурна $54 400$ км.
        
        \question[2] Получить формулы для вычисления потенциала и силы притяжения вулкана на его оси симметрии. Принять вулкан за конус с углом при вершине $90^{\circ}$. Сделать вычисления для вершины горы Фудзияма ($H = 3776$ м), плотность $\delta = 3300$ кг/м$^3$.
              
        \question[3] Центр однородного шара радиуса $R$ находится под землёй на глубине $a$ $(R < a)$.
        Плотность шара $\delta$ больше, чем плотность поверхностных слоёв Земли. Землю считать плоской. Начало прямоугольной системы координат выбрать в плоскости над центром шара: ось $x$ направлена на восток, ось $y$ --- на север, ось $z$ --- в зенит.
        \begin{parts}
            \part Определить потенциал $T$, вызванный аномальной массой шара, на земной поверхности.
            \part Определить производные потенциала $\dpd{T}{x}$, $\dpd{T}{y}$ и $\dpd{T}{z}$.
            \part Определить вторые производные потенциала $\dpd[2]{T}{x}$, $\dpd[2]{T}{y}$, $\dpd[2]{T}{z}$, $\dfrac{\partial^2 T}{\partial x \partial y}$, $\dfrac{\partial^2 T}{\partial x \partial z}$, $\dfrac{\partial^2 T}{\partial y \partial z}$.
            \part Построить графики $T, T_z, T_{xz}, T_{zz}$ в плоскости $(xz)$ в диапазоне $-3a \leq x \leq +3a$.
        \end{parts}
        Исходные данные для вычислений: $a = 60 + 2 \times i$ м, $R = 50 + 2 \times i$ м, $\delta = 5 + 0,1 \times i$ г/cм$^3$, $\Delta = 1,9$ г/cм$^3$ (песок).
\end{questions}
%\printbibliography
\end{document}
