\documentclass[11pt, a4paper,addpoints]{exam}

% Languages and fonts
\usepackage{cmap} 
\usepackage[T2A]{fontenc}
\usepackage[utf8]{inputenc} 
\usepackage[english, russian]{babel}
\usepackage{microtype}
\usepackage{indentfirst}
\frenchspacing

% Mathematics
\usepackage{amsmath, amssymb, amsfonts, amsthm, mathtools, fixmath}
\mathtoolsset{showonlyrefs=true}
\usepackage{esint, esvect} % integrals and vectors
\usepackage{systeme} % equation system
\usepackage{commath} % partials and differentials
\usepackage{icomma} % smart comma ($0,2$ is a number)
\usepackage{mathabx}% astronomy

% Floats
\usepackage{float}

% Tables
\usepackage{array,tabularx,tabulary,booktabs} % better tables
\usepackage{longtable}
\usepackage{multirow}

% Graphics
\usepackage[pdftex]{graphicx}
\usepackage{wrapfig}

% Theorems
\renewcommand{\proofname}{Доказательство}

%\theoremstyle{plain}
\newtheorem{theorem}{Теорема}[section]

%\theoremstyle{definition}
\newtheorem{definition}{Определение}
\newtheorem{corollary}{Следствие}[theorem]

\theoremstyle{remark}
\newtheorem{remark}{Замечание}

\usepackage[top=20mm,bottom=20mm,left=20mm,right=20mm]{geometry}

\usepackage{soul}
\usepackage{enumerate} % better numbered lists
\usepackage{hyperref}
\usepackage{xcolor}
\usepackage{tikz} % drawing

%\usepackage{csquotes}
%\usepackage[style=authoryear,maxcitenames=2,backend=biber,sorting=nty]{biblatex}
%\bibliography{}

\renewcommand{\epsilon}{\ensuremath{\varepsilon}}
\renewcommand{\phi}{\ensuremath{\varphi}}
\renewcommand{\theta}{\vartheta}
\renewcommand{\kappa}{\ensuremath{\varkappa}}
\renewcommand{\le}{\ensuremath{\leqslant}}
\renewcommand{\leq}{\ensuremath{\leqslant}}
\renewcommand{\ge}{\ensuremath{\geqslant}}
\renewcommand{\geq}{\ensuremath{\geqslant}}

\usepackage[useregional]{datetime2}

% exam
\pointsinrightmargin
\marginpointname{ б.}

% custom maketitle
\usepackage{titling}
\setlength{\droptitle}{-4em}
\posttitle{\end{center}\vspace{-4em}}

\title{{\Large Геодезическая гравиметрия 2019}\\ 
    {\bf\Large Домашнее задание № 4}}
\author{}
\DTMsavedate{deadline}{2019-04-08}

\date{\normalsize\bf Крайний срок сдачи: \DTMusedate{deadline}}

\begin{document}
\maketitle
\thispagestyle{empty}
\begin{questions}
        \question[1] Найти градиент $\nabla \phi$ и вычислить лапласиан скалярного поля $\Delta \phi$ для следующих полей:
        \begin{parts}
        	\part $\phi = \dfrac{z}{(x^2 + y^2 + z^2)^{3/2}}$,
        	\part $\phi = \dfrac{2z^2 - x^2 - y^2}{(x^2 + y^2 + z^2)^{5/2}}$,
        	\part $\phi = \ln (xyz)$.        
        \end{parts}
        \question[1] В каких точках пространства градиент поля
        \begin{equation*}
        	\phi = x^3 + y^3 + z^3 - 3xyz
        \end{equation*}
        \begin{parts}
        	\part перпендикулярен к оси $z$,
        	\part параллелен оси $z$,
        	\part равен нулю?
        \end{parts}
        \question[1] Найти угол между
        \begin{parts}
        	\part градиентами поля $ \phi = \dfrac{x}{x^2 + y^2 + z^2}$ в точках $A(1,2+i,2)$ и $B(-3,i,0)$,
        	\part градиентами полей $ \phi = x^2 + y^2 - z^2$ и $\psi = \arcsin \dfrac{x}{x+y}$ в точке $O\left(i,i,\sqrt{7}\right)$.
        \end{parts}
        \question[2] Пусть $ \phi = xy - z^2$. Найти величину и направление $\nabla \phi$  в точке $O(-9+i, 12-i, 10+i)$. Чему равна производная $\dfrac{\dif \phi}{\dif l}$ в направлении биссектрисы координатного угла $xy$?
\end{questions}
%\printbibliography
\end{document}
