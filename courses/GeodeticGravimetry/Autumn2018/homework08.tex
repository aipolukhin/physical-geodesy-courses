\documentclass[11pt, a4paper,addpoints]{exam}

% Languages and fonts
\usepackage{cmap} 
\usepackage[T2A]{fontenc}
\usepackage[utf8]{inputenc} 
\usepackage[english, russian]{babel}
\usepackage{microtype}
\usepackage{indentfirst}
\frenchspacing

% Mathematics
\usepackage{amsmath, amssymb, amsfonts, amsthm, mathtools, fixmath}
%\mathtoolsset{showonlyrefs=true}
\usepackage{esint, esvect} % integrals and vectors
\usepackage{systeme} % equation system
\usepackage{commath} % partials and differentials
\usepackage{icomma} % smart comma ($0,2$ is a number)

% Floats
\usepackage{float}

% Tables
\usepackage{array,tabularx,tabulary,booktabs} % better tables
\usepackage{longtable}
\usepackage{multirow}

% Graphics
\usepackage[pdftex]{graphicx}
\usepackage{wrapfig}

% Theorems
\renewcommand{\proofname}{Доказательство}

%\theoremstyle{plain}
\newtheorem{theorem}{Теорема}[section]
\newtheorem*{theorem*}{Теорема}

%\theoremstyle{definition}
\newtheorem{definition}{Определение}
\newtheorem{corollary}{Следствие}[theorem]

\theoremstyle{remark}
\newtheorem{remark}{Замечание}

\usepackage[top=20mm,bottom=20mm,left=20mm,right=20mm]{geometry}

\usepackage{soul}
\usepackage{enumerate} % better numbered lists
\usepackage{hyperref}
\usepackage{xcolor}
\usepackage{tikz} % drawing

\usepackage{csquotes}
\usepackage[style=numeric,maxcitenames=2,backend=biber,sorting=nty]{biblatex}
\bibliography{../../../../bibliography}

\footer{}{\thepage}{}
{}

\renewcommand{\epsilon}{\ensuremath{\varepsilon}}
\renewcommand{\phi}{\ensuremath{\varphi}}
\renewcommand{\theta}{\vartheta}
\renewcommand{\kappa}{\ensuremath{\varkappa}}
\renewcommand{\le}{\ensuremath{\leqslant}}
\renewcommand{\leq}{\ensuremath{\leqslant}}
\renewcommand{\ge}{\ensuremath{\geqslant}}
\renewcommand{\geq}{\ensuremath{\geqslant}}

\usepackage[useregional]{datetime2}

% exam
\pointsinrightmargin
\marginpointname{ б.}

% custom maketitle
\usepackage{titling}
\setlength{\droptitle}{-4em}
\posttitle{\end{center}\vspace{-4em}}

\title{{\Large Геодезическая гравиметрия 2018}\\ 
    {\bf\Large Домашнее задание № 8}}
\author{}
\DTMsavedate{deadline}{2018-12-21}

\date{\normalsize\bf Крайний срок сдачи: \DTMusedate{deadline}}

\begin{document}
\maketitle
\begin{questions}
    \question Построить график функции Стокса 
    \begin{equation*}
        S \left( \psi \right) = \cosec{\dfrac{\psi}{2}} + 1 - 6\sin{\dfrac{\psi}{2}} - 
        \cos{\psi}\left[ 5 + 3\ln\left( \sin{\dfrac{\psi}{2}} + \sin^2{\dfrac{\psi}{2}} \right) \right]
    \end{equation*}
    в интервале $\psi$ от
    $0^\circ$ до $180^\circ$. Опиcать поведение функции (нули, минимум, максимум, экстремумы).

    \question Вычислить местную аномалию высоты по гравиметрической карте в четырёх точках.

    Для выполнения задания в середине гравиметрической карты необходимо выбрать четыре соседние
    трапеции (координаты центров этих трапеций и будут служить координатами точек вычисления).

    Гравиметрическую аномалию высоты в ограниченной области можно получить по формуле
    \begin{equation*}
        \zeta = \dfrac{R}{4\pi\gamma} \sum\limits_{i} \Delta \bar{g}_i S_i\left( \psi \right)
        \Delta\omega_i,
    \end{equation*}
    где $S_i\left( \psi \right)$ --- функция Стокса, расчитанная для центра трапеции,
    $\Delta\omega_i = \cos{B_{\textrm{ср.}}}\Delta L_i \Delta B_i$ --- площадь трапеции, $\gamma$
    --- значение нормальной силы тяжести в определяемой точке, которую достаточно принять равной
    $\gamma_0$ --- нормальной силе тяжести на эллипсоиде.
    Коэффициенты $\dfrac{R}{4\pi\gamma} S_i\left(\psi \right)\Delta\omega_i$ не зависят от
    долготы исследуемой точки, их можно рассчитать заранее для каждой параллели.

    Суммирование ведут по всем трапециям, кроме той, которая содержит саму определяемую точку. В ней
    функция Стокса обращается в бесконечность, поэтому её необходимо проинтегрировать аналитически,
    а область вычисления разделить на две части --- нулевую зону, непосредственно примыкающую к точке, и
    ближнюю зону, вычисление в которой ведется по приведенной выше формуле. 
    Значение в нулевой зоне может быть приближённо представлено (после интегрирования функции
    Стокса) так
    \begin{equation*}
        \zeta_{\textrm{н.з.}} \approx \dfrac{s_0}{\gamma} \Delta g_P \approx \dfrac{\sqrt{\Delta x\Delta
        y}}{\gamma \sqrt{\pi}} \Delta g_P,
    \end{equation*}
    где $s_0$ --- радиус нулевой зоны, $\Delta x, \Delta y$ --- размеры трапеции в километрах,
    $\Delta g_P$ --- аномалия силы тяжести в определяемой точке. 

    Оба результата, в ближней и нулевой зоне, складывают $\zeta = \zeta_{\textrm{н.з.}} +
    \zeta_{\textrm{б.з.}}$ и получают местную гравиметрическую аномалию высоты.
\end{questions}
%\section*{\centering Исходные данные}
%\printbibliography
\end{document}
