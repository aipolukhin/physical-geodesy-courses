\documentclass[11pt, a4paper,addpoints]{exam}

% Languages and fonts
\usepackage{cmap} 
\usepackage[T2A]{fontenc}
\usepackage[utf8]{inputenc} 
\usepackage[english, russian]{babel}
\usepackage{microtype}
\usepackage{indentfirst}
\frenchspacing

% Mathematics
\usepackage{amsmath, amssymb, amsfonts, amsthm, mathtools, fixmath}
\mathtoolsset{showonlyrefs=true}
\usepackage{esint, esvect} % integrals and vectors
\usepackage{systeme} % equation system
\usepackage{commath} % partials and differentials
\usepackage{icomma} % smart comma ($0,2$ is a number)

% Floats
\usepackage{float}

% Tables
\usepackage{array,tabularx,tabulary,booktabs} % better tables
\usepackage{longtable}
\usepackage{multirow}

% Graphics
\usepackage[pdftex]{graphicx}
\usepackage{wrapfig}

% Theorems
\renewcommand{\proofname}{Доказательство}

%\theoremstyle{plain}
\newtheorem{theorem}{Теорема}[section]

%\theoremstyle{definition}
\newtheorem{definition}{Определение}
\newtheorem{corollary}{Следствие}[theorem]

\theoremstyle{remark}
\newtheorem{remark}{Замечание}

\usepackage[top=20mm,bottom=20mm,left=20mm,right=20mm]{geometry}

\usepackage{soul}
\usepackage{enumerate} % better numbered lists
\usepackage{hyperref}
\usepackage{xcolor}
\usepackage{tikz} % drawing

\footer{}{\thepage}{}
{}

%\usepackage{csquotes}
%\usepackage[style=authoryear,maxcitenames=2,backend=biber,sorting=nty]{biblatex}
%\bibliography{}

\renewcommand{\epsilon}{\ensuremath{\varepsilon}}
\renewcommand{\phi}{\ensuremath{\varphi}}
\renewcommand{\theta}{\vartheta}
\renewcommand{\kappa}{\ensuremath{\varkappa}}
\renewcommand{\le}{\ensuremath{\leqslant}}
\renewcommand{\leq}{\ensuremath{\leqslant}}
\renewcommand{\ge}{\ensuremath{\geqslant}}
\renewcommand{\geq}{\ensuremath{\geqslant}}

\usepackage[useregional]{datetime2}

% exam
\pointsinrightmargin
\marginpointname{ б.}

% custom maketitle
\usepackage{titling}
\setlength{\droptitle}{-4em}
\posttitle{\end{center}\vspace{-4em}}

\title{{\Large Геодезическая гравиметрия 2018}\\ 
    {\bf\Large Домашнее задание № 7}}
\author{}
\DTMsavedate{deadline}{2018-12-07}

\date{\normalsize\bf Крайний срок сдачи: \DTMusedate{deadline}}

\begin{document}
\maketitle
\thispagestyle{empty}
\begin{questions}
    \question Вычисление геопотенциальных чисел.

    Разность геопотенциальных чисел между двумя смежными пунктами нивелирной сети можно вычислить как
    \begin{equation*}
        W_i - W_{i + 1} = g_m \Delta h,
    \end{equation*}
    где $g_m$ --- среднее значение силы тяжести на пунктах $i$ и ${i+1}$, $\Delta h$ --- превышение.

    Обработку нивелирного полигона следует начинать с вычисления невязки $\epsilon$ полигона и
    уравнивания разностей $W_i - W_{i+1}$. Поправку $\Delta_i$ в измеренную разность вычисляется по
    формуле
    \begin{equation*}
        \Delta_i = -\dfrac{\epsilon}{\sum\dfrac{1}{p_i}} \dfrac{1}{p_i},
    \end{equation*}
    где $p_i$ -- вес, который можно выбрать обратно пропорциональным длине секции $L$ в километрах, то есть
    $p_i = \dfrac{1}{L}$.

    По уравненным разностям $W_i - W_{i+1}$ вычисляют уравненные геопотенциальные числа $W_0 - W_i$ 
    всех реперов, приняв в качестве исходного уравненное значения для одного из пунктов полигона из
    каталога, то есть из исходной сети. Теперь можно сравнить полученные величины со значениями из
    каталога. Они будут различными, потому что последние были получены в результате уравнивания всей
    сети, однако значения геопотенциальных чисел будут близкими и служат для контроля грубых ошибок
    вычислений.

    \question Вычисление нормальных и динамических высот.

    Нормальная высота $H^\gamma$ определяется по формуле
    \begin{equation*}
        H^\gamma = \dfrac{W_0 - W_i}{\gamma_m^i}
    \end{equation*}
    где $\gamma_m^i$ --- средняя нормальная сила тяжести для точке на высоте $H^\gamma / 2$. Поскольку
    нормальная высота входит в левую и правую части, то её находят последовательными приближениями. 

    В начальном приближении можно принять
    \begin{equation*}
        H^\gamma_{\textrm{прибл.}} = \dfrac{W_i - W_{i+1}}{\gamma_0^i},
    \end{equation*}
    где $\gamma_0^i$ --- значение нормальной силы тяжести на эллипсоиде на широте репера. По приближённой высоте
    вычисляют нормальную силу тяжести
    \begin{equation*}
        \gamma_m^i = \gamma_0^i + \dpd{\gamma}{H} \dfrac{H_i^\gamma}{2} =
        \gamma_0^i - 0,3086 \dfrac{H_i^\gamma}{2}.
    \end{equation*}
    После этого вычисляют нормальную высоту в первом приближении, которого обычно бывает достаточно
    (если нет, то итерации повторяют).

    Динамическую высоту вычисляют по формуле
    \begin{equation*}
        H^d = \dfrac{W_0 - W_i}{\gamma_0^{45}},
    \end{equation*}
    где $\gamma_0^{45}$ --- значение нормальной силы тяжести на эллипсоиде на широте $45^\circ$.
    Если широта репера не равна $45^\circ$ и высота над уровнем моря велика, значение
    $\gamma_0^{45}$ будет сильно отличаться от $\gamma_m$, поэтому динамические высоты в этом случае
    сильно отличаются от нормальных. Если участок работ небольшой, то для вычисления динамических
    высот целесообразно использовать среднее значение силы тяжести $g_{\textrm{ср.}}$ на этом
    участке, тогда динамическая высота будет близка к сумме измеренных превышений $\Delta h$.

    В конце необходимо найти разность вычисленных динамических и нормальных высот $H^d - H^\gamma$.
\end{questions}

%\printbibliography
\end{document}
