\documentclass[11pt, a4paper,addpoints]{exam}

% Languages and fonts
\usepackage{cmap} 
\usepackage[T2A]{fontenc}
\usepackage[utf8]{inputenc} 
\usepackage[english, russian]{babel}
\usepackage{microtype}
\usepackage{indentfirst}
\frenchspacing

% Mathematics
\usepackage{amsmath, amssymb, amsfonts, amsthm, mathtools, fixmath}
\mathtoolsset{showonlyrefs=true}
\usepackage{esint, esvect} % integrals and vectors
\usepackage{systeme} % equation system
\usepackage{commath} % partials and differentials
\usepackage{icomma} % smart comma ($0,2$ is a number)

% Floats
\usepackage{float}

% Tables
\usepackage{array,tabularx,tabulary,booktabs} % better tables
\usepackage{longtable}
\usepackage{multirow}

% Graphics
\usepackage[pdftex]{graphicx}
\usepackage{wrapfig}

% Theorems
\renewcommand{\proofname}{Доказательство}

%\theoremstyle{plain}
\newtheorem{theorem}{Теорема}[section]

%\theoremstyle{definition}
\newtheorem{definition}{Определение}
\newtheorem{corollary}{Следствие}[theorem]

\theoremstyle{remark}
\newtheorem{remark}{Замечание}

\usepackage[top=20mm,bottom=20mm,left=20mm,right=20mm]{geometry}

\usepackage{soul}
\usepackage{enumerate} % better numbered lists
\usepackage{hyperref}
\usepackage{xcolor}
\usepackage{tikz} % drawing

\footer{}{\thepage}{}
{}

%\usepackage{csquotes}
%\usepackage[style=authoryear,maxcitenames=2,backend=biber,sorting=nty]{biblatex}
%\bibliography{}

\renewcommand{\epsilon}{\ensuremath{\varepsilon}}
\renewcommand{\phi}{\ensuremath{\varphi}}
\renewcommand{\theta}{\vartheta}
\renewcommand{\kappa}{\ensuremath{\varkappa}}
\renewcommand{\le}{\ensuremath{\leqslant}}
\renewcommand{\leq}{\ensuremath{\leqslant}}
\renewcommand{\ge}{\ensuremath{\geqslant}}
\renewcommand{\geq}{\ensuremath{\geqslant}}

\usepackage[useregional]{datetime2}

% exam
\pointsinrightmargin
\marginpointname{ б.}

% custom maketitle
\usepackage{titling}
\setlength{\droptitle}{-4em}
\posttitle{\end{center}\vspace{-4em}}

\title{{\Large Геодезическая гравиметрия 2018}\\ 
    {\bf\Large Домашнее задание № 7}}
\author{}
\DTMsavedate{deadline}{2018-12-14}

\date{\normalsize\bf Крайний срок сдачи: \DTMusedate{deadline}}

\begin{document}
\maketitle
\thispagestyle{empty}
\begin{questions}
    \question Вычисление геопотенциальных чисел.

    Разность геопотенциальных чисел между двумя смежными пунктами нивелирной сети можно вычислить как
    \begin{equation*}
        W_i - W_{i + 1} = g_m \Delta h,
    \end{equation*}
    где $g_m$ --- среднее значение силы тяжести на пунктах $i$ и ${i+1}$, $\Delta h$ --- превышение.

    Обработку нивелирного полигона следует начинать с вычисления невязки $\epsilon$ полигона и
    уравнивания разностей $W_i - W_{i+1}$. Поправку $\Delta_i$ в измеренную разность вычисляется по
    формуле
    \begin{equation*}
        \Delta_i = -\dfrac{\epsilon}{\sum\dfrac{1}{p_i}} \dfrac{1}{p_i},
    \end{equation*}
    где $p_i$ -- вес, который можно выбрать обратно пропорциональным длине секции $L$ в километрах, то есть
    $p_i = \dfrac{1}{L}$.

    По уравненным разностям $W_i - W_{i+1}$ вычисляют уравненные геопотенциальные числа $W_0 - W_i$ 
    всех реперов, приняв в качестве исходного уравненное значения для одного из пунктов полигона из
    каталога, то есть из исходной сети. Теперь можно сравнить полученные величины со значениями из
    каталога. Они будут различными, потому что последние были получены в результате уравнивания всей
    сети, однако значения геопотенциальных чисел будут близкими и служат для контроля грубых ошибок
    вычислений.

    \question Вычисление нормальных и динамических высот.

    Нормальная высота $H^\gamma$ определяется по формуле
    \begin{equation*}
        H^\gamma = \dfrac{W_0 - W_i}{\gamma_m^i}
    \end{equation*}
    где $\gamma_m^i$ --- средняя нормальная сила тяжести для точке на высоте $H^\gamma / 2$. Поскольку
    нормальная высота входит в левую и правую части, то её находят последовательными приближениями. 

    В начальном приближении можно принять
    \begin{equation*}
        H^\gamma_{\textrm{прибл.}} = \dfrac{W_0 - W_{i}}{\gamma_0^i},
    \end{equation*}
    где $\gamma_0^i$ --- значение нормальной силы тяжести на эллипсоиде на широте репера. По приближённой высоте
    вычисляют нормальную силу тяжести
    \begin{equation*}
        \gamma_m^i = \gamma_0^i + \dpd{\gamma}{H} \dfrac{H_{\textrm{прибл.}}^\gamma}{2} =
        \gamma_0^i - 0,3086 \dfrac{H_{\textrm{прибл.}}^\gamma}{2}.
    \end{equation*}
    После этого вычисляют нормальную высоту в первом приближении, которого обычно бывает достаточно
    (если нет, то итерации повторяют).

    Динамическую высоту вычисляют по формуле
    \begin{equation*}
        H^d = \dfrac{W_0 - W_i}{\gamma_0^{45}},
    \end{equation*}
    где $\gamma_0^{45}$ --- значение нормальной силы тяжести на эллипсоиде на широте $45^\circ$.
    Если широта репера не равна $45^\circ$ и высота над уровнем моря велика, значение
    $\gamma_0^{45}$ будет сильно отличаться от $\gamma_m$, поэтому динамические высоты в этом случае
    сильно отличаются от нормальных. Если участок работ небольшой, то для вычисления динамических
    высот целесообразно использовать среднее значение силы тяжести $g_{\textrm{ср.}}$ на этом
    участке, тогда динамическая высота будет близка к сумме измеренных превышений $\Delta h$.

    В конце необходимо найти разность вычисленных динамических и нормальных высот $H^d - H^\gamma$.

    \question Вычисление теоретической невязки замкнутого нивелирного хода и разности нормальных высот.

    Разность нормальных высот необходимо вычислить для заданного нивелирного полигона сети UELN. Для
    каждых двух последовательных реперов $I$ и $K$ получают
    \begin{equation*}
        H^\gamma_{q_k} - H^\gamma_{q_i} = h_{ik} + f,
    \end{equation*}
    где $H^\gamma_{q_k} - H^\gamma_{q_i}$ --- нормальные высоты реперов $K$, $I$,  $h_{ik}$ ---
    измеренное превышение, $f$ --- поправка за переход к разности нормальных высот, которая
    вычисляется по формуле
    \begin{equation*}
        f = -\dfrac{1}{\gamma_m} \left( \gamma_{0_k} - \gamma_{0_i} \right) H_m +
        \dfrac{1}{\gamma_m} \left( g - \gamma \right)_m h_{ik}.
    \end{equation*}
    Здесь $\gamma_m$ --- приближённое значение нормальной силы тяжести, принимаемое постоянным для
    всей территории страны и равным 980 000 мГал, $\gamma_{0_k}, \gamma_{0_i}$ --- значения
    нормальных силы тяжести на отсчётном эллипсоиде в точках $K$ и $I$, вычисляемые по формуле
    Гельмерта, $H_m$ --- средняя высота реперов $I$ и $K$, $\left( g - \gamma \right)_m$ --- среднее
    арифметическое из аномалий силы тяжести. 

    Приближённую высоту репера получают по измеренным
    превышениям, прибавляя их последовательно к уравненной нормальной высоте начального пункта
    полигона. Полученные таким образом и округлённые до целых метров величины и используют в
    вычислениях.

    \paragraph{Примечание.} Пример вычисления поправок в превышение за переход к разностям нормальных высот приведен в
    <<Инструкции по вычислению нивелировок. М. Изд-во <<Недра>>, 1971, с. 102>>.

    Параллельно с разностями нормальных высот получают теоретическую невязку нивелирного полигона,
    равную сумме поправок $f$ с обратным знаком $(\sum\Delta h)_\textrm{теор.} =  -\sum f$, и
    оценивают невязку $\sum\Delta h - (\sum\Delta h)_\textrm{теор.}$, которую, также как и сумму
    измеренных превышений $\sum\Delta h$, следует сравнить с 
    допустимой невязкой в полигонах и по линиям. Для I класса она составляет 
    $3\,\textrm{мм}\sqrt{L}$, 
    а для II класса $5\,\textrm{мм}\sqrt{L}$. Здесь $L$ --- длина хода в километрах.

    Случайная ошибка нивелирования $\eta$ определяется через полученную невязку
    \begin{equation*}
        \eta = \dfrac{\sum\Delta h - (\sum\Delta h)_\textrm{теор.}}{\sqrt{L}}
    \end{equation*}
    и выражается в $\textrm{мм} / \sqrt{\textrm{км}}$.
\end{questions}

%\printbibliography
\end{document}
