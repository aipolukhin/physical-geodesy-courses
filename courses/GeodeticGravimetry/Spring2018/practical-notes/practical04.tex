\documentclass[11pt, a4paper]{article}

% Languages and fonts
\usepackage{cmap} 
\usepackage[T2A]{fontenc}
\usepackage[utf8]{inputenc} 
\usepackage[english, russian]{babel}
\usepackage{microtype}
\usepackage{indentfirst}
\frenchspacing

% Mathematics
\usepackage{amsmath, amssymb, amsfonts, amsthm, mathtools, fixmath}
\mathtoolsset{showonlyrefs=true}
\usepackage{esint, esvect} % integrals and vectors
\usepackage{systeme} % equation system
\usepackage{commath} % partials and differentials
\usepackage{icomma} % smart comma ($0,2$ is a number)

% Floats
\usepackage{float}

% Tables
\usepackage{array,tabularx,tabulary,booktabs} % better tables
\usepackage{longtable}
\usepackage{multirow}

% Graphics
\usepackage[pdftex]{graphicx}
\usepackage{wrapfig}

% Theorems
\renewcommand{\proofname}{Доказательство}

\theoremstyle{plain}
\newtheorem{theorem}{Теорема}[section]

\theoremstyle{definition}
\newtheorem{definition}{Определение}
\newtheorem{corollary}{Следствие}[theorem]
\newtheorem{problem}{Задача}[section]

\theoremstyle{remark}
\newtheorem{remark}{Замечание}
\newtheorem*{solution}{Решение}

\usepackage[top=20mm,bottom=20mm,left=20mm,right=20mm]{geometry}

\usepackage{lastpage} % how many pages

\usepackage{soul}

\usepackage{framed} % easy frames
\usepackage{enumerate} % better numbered lists

\usepackage{hyperref}
\usepackage{xcolor}

\usepackage{tikz} % drawing

\usepackage{csquotes}
\usepackage[style=numeric,backend=biber,sorting=none]{biblatex}
\addbibresource{../../../../bibliography.bib}

\renewcommand{\epsilon}{\ensuremath{\varepsilon}}
\renewcommand{\phi}{\ensuremath{\varphi}}
%\renewcommand{\theta}{\vartheta}
\renewcommand{\kappa}{\ensuremath{\varkappa}}
\renewcommand{\le}{\ensuremath{\leqslant}}
\renewcommand{\leq}{\ensuremath{\leqslant}}
\renewcommand{\ge}{\ensuremath{\geqslant}}
\renewcommand{\geq}{\ensuremath{\geqslant}}

\usepackage[useregional]{datetime2}

% custom maketitle
\usepackage{titling}
\setlength{\droptitle}{-4em}
\posttitle{\end{center}\vspace{-3em}}

\title{{\Large Геодезическая гравиметрия 2018}\\ 
    {\bf\Large Практическое занятие № 4} \\
{\Large Притяжение тел простой формы II}}
\author{}
\DTMsavedate{lessondate}{2018-03-06}
\date{\DTMusedate{lessondate}}

\begin{document}
\maketitle

\section{Притяжение тел простой формы}
\subsection{Притяжение диска}

Силу притяжение бесконечно тонкого однородного диска радиуса $a$ можно представить
через притяжение простого слоя $V = G\int\limits_{\sigma}\frac{\mu\dif\sigma}{r}$.
Удобно воспользоваться цилиндрической системой координат ($\rho$, $\alpha$, $z$), где $\rho$ ---
полярный радиус, $\alpha$ --- полярный угол, $z$ --- аппликата точки. Таким образом, цилиндрическая
система координат является расширением полярной (плоской) системы координат.

Найдем силу притяжения диска для точки, лежащей на его оси симметрии. Можно записать
\begin{equation*}
    F = \dpd{V}{z} = - G\mu\int\limits_{\sigma}\dfrac{\dif\sigma}{r^2}\cos\left( r, z \right),
\end{equation*}
\begin{equation*}
    \dif\sigma = \rho\dif\alpha\dif\rho,
\end{equation*}
откуда
\begin{equation*}
    F = \dpd{V}{z} = - G\mu\int\limits_{\sigma}\dfrac{\rho\dif\rho\dif\alpha}{r^2}\cos\left( r, z \right).
\end{equation*}
Заметим, что
\begin{equation*}
    r^2 = \rho^2 + z^2,\qquad r\dif r = \rho\dif rho,\qquad \cos\left( r, z \right) = \dfrac{z}{r},
\end{equation*}
тогда
\begin{equation*}
    F_e = - G\mu\int\limits_{z}^{\sqrt{z^2 + a^2}}\int\limits_{0}^{2\pi}\dfrac{r\dif
    r\dif\alpha}{r^3} = -2\pi G\mu z\int\limits_{z}^{\sqrt{z^2 + a^2}}\dfrac{\dif z}{r^2} =
    -2\pi G\mu\left[ 1 - \dfrac{z}{\sqrt{z^2 + a^2}}\right].
\end{equation*}
Если $z < 0$, то $F_e = +2\pi G\mu\left[ 1 - \dfrac{z}{\sqrt{z^2 + a^2}}\right]$. \\
Теперь найдём значение силы на самом слое
\begin{equation*}
    \lim\limits_{z\to 0} F_e = \pm2\pi G\mu\lim\limits_{z\to 0} \left[ 1 - \dfrac{z}{\sqrt{z^2 +
    a^2}}\right] = \pm2\pi G\mu = const.
\end{equation*}
Прямое значение на слое, равное среднему из двух пределов, $F_0 = 0$, что можно было бы получить и
чисто опираясь на физический смысл силы. Таким образом, сила притяжение
терпит разрыв на величину $4\pi G\mu$ при переходе через слой.

\subsection{Притяжение плоскости}
Силу притяжения плоскости получим из силы притяжения диска, радиус которого стремится к
бесконечности
\begin{equation*}
    F = \pm2\pi G\mu\lim\limits_{a\to\infty} \left[ 1 - \dfrac{z}{\sqrt{z^2 + a^2}}\right] =
    \pm2\pi G\mu = const,
\end{equation*}
откуда следует, что сила притяжения плоскости не зависит от расстояния до неё.

\subsection{Притяжение цилиндра}
Будем рассматривать притяжение точки, находящейся на оси однородного цилиндра высотой $H$ и радиуса $a$. Для
простоты и приложений будем считать, что точка находится на верхней плоскости его основания ($z = H$).
Элементарная сила бесконечно тонкого диска будет равна
\begin{equation*}
    \dif F_z = -2\pi G\delta\left[ 1 - \dfrac{z}{\sqrt{z^2 + a^2}}\right] \dif z,
\end{equation*}
тогда для всего цилиндра
\begin{equation*}
    F_H = -2\pi G\delta\int\limits_{0}^{H}\left[ 1 - \dfrac{z}{\sqrt{z^2 + a^2}}\right] \dif z =
    -2\pi G\delta \left[ z - \sqrt{a^2 + z^2} \right] \bigg|_0^{H} =
    -2\pi G\delta\left( H - \sqrt{a^2 + H^2} + a \right).
\end{equation*}
Если $a >> H$, то можно представить (ряд Тейлора)
\begin{equation*}
    \sqrt{a^2 + H^2} = a\sqrt{1 + \dfrac{H^2}{a^2}} \approx
    a+\dfrac{H^2}{2a}-\dfrac{H^4}{8a^3}+\dots,
\end{equation*}
тогда
\begin{equation*}
    F_H = 
    -2\pi G\delta H\left( 1 - \dfrac{H}{2a} + \dots \right).
\end{equation*}

\subsection{Притяжение плоскопараллельного слоя}
Силу притяжения плоскопараллельного (то есть заключенного между двумя плоскостями) слоя толщиной $H$
получим из силы притяжения цилиндра, радиус которого стремится к бесконечности
\begin{equation*}
    F = 
    -2\pi G\delta H\lim\limits_{a\to\infty}\left( 1 - \dfrac{H}{2a} + \dots \right) = 
    -2\pi G\delta H = const.
\end{equation*}
Таким образом, сила притяжения плоскопараллельного слоя является величиной
постоянной. 

Редукция (поправка), вводимая в измеренные значения силы тяжести и вычисляемая по формуле $-2\pi
G\delta H$ называется редукцией Буге. Так может быть учтено притяжение топографических масс, снега,
грунтовых вод и т.д. То есть в тех случаях, когда высота притягивающих масс много меньше их радиуса.
Плотность $\delta$, конечно, будет меняться.

Поле, создаваемое плоскостью или плоскопараллельным слоем называется однородным. Сила здесь имеет одно и
то же направление --- перпендикулярное плоскости --- и одинаковую величину в любой точке. Уровенные
поверхности такого поля --- плоскости, параллельные слою. В прикладной геодезии, например, в
рядовых строительных работах, почти всегда предполагается, что работы проводятся именно в однородном поле.

\section{Свойства потенциала притяжения}

\subsection{Аналитические свойства потенциала объёмных масс}
Потенциал объёмных масс является функцией непрерывной, однозначной и конечной во всём проcтранстве.
Эими же свойствами обладают и первые производные потенциала.

Исследуем вторые производные. Наййдём вторые производные потенциала объёмных масс в прямоугольных
координатах, получим
\begin{align*}
    &\dpd[2]{V}{x} = G\iiint\limits_{\tau} \delta \left[ \dfrac{3 \left( x - x_1 \right)^2}{r^5} -
    \dfrac{1}{r^3} \right]\dif\tau, \\
    &\dpd[2]{V}{y} = G\iiint\limits_{\tau} \delta \left[ \dfrac{3 \left( y - y_1 \right)^2}{r^5} -
    \dfrac{1}{r^3} \right]\dif\tau, \\
    &\dpd[2]{V}{z} = G\iiint\limits_{\tau} \delta \left[ \dfrac{3 \left( z - z_1 \right)^2}{r^5} -
    \dfrac{1}{r^3} \right]\dif\tau.
\end{align*}
Складываем все три равенства, тогда
\begin{equation*}
    \dpd[2]{V}{x} + \dpd[2]{V}{y} + \dpd[2]{V}{y} = 
    G\iiint\limits_{\tau} \delta \left[ \dfrac{3 \left( x - x_1 \right)^2 +
        3 \left( y - y_1 \right)^2 +3 \left( z - z_1 \right)^2}{r^5} - \dfrac{3}{r^3} 
    \right]\dif\tau =
    G\iiint\limits_{\tau} \delta \left[ \dfrac{3}{r^3} - \dfrac{3}{r^3} \right]\dif\tau = 0.
\end{equation*}
То есть
\begin{equation*}
    \Delta V = \dpd[2]{V}{x} + \dpd[2]{V}{y} + \dpd[2]{V}{y} = 0.
\end{equation*}
Это уравнение Лапласа --- дифференциальное уравнение в частных производных второго порядка. Здесь
$\Delta = \dpd[2]{}{x} + \dpd[2]{}{y} + \dpd[2]{}{y}$ --- оператор Лапласа или <<лапласиан>>.
Функция, непрерывная в некоторой области вместе со своими частными производными первого и второго
порядков и удовлетворяющая уравнению Лапласа, называется \textbf{гармонической}. Таким образом,
потенциал притяжения во внешнем пространстве является гармонической функцией.

Внутри притягивающих масс потенциал удовлетворяет уравнению Пуассона
\begin{equation*}
    \Delta V = -4\pi G\delta
\end{equation*}
и, следовательно, не является гармонической функцией. Легко заметить, что вне притягивающих масс
$\delta = 0$ уравнение Пуассона переходит в уравнение Лапласа.

Исследуем поведение потенциала на бесконечности. Напишем неравенство
\begin{equation*}
    G\int\limits_\tau\dfrac{\delta\dif\tau}{r_{min}} >
    G\int\limits_\tau\dfrac{\delta\dif\tau}{r} >
    G\int\limits_\tau\dfrac{\delta\dif\tau}{r_{max}},
\end{equation*}
где $r_{min}$, $r_{max}$ --- минимальное и максимальное расстояние от притягиваемой точки до
притягиваемого тела. Поскольку $\int\limits_\tau\delta\dif\tau = M$, то
\begin{equation*}
    G\dfrac{M}{r_{min}} >
    G\dfrac{M}{r} >
    G\dfrac{M}{r_{max}}.
\end{equation*}
Образуем производную $\pd{V}{r} = -G\int\limits_\tau\dfrac{\delta\dif\tau}{r^2}$ и неравенство
\begin{equation*}
    G\dfrac{M}{r^2_{min}} >
     \left|\dpd{V}{r}\right| >
    G\dfrac{M}{r^2_{max}}.
\end{equation*}
Теперь умножаем предпоследнее неравенство на $r$, а последнее --- на $r^2$, тогда
\begin{equation*}
    G\dfrac{Mr}{r_{min}} >
    Vr >
    G\dfrac{Mr}{r_{max}}, \qquad
    G\dfrac{Mr^2}{r^2_{min}} >
     \left|\dpd{V}{r}\right| r^2 >
    G\dfrac{M r^2}{r^2_{max}}.
\end{equation*}
Пусть $r\to\infty$, тогда
\begin{equation*}
    \lim\limits_{r\to\infty}\dfrac{r}{r_{min}} = 1,\qquad 
    \lim\limits_{r\to\infty}\dfrac{r}{r_{max}} = 1,
\end{equation*}
и
\begin{equation*}
    \lim\limits_{r\to\infty} V = 0,\qquad 
    \lim\limits_{r\to\infty} rV = GM,\qquad
    \lim\limits_{r\to\infty} \left|\dpd{V}{r}\right| r^2 = GM.
\end{equation*}
Функция, удовлетворяющая всем трём последним пределам называется \textbf{регулярной на
бесконечности}. Следовательно, потенциал является функцией, регулярной на бесконечности.

\subsection{Локальные свойства потенциала}
Пусть точка $P\left( x,y,z \right)$ переместилась в точку $P_1\left( x + \dif x, y + \dif y, z +
\dif z \right)$, находящуюся на бесконечно малом расстоянии $dl$ от $P$, тогда
\begin{equation*}
    \dif V = \pd{V}{x}\dif x + \pd{V}{y}\dif y + \pd{V}{z}\dif z.
\end{equation*}
Для производных и направления $\dif l$ можно записать
\begin{align*}
    \dpd{V}{x} = |\vv{F}|\cos\left( \vv{F}, x \right),\qquad \dif x = \dif l\cos\left( x, \dif l
    \right),\\
    \dpd{V}{y} = |\vv{F}|\cos\left( \vv{F}, y \right),\qquad \dif y = \dif l\cos\left( y, \dif l
    \right),\\
    \dpd{V}{z} = |\vv{F}|\cos\left( \vv{F}, z \right),\qquad \dif z = \dif l\cos\left( z, \dif l
    \right),
\end{align*}
тогда
\begin{equation*}
    \dif V = |\vv{F}|\left[
        \cos\left( \vv{F}, x \right)\cos\left( x, \dif l \right) +
        \cos\left( \vv{F}, y \right)\cos\left( y, \dif l \right) +
    \cos\left( \vv{F}, z \right)\cos\left( z, \dif l \right) \right]\dif l,
\end{equation*}
или
\begin{equation*}
    \dif V = |\vv{F}|cos\left( \vv{F}, \dif l \right)\dif l.
\end{equation*}
Из этого выражения силедуют несколько важных и полезных свойств.
\begin{enumerate}
    \item Производная потенциала $\dif V/ \dif l$ по любому направлению $l$ равна проекции силы на
        это направление
        \begin{equation*}
            \od{V}{l} =  |\vv{F}|cos\left( \vv{F}, \dif l \right).
        \end{equation*}
    \item Если $\cos\left( \vv{F}, \dif l \right) = 1$, то изменение потенциала максимально и
        $dl$ направлено по линии действия силы. Тогда $\vv{F}$ является вектор--градиентом
        потенциала
        \begin{equation*}
            \vv{F} = \textrm{grad}\,V.
        \end{equation*}
    \item Если $\cos\left( \vv{F}, \dif l \right) = 0$, то $\dif V = 0$ и 
        \begin{equation*}
            V\left( x, y, z \right) = const.
        \end{equation*}
        Уравнение поверхности, в каждой точке которой сила направлена по нормали, а потенциал ---
        постоянен. Это уровенная или эквипотенциальная поверхность. Заметим, что на силу не
        накладываются никакие ограничения и, вообще говоря, она может меняться на уровенной
        поверхности.
    \item Если $\cos\left( \vv{F}, \dif l \right) = -1$ и $\dif l = \dif h$ --- расстояние между
        бусконечно близкими уровенными поверхностями, то
        \begin{equation*}
            \dif V = - |\vv{F}| \dif h, \qquad \dif h = -\dfrac{\dif V}{|\vv{F}|}.     
        \end{equation*}
    \item Пусть $\dif h\to 0$, тогда получим кривую, перпендикулярную ко всем уровенным
        поверхностям, в каждой точке которой касательная совпадает с направлением силы. Эта кривая
        называется \textbf{силовой линией}. Найдём длину отрезка $OM$ силовой линии между двумя уровенным
        поверхностями
        \begin{equation*}
            h = -\int\limits_{O}^{M} \dfrac{\dif V}{|\vv{F}|} = - \dfrac{V_M - V_O}{F_m} =
            \dfrac{V_O - V_M}{F_m},
        \end{equation*}
        где $F_m$ --- среднее интегральное значение силы на отрезке $OM$. Таким образом, для
        определения высоты в гравитационном поле необходимо знать разность потенциалов и среднее
        значение силы между уровенными поверхностями. Эта формула имеет важное значение в теории
        высот.
\end{enumerate}

%\printbibliography
\end{document}
