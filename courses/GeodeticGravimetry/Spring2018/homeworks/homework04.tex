\documentclass[11pt, a4paper,addpoints]{exam}

% Languages and fonts
\usepackage{cmap} 
\usepackage[T2A]{fontenc}
\usepackage[utf8]{inputenc} 
\usepackage[english, russian]{babel}
\usepackage{microtype}
\usepackage{indentfirst}
\frenchspacing

% Mathematics
\usepackage{amsmath, amssymb, amsfonts, amsthm, mathtools, fixmath}
\mathtoolsset{showonlyrefs=true}
\usepackage{esint, esvect} % integrals and vectors
\usepackage{systeme} % equation system
\usepackage{commath} % partials and differentials
\usepackage{icomma} % smart comma ($0,2$ is a number)
\usepackage{mathabx}% astronomy

% Floats
\usepackage{float}

% Tables
\usepackage{array,tabularx,tabulary,booktabs} % better tables
\usepackage{longtable}
\usepackage{multirow}

% Graphics
\usepackage[pdftex]{graphicx}
\usepackage{wrapfig}

% Theorems
\renewcommand{\proofname}{Доказательство}

%\theoremstyle{plain}
\newtheorem{theorem}{Теорема}[section]

%\theoremstyle{definition}
\newtheorem{definition}{Определение}
\newtheorem{corollary}{Следствие}[theorem]

\theoremstyle{remark}
\newtheorem{remark}{Замечание}

\usepackage[top=20mm,bottom=20mm,left=20mm,right=20mm]{geometry}

\usepackage{soul}
\usepackage{enumerate} % better numbered lists
\usepackage{hyperref}
\usepackage{xcolor}
\usepackage{tikz} % drawing

%\usepackage{csquotes}
%\usepackage[style=authoryear,maxcitenames=2,backend=biber,sorting=nty]{biblatex}
%\bibliography{}

\renewcommand{\epsilon}{\ensuremath{\varepsilon}}
\renewcommand{\phi}{\ensuremath{\varphi}}
\renewcommand{\theta}{\vartheta}
\renewcommand{\kappa}{\ensuremath{\varkappa}}
\renewcommand{\le}{\ensuremath{\leqslant}}
\renewcommand{\leq}{\ensuremath{\leqslant}}
\renewcommand{\ge}{\ensuremath{\geqslant}}
\renewcommand{\geq}{\ensuremath{\geqslant}}

\usepackage[useregional]{datetime2}

% exam
\pointsinrightmargin
\marginpointname{ б.}

% custom maketitle
\usepackage{titling}
\setlength{\droptitle}{-4em}
\posttitle{\end{center}\vspace{-4em}}

\title{{\Large Геодезическая гравиметрия 2018}\\ 
    {\bf\Large Домашнее задание № 4}}
\author{}
\DTMsavedate{deadline}{2018-03-26}

\date{\normalsize\bf Крайний срок сдачи: \DTMusedate{deadline}}

\begin{document}
\maketitle
\thispagestyle{empty}
\begin{questions}
        \question[1] Вычислить
        \begin{equation*}
            \Delta\left( \dmd{1/r}{3}{x}{}{y}{2} \right),
        \end{equation*}
        где
        \begin{equation*}
            r^2 = \left( x - a \right)^2 + \left( y - a \right)^2 + \left( z - a \right)^2.
        \end{equation*}
        \question[1] Уравнеие Лапласа в сферических координатах ($r$, $\theta$, $\lambda$) имеет
        следующий вид
        \begin{equation*}
            \Delta f = \dfrac{1}{r}\dpd{}{r}\left( r^2 \dpd{f}{r} \right) +
            \dfrac{1}{r^2\sin\theta}\dpd{}{\theta}\left( \sin\theta \dpd{f}{\theta} \right) +
            \dfrac{1}{r^2\sin^2\theta}\dpd[2]{f}{\lambda} = 0.
        \end{equation*}
        Докажите, что функция 
        \begin{equation*}
            f\left( r, \theta, \lambda \right) = \dfrac{1}{r^4}\sin^2\theta \cos\theta\cos{2\lambda}
        \end{equation*}
        является гармонической для всех $r\neq 0$.
        % Локальные свойства
        \question[1] Расстояние между очень близкими уровенными поверхностями потенциала $W$ силы
        $g$ в точках $A$ и $B$ равны $H_1$ и $H_2$. Определить силу $q$ в точке $B$, считая
        известной силу $g_A$ в точке $A$.
        \question[2] Вычислить притяжение колец Сатурна на оси вращения планеты. Найти значение
        притяжения на высоте $z = 100\times i\,\text{км}$ ($i$ --- вариант). Масса колец Сатурна 
        $9,6\times 10^{20}\,\text{кг}$, внутренний
        радиус $\rho_1 = 72\ 000\,\text{км}$, внешний $\rho_2 = 139\ 000\,\text{км}$, толщиной колец
        пренебречь. Полярный радиус Сатурна $54\ 400\,\text{км}$.
\end{questions}
%\printbibliography
\end{document}
