\documentclass[11pt, a4paper,addpoints]{exam}

% Languages and fonts
\usepackage{cmap} 
\usepackage[T2A]{fontenc}
\usepackage[utf8]{inputenc} 
\usepackage[english, russian]{babel}
\usepackage{microtype}
\usepackage{indentfirst}
\frenchspacing

% Mathematics
\usepackage{amsmath, amssymb, amsfonts, amsthm, mathtools, fixmath}
\mathtoolsset{showonlyrefs=true}
\usepackage{esint, esvect} % integrals and vectors
\usepackage{systeme} % equation system
\usepackage{commath} % partials and differentials
\usepackage{icomma} % smart comma ($0,2$ is a number)
\usepackage{mathabx}% astronomy

% Floats
\usepackage{float}

% Tables
\usepackage{array,tabularx,tabulary,booktabs} % better tables
\usepackage{longtable}
\usepackage{multirow}

% Graphics
\usepackage[pdftex]{graphicx}
\usepackage{wrapfig}

% Theorems
\renewcommand{\proofname}{Доказательство}

%\theoremstyle{plain}
\newtheorem{theorem}{Теорема}[section]

%\theoremstyle{definition}
\newtheorem{definition}{Определение}
\newtheorem{corollary}{Следствие}[theorem]

\theoremstyle{remark}
\newtheorem{remark}{Замечание}

\usepackage[top=20mm,bottom=20mm,left=20mm,right=20mm]{geometry}

\usepackage{soul}
\usepackage{enumerate} % better numbered lists
\usepackage{hyperref}
\usepackage{xcolor}
\usepackage{tikz} % drawing

%\usepackage{csquotes}
%\usepackage[style=authoryear,maxcitenames=2,backend=biber,sorting=nty]{biblatex}
%\bibliography{}

\renewcommand{\epsilon}{\ensuremath{\varepsilon}}
\renewcommand{\phi}{\ensuremath{\varphi}}
\renewcommand{\theta}{\vartheta}
\renewcommand{\kappa}{\ensuremath{\varkappa}}
\renewcommand{\le}{\ensuremath{\leqslant}}
\renewcommand{\leq}{\ensuremath{\leqslant}}
\renewcommand{\ge}{\ensuremath{\geqslant}}
\renewcommand{\geq}{\ensuremath{\geqslant}}

\usepackage[useregional]{datetime2}

% exam
\pointsinrightmargin
\marginpointname{ б.}

% custom maketitle
\usepackage{titling}
\setlength{\droptitle}{-4em}
\posttitle{\end{center}\vspace{-4em}}

\title{{\Large Теория фигур планет и гравиметрия 2018}\\ 
    {\bf\Large Домашнее задание № 2}}
\author{}
\DTMsavedate{deadline}{2018-03-11}

\date{\normalsize\bf Крайний срок сдачи: \DTMusedate{deadline}}

\begin{document}
\maketitle
\thispagestyle{empty}
\begin{questions}
    \question[1] В каких единицах выражаются работа и потенциал?
    \question[2] Известный французский писатель Жуль Верн в 1865 году в романе <<С Земли на Луну>> поставил
    следующий вопрос: какова должна быть начальная скорость запущенного с
    поверхности Земли пушечного ядра для того, чтобы оно долетело до Луны? Легко догадаться, что
    для достижения цели необходимо, чтобы пущенное с искомой начальной скоростью ядро
    достигло такой точки, в которой
    сила притяжения Луны будет больше силы притяжения Земли. 
    На каком расстоянии от поверхности Земли расположена такая <<точка невозврата>>, после
    пересечения которой ядро не вернётся на Землю? Массы Земли и Луны считать сосредоточенными в их
    центрах, а сами два небесных тела --- неподвижными. Поверхность Земли --- сфера.\\
    Исходные данные:
    \begin{align*}
        &GM_\Earth = 3,9860\times10^{14}\ \textrm{м$^3$с$^{-2}$ -- геоцентрическая гравитационная постоянная}, \\
        &GM_\Moon = 4,9049\times10^{12}\ \textrm{м$^3$с$^{-2}$ -- селеноцентрическая гравитационная постоянная}, \\
        &R_\Earth = 6371\times10^3\ \textrm{м -- средний радиус Земли}, \\
        &R = 3844\times10^5\ \textrm{м -- расстояния между центрами Земли и Луны}.
    \end{align*}
    \question[2] Две точечные массы $m_1 = m_2 = m$ расположены в плоскости $xy$ на расстоянии $d$ друг от друга.
    Притягиваемая единичная масса рсположена в вершине равностороннего треугольника с основанием $d$
    (т.\,е. $m_1$ и $m_2$ --- две другие вершины, а высота $z = \frac{\sqrt{3}}{2}d$). 
    \begin{parts}
        \part Написать выражение для потенциала
    притяжения $V$ поля двух точечных масс. 
        \part Написать выражения для составляющих результирующей силы 
    притяжения $F_x, F_y, F_z$. 
        \part Вычислить значение силы притяжение $F$. Куда она направлена?\\
    \end{parts}
    Исходные данные ($i$ --- вариант): $m = 100\times i\,\text{кг}$, $d = 10\times i\,\text{м}$.
    \question[2] Пользуясь условием и результатами предыдущей задачи, написать уравнение уровенных
    поверхностей поля двух точечных масс. Построить в плоскости $xy$ след уровенной поверхности,
        на которой потенциал равен $C$. Рассмотреть несколько случаев, когда $C \leq \dfrac{4Gm}{d}$.
        \paragraph{Подсказка.} Следом поверхности в плоскости называется двухмерная кривая, образованная в
        результате
        пересечения поверхности этой плоскостью. Следы в плоскостях, параллельных $xy$ могут быть
        заданы через уравнение $z = f\left( x, y \right) = C$, где $C$ --- постоянная.
        Горизонтали на топографической карте являются следом рельефа.
        Таким образом, изолинии изображают проекцию следа на плоскость $z=0$.
\end{questions}
%\printbibliography
\end{document}
