\documentclass[11pt, a4paper,addpoints]{exam}

% Languages and fonts
\usepackage{cmap} 
\usepackage[T2A]{fontenc}
\usepackage[utf8]{inputenc} 
\usepackage[english, russian]{babel}
\usepackage{microtype}
\usepackage{indentfirst}
\frenchspacing

% Mathematics
\usepackage{amsmath, amssymb, amsfonts, amsthm, mathtools, fixmath}
\mathtoolsset{showonlyrefs=true}
\usepackage{esint, esvect} % integrals and vectors
\usepackage{systeme} % equation system
\usepackage{commath} % partials and differentials
\usepackage{icomma} % smart comma ($0,2$ is a number)
\usepackage{mathabx}% astronomy

% Floats
\usepackage{float}

% Tables
\usepackage{array,tabularx,tabulary,booktabs} % better tables
\usepackage{longtable}
\usepackage{multirow}

% Graphics
\usepackage[pdftex]{graphicx}
\usepackage{wrapfig}

% Theorems
\renewcommand{\proofname}{Доказательство}

%\theoremstyle{plain}
\newtheorem{theorem}{Теорема}[section]

%\theoremstyle{definition}
\newtheorem{definition}{Определение}
\newtheorem{corollary}{Следствие}[theorem]

\theoremstyle{remark}
\newtheorem{remark}{Замечание}

\usepackage[top=20mm,bottom=20mm,left=20mm,right=20mm]{geometry}

\usepackage{soul}
\usepackage{enumerate} % better numbered lists
\usepackage{hyperref}
\usepackage{xcolor}
\usepackage{tikz} % drawing

%\usepackage{csquotes}
%\usepackage[style=authoryear,maxcitenames=2,backend=biber,sorting=nty]{biblatex}
%\bibliography{}

\renewcommand{\epsilon}{\ensuremath{\varepsilon}}
\renewcommand{\phi}{\ensuremath{\varphi}}
\renewcommand{\theta}{\vartheta}
\renewcommand{\kappa}{\ensuremath{\varkappa}}
\renewcommand{\le}{\ensuremath{\leqslant}}
\renewcommand{\leq}{\ensuremath{\leqslant}}
\renewcommand{\ge}{\ensuremath{\geqslant}}
\renewcommand{\geq}{\ensuremath{\geqslant}}

\usepackage[useregional]{datetime2}

% exam
\pointsinrightmargin
\marginpointname{ б.}

% custom maketitle
\usepackage{titling}
\setlength{\droptitle}{-4em}
\posttitle{\end{center}\vspace{-4em}}

\title{{\Large Теория фигур планет и гравиметрия 2018}\\ 
    {\bf\Large Домашнее задание № 5}}
\author{}
\DTMsavedate{deadline}{2018-04-01}

\date{\normalsize\bf Крайний срок сдачи: \DTMusedate{deadline}}

\begin{document}
\maketitle
\thispagestyle{empty}
\begin{questions}
        \question[5] Построить графики поправок в измеренные значения силы тяжести 
        \begin{parts}
            \part за движение полюса,
            \part за приливные вариации силы тяжести, 
            \part за изменение атмосферного давления
        \end{parts}
        для фундаментального гравиметрического пункта <<ЦНИИГАиК>> (Москва):
        \begin{table}[h]
            \centering
            \begin{tabular}{|c|c|c|}
                \hline
                 $\phi\, [^\circ]$ & $\lambda\, [^\circ]$ &$H\, [\text{м}]$ \\\hline
                 55,85503 & 37,51604 & 153\\\hline
            \end{tabular}
        \end{table}

        График построить на 60 дней, начиная с 1 мая 2016 года + $30\times (i - 1)$ дней, где $i$ ---
        вариант по журналу промежуточной успеваемости. Воспользоваться указанием на следующей странице.
\end{questions}
\newpage
\thispagestyle{empty}
\begin{center}
    \textbf{Указание к вычислению поправок в измеренное значение силы тяжести}
\end{center}
\begin{flushleft}
    \textbf{Поправка за движение полюса} 
\end{flushleft}
Поправка за движение полюса вычисляется по формуле
    \begin{equation*}
            \Delta g_p = - 1,164\times 10^8 \omega^2 a \sin{2\phi} \left( \dfrac{x_p}{\rho''} \cos\lambda -
            \dfrac{y_p}{\rho''}\sin\lambda \right)\quad[\text{мкГал}],
        \end{equation*}
        где $\omega = 7,292115\times 10^{-5}\,\text{рад/с}$ --- угловая скорость вращения Земли; $a
        = 6378137\,\text{м}$ --- большая полуось; $\phi$, $\lambda$ --- широта
        и долгота пункта; $x_p$, $y_p$ --- координаты полюса; $\rho'' = \dfrac{360^\circ \times 60'
        \times 60''}{2\pi} \approx 206265''$.\\
        Данные взять с сайта Международной службы вращения Земли (IERS):\par
        \url{https://datacenter.iers.org/eop/-/somos/5Rgv/latest/9}\\
        Описание данных:\par
        \url{https://datacenter.iers.org/eop/-/somos/5Rgv/getMeta/9/finals2000A.all} \\
        Вырезка на заданные даты приведена в файле \textrm{eop.dat}. В файле первые столбцы:\\
        год, месяц, день, модифицированная юлианская дата (MJD), $x_p$, $\sigma_x$, $y_p$, $\sigma_y$, \dots\\ 
        где $x_p$ и  $y_p$ --- координаты полюса в угловых секундах; год, месяц, день - двухзначные поля
        с фиксированной шириной -- 2. 
\begin{flushleft}
    \textbf{Поправка за приливные изменения силы тяжести} 
\end{flushleft}
Поправка за прилив вычислена по отечественной программе ATLANTIDA3.1\_2014 \\
Подробнее: \url{http://www.ifz.ru/applied/prognoz-parametrov-zemnykh-prilivov/}\\
При вычислении учтены упругие свойства Земли и океанические приливы.\\
Поправка за прилив дана в файле \textrm{tides.dat} с дискретностью 10 минут. Файл имеет структуру:\\
год-месяц-день час:минута:секунда, поправка за прилив [мкГал]\\

\begin{flushleft}
    \textbf{Поправка за изменение атмосферного давления} 
\end{flushleft}

Поправка за изменение атмосферного давления вычисляется по формуле
\begin{equation*}
    \Delta g_a = K \left( P - P_0 \right),
\end{equation*}
где $K$ --- барометрический фактор, $P$ --- атмосферное давление в милибарах, $P_0$ --- нормальное
(модельное) атмосферное давление.

По рекомендации Международной ассоциации геодезии (IAG) $K = 0,3\, \text{мкГал/мбар}$.

Нормальное атмосферное давление вычисляется так
\begin{equation*}
    P_0 = 1013,25 \left( 1 - \dfrac{0,0065 H}{288,15} \right)^{5.2599}\quad[\text{мбар}],
\end{equation*}
где $H$ --- высота пункта над уровнем моря.

Информация о давлении дана в файле \textrm{pressure.dat} с дискретностью 10 минут. Измерения
получены с приливного гравиметра gPhone №117. Файл имеет структуру:\\
год-месяц-день час:минута:секунда, давление [мбар]\\

%\printbibliography
\end{document}
