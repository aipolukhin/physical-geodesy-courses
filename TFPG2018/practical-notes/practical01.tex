\documentclass[11pt, a4paper]{article}

% Languages and fonts
\usepackage{cmap} 
\usepackage[T2A]{fontenc}
\usepackage[utf8]{inputenc} 
\usepackage[english, russian]{babel}
\usepackage{microtype}
\usepackage{indentfirst}
%\frenchspacing

% Mathematics
\usepackage{amsmath, amssymb, amsfonts, amsthm, mathtools, fixmath}
\mathtoolsset{showonlyrefs=true}
\usepackage{esint, esvect} % integrals and vectors
\usepackage{systeme} % equation system
\usepackage{commath} % partials and differentials
\usepackage{icomma} % smart comma ($0,2$ is a number)

% Floats
\usepackage{float}

% Tables
\usepackage{array,tabularx,tabulary,booktabs} % better tables
\usepackage{longtable}
\usepackage{multirow}

% Graphics
\usepackage[pdftex]{graphicx}
\usepackage{wrapfig}

% Theorems
\renewcommand{\proofname}{Доказательство}

\theoremstyle{plain}
\newtheorem{theorem}{Теорема}[section]

\theoremstyle{definition}
\newtheorem{definition}{Определение}
\newtheorem{corollary}{Следствие}[theorem]
\newtheorem{problem}{Задача}[section]

\theoremstyle{remark}
\newtheorem{remark}{Замечание}
\newtheorem*{solution}{Решение}

\usepackage[top=20mm,bottom=20mm,left=20mm,right=20mm]{geometry}

\usepackage{lastpage} % how many pages

\usepackage{soul}

\usepackage{framed} % easy frames
\usepackage{enumerate} % better numbered lists

\usepackage{hyperref}
\usepackage{xcolor}

\usepackage{tikz} % drawing

\usepackage{csquotes}
\usepackage[style=numeric,backend=biber,sorting=nty]{biblatex}
\addbibresource{../../bibliography.bib}

\renewcommand{\epsilon}{\ensuremath{\varepsilon}}
\renewcommand{\phi}{\ensuremath{\varphi}}
\renewcommand{\theta}{\vartheta}
\renewcommand{\kappa}{\ensuremath{\varkappa}}
\renewcommand{\le}{\ensuremath{\leqslant}}
\renewcommand{\leq}{\ensuremath{\leqslant}}
\renewcommand{\ge}{\ensuremath{\geqslant}}
\renewcommand{\geq}{\ensuremath{\geqslant}}

\usepackage[useregional]{datetime2}

% custom maketitle
\usepackage{titling}
\setlength{\droptitle}{-4em}
\posttitle{\end{center}\vspace{-3em}}

\title{{\Large Теория фигур планет и гравиметрия 2018}\\ 
    {\bf\Large Практическое занятие № 1} \\
{\Large Введение. Краткие сведения из математики и высшей геодезии}}
\author{}
\DTMsavedate{lessondate}{2018-02-09}
\date{\DTMusedate{lessondate}}

\begin{document}
\maketitle

\section{Организационные вопросы}

Курс <<Теория фигур планет и гравиметрия>> читается на третьем (весенний семестр) и четвёртом
(осенний семестр) курсах для студентов геодезического факультета МИИГАиК специальности «Прикладная геодезия».

Весенний семестр посвящен изучению поля тяготения и его свойств, а также измерениям силы тяжести на
поверхности Земли. 
Осенний семестр будет включать в себя вопросы моделирования гравитационного поля планет и решение
геодезических задач с использованием информации о гравитационном поле Земли.

\subsection{Контакты}

Канал в Телеграм (для оповещений и анонсов):
\href{https://t.me/miigaik_tfpgcourse_2018}{miigaik\_tfpg\_course\_2018}

Материалы курса доступны в git-репозитории на GitHub:
\href{https://github.com/ioshchepkov/physical-geodesy-courses}{ioshchepkov/physical-geodesy-courses}\\

Репозиторий находится в стадии наполнения и будет обновляться по ходу курса.

\subsection{Программа практических занятий (весенний семестр)}
\begin{enumerate}
    \item Введение. Краткие сведения из математики и высшей геодезии.
    \item Притяжение. Основные понятия и свойства.
    \item Притяжение тел простой формы I.
    \item Притяжение тел простой формы II.
    \item Притяжение тел сложной формы. Гармонические функции.
    \item Гравитационное поле Земли и планет. Общая характеристика.
    \item Гравитационное поле Земли и планет. Изменение гравитационного поля во времени.
    \item Наземные методы и средства измерений. Абсолютные измерения силы тяжести.
    \item Статический метод измерения силы тяжести.
    \item Исследования статических гравиметров I.
    \item Исследования статических гравиметров II.
    \item Метрология. Сравнения и эталонирование гравиметров.
    \item Гравиметрический рейс.
    \item Обработка гравиметрического рейса. Гравиметрические сети.
\end{enumerate}

\subsection{Контроль знаний и выставление оценок}
В курсе (весенний семестр) предусмотрены следующие формы контроля знаний: 
\begin{itemize}
    \item лабораторные работы,
    \item домашние задания,
    \item самостоятельные работы,
    \item контрольные работы,
    \item зачёт.
\end{itemize}

На практических занятиях будут разбираться основные понятия для закрепления теоретического материала
лекций, а также будут решаться и разбираться простейшие и/или типовые примеры и задачи.
\textbf{Только} на
практических занятиях будут выполняться лабораторные работы с гравиметрами и разбираться отдельные
темы по разделу курса <<гравиметрия>>. Пропуски занятий с гравиметрами не допускаются, ибо в связи с
большим числом студентов и ограниченным числом преподавателей, у нас нет возможности заниматься с
вами вне аудиторных часов. Лабораторные работы с гравиметрами должны быть аккуратно оформлены и
защищены. Оцениваются работы по двоичной системе (зачёт/незачёт).

\subsubsection{Домашние задания}
Домашние задания (ДЗ) будут выдаваться после (почти) каждого практического занятия и будут
состоять из контрольных вопросов, обязательных типовых задач, а также дополнительных задач
повышенной сложности. Каждый вопрос и каждая задача в задании будут иметь свою «стоимость» в баллах.
Общая оценка за одно домашнее задание равна сумме баллов за все вопросы, примеры и задачи.
Максимальное число баллов за каждое домашнее задание --- $5,0$. Баллы, заработанные за решение задач
повышенной сложности могут быть зачтены в другие задания и формы контроля. Все домашние задания
должны быть защищены, что включает в себя несколько контрольных вопросов по теме и/или ходу решения.
Незащищённые задания не могут быть зачтены. Крайний срок сдачи --- две недели с момента выдачи
задания. После дедлайна домашние задания не принимаются и могут полностью войти в программу зачёта
для несдавшего студента. Задания можно высылать как в электронном (что крайне приветствуется) виде
вне занятий, так и сдавать их в рукописном виде в часы занятий. За использование системы
компьютерной вёрстки \LaTeX (читается как латех) при сдаче работ будет начисляться дополнительный балл.

\subsubsection{Самостоятельные работы}
Самостоятельные работы (СР) будут проводиться в течение 5 -- 8 минут в начале (почти) каждого
практического занятия по пройденному материалу, включая лекции. Они будут состоять из 2--3 вопросов
и/или простых задач. Максимальное число баллов за одну самостоятельную~---~$5,0$. 

\subsubsection{Контрольные работы}
В середине семестра будет проведена первая (КР№1) контрольная работа (КР), а в конце семестра
(скорее всего, на зачётном занятии) -- вторая (КР№2). Обе продолжительностью в один академический
час (половина пары). Программа
контрольных будет включать в себя теоретические и практические вопросы по пройденному материалу, в
том числе лекционному. Максимальное число баллов за контрольную --- $5,0$.\par
Сдача обеих контрольных работ на
положительную оценку ($> 3,0$) является условием допуска студента к зачёту.

\subsubsection{Зачёт}
При условии всех сданных и защищённых лабораторных работах, итоговая оценка за семестр выставляется
следующим образом. По всем видам контроля выводятся средние оценки, которые затем подставляются в
выражение

\begin{equation*}
    \text{О\_И = 0,5*О\_ДЗ + 0,2*О\_СР + 0,3*О\_КР},
\end{equation*}

где О\_И – итоговая оценка, О\_ДЗ – средняя оценка по домашним заданиям, О\_СР – средняя оценка по
самостоятельным работам, О\_КР – средняя оценка по двум контрольным.

Если итоговая оценка на конец семестра получается не менее 4,0 ($\text{О\_И} >= 4,0$), то студент получает
\textbf{зачёт автоматом}. 

К зачету студент собирает портфолио, то есть всё, что он сделал за семестр:
\begin{itemize}
    \item лабораторные работы;
    \item домашние задания;
    \item дополнительные задания, если имеются;
    \item конспекты практических занятий и лекций;
    \item контрольные работы, независимо от оценки.
\end{itemize}

Зачёт будет состоять из письменной и устной части. Письменно студент выполняет задания, по
которым за семестр он получил неудовлетворительную ($< 3,0$) оценку (прежде всего -- контрольные
работы). Устная часть включает вопросы по практическим и лекционным занятиям.

\subsubsection{Теоретический минимум}
Особое внимание необходимо обратить на вопросы теоретического минимума. Незнание уверенных
ответов на них автоматически влечёт за собой неудовлетворительную оценку ($0,0$) по всем видам
контроля знаний (домашние задания, самостоятельные и контрольные работы, зачёт). Обратное неверно,
знание ответов только на эти вопросы зачёта не гарантирует. Вопросы из этого списка будут
включаться в контроль по мере их появления в курсе.

\subsubsection{Бонусы}

При желании, студенту могут быть даны дополнительные <<творческие>> задания, которые могут быть
выполнены как индивидуально, так и коллективно (2--3 человека). За выполнение таких заданий будут
начисляться бонусные баллы. Штрафных санкций не предусмотрено.

\subsection{Литература}
\begin{refsection}
    \nocite{Shimbirev1975, Ogorodova2013, Yuzefovich2014}
    \printbibliography[title={\normalsize Рекомендуемая литература}]
\end{refsection}
\begin{refsection}
    \nocite{Yuzefovich1980, Torge1999, Ogorodova2006, Pellinen1978, Moritz2007, Moritz1983, Ogorodova2011}
    \printbibliography[title={\normalsize Дополнительная литература}]
\end{refsection}

\section{Предмет и задачи курса}

Название курса состоит из двух частей: <<теория фигур планет>> и <<гравиметрия>>. Под планетой мы,
конечно, будем в первую очередь иметь ввиду Землю. Однако рассматриваемые методы (а также и другие,
которые на Земле не используются) могут быть успешно
применены и применяются для исследования других планет, особенно твёрдых, а также их естественных спутников,
поэтому по ходу курса мы будем иногда обращать внимание и во внеземное пространство.

Вспомним, что основной научной задачей геодезии является определение фигуры и внешнего
гравитационного поля Земли и их изменений во времени. Теория фигуры Земли решает эту задачу с
использованием преимущественно гравиметрических данных (то есть по измерениям величин,
характеризующих гравитационное поле Земли). Синонимами
являются дисциплины <<физическая геодезия>> и <<геодезическая гравиметрия>>. Решением той же задачи,
но с использованием всей совокупности существующих исходных данных (например, спутниковых)
занимается теоретическая геодезия, которая преподается обычно на последних курсах геодезических
специальностей. 

Итак, теория фигуры Земли --- это наука, главной задачей которой является определение внешнего
гравитационного поля и фигуры Земли по гравиметрическим данным. Задача же получения этих данных с
необходимой плотностью и точностью стоит перед наукой, 
которая называется <<гравиметрия>> (или <<экспериментальная гравиметрия>>).

Определение внешнего гравитационного поля Земли в сущности является 
геофизической задачей также, как и изучение
магнитного и других физических полей. Однако, учитывая , что
элементы внешнего гравитационного поля Земли геодезисты,
как правило, определяют одновременно с параметрами фигуры
Земли из обработки одних и тех же данных, а в дальнейшем
совместно их используют, закономерно проблему определения
внешнего гравитационного поля Земли включать в формулировку 
основной научной задачи и геодезии\cite{Pellinen1978}.

Для того, что бы лучше понять роль гравитационного поля при решении повседневных геодезических
задач, ответьте на следующие вопросы:
\begin{enumerate}
    \item Назовите основные геометрические условия в нивелирах и угломерных приборах.
    \item Что происходит с геодезическими приборами, когда мы выставляем их по уровням? 
    \item В какой системе координат выполняются измерения на поверхности Земли?
    \item Чему равна сумма измеренных углов в треугольнике, если измерения считать безошибочными?
    \item Как расположена визирная ось поверенного и выставленного по уровням нивелира?
\end{enumerate}

Понятие фигуры планеты неоднозначно и может подразумевать под собой
\begin{itemize}
    \item геометрическую фигуру простой и правильной формы (шар, эллипсоид);
    \item фигуру гидростатического равновесия;
    \item фигуру конкретной эквипотенциальной (уровенной) поверхности (Земля --- геоид, Луна ---
        селеноид, Марс --- ареоид);
    \item фигуру её физической поверхности.
\end{itemize}
Исторически дисциплина развивалась точно также, от простого к сложному (см.
\cite{Ogorodova2013,Yuzefovich2014}). 

\begin{itemize}
    \item векторный анализ;
    \item теория потенциала;
    \item специальные функции;
    \item дифференциальные уравнения (обыкновенные и в частных производных);
    \item краевые задачи;
\end{itemize}

\section{Системы координат}
\subsection{Прямоугольная система координат}

\subsection{Сферическая и астрономическая системы координат}

\subsection{Эллипсоид. Геодезическая система коодинат}

\section{Математика}

\printbibliography
\end{document}
