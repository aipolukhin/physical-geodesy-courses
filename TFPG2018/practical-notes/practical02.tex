\documentclass[11pt, a4paper]{article}

% Languages and fonts
\usepackage{cmap} 
\usepackage[T2A]{fontenc}
\usepackage[utf8]{inputenc} 
\usepackage[english, russian]{babel}
\usepackage{microtype}
\usepackage{indentfirst}
\frenchspacing

% Mathematics
\usepackage{amsmath, amssymb, amsfonts, amsthm, mathtools, fixmath}
\mathtoolsset{showonlyrefs=true}
\usepackage{esint, esvect} % integrals and vectors
\usepackage{systeme} % equation system
\usepackage{commath} % partials and differentials
\usepackage{icomma} % smart comma ($0,2$ is a number)

% Floats
\usepackage{float}

% Tables
\usepackage{array,tabularx,tabulary,booktabs} % better tables
\usepackage{longtable}
\usepackage{multirow}

% Graphics
\usepackage[pdftex]{graphicx}
\usepackage{wrapfig}

% Theorems
\renewcommand{\proofname}{Доказательство}

\theoremstyle{plain}
\newtheorem{theorem}{Теорема}[section]

\theoremstyle{definition}
\newtheorem{definition}{Определение}
\newtheorem{corollary}{Следствие}[theorem]
\newtheorem{problem}{Задача}[section]

\theoremstyle{remark}
\newtheorem{remark}{Замечание}
\newtheorem*{solution}{Решение}

\usepackage[top=20mm,bottom=20mm,left=20mm,right=20mm]{geometry}

\usepackage{lastpage} % how many pages

\usepackage{soul}

\usepackage{framed} % easy frames
\usepackage{enumerate} % better numbered lists

\usepackage{hyperref}
\usepackage{xcolor}

\usepackage{tikz} % drawing

\usepackage{csquotes}
\usepackage[style=numeric,backend=biber,sorting=nty]{biblatex}
\addbibresource{../../bibliography.bib}

\renewcommand{\epsilon}{\ensuremath{\varepsilon}}
\renewcommand{\phi}{\ensuremath{\varphi}}
\renewcommand{\theta}{\vartheta}
\renewcommand{\kappa}{\ensuremath{\varkappa}}
\renewcommand{\le}{\ensuremath{\leqslant}}
\renewcommand{\leq}{\ensuremath{\leqslant}}
\renewcommand{\ge}{\ensuremath{\geqslant}}
\renewcommand{\geq}{\ensuremath{\geqslant}}

\usepackage[useregional]{datetime2}

% custom maketitle
\usepackage{titling}
\setlength{\droptitle}{-4em}
\posttitle{\end{center}\vspace{-3em}}

\title{{\Large Теория фигур планет и гравиметрия 2018}\\ 
    {\bf\Large Практическое занятие № 2} \\
{\Large Притяжение. Основные понятия и свойства}}
\author{}
\DTMsavedate{lessondate}{2018-02-16}
\date{\DTMusedate{lessondate}}

\begin{document}
\maketitle

\section{Закон всемирного тяготения Ньютона}
Закон гравитационного притяжения был сформулирован Ньютоном и опубликован в Математических началах
натуральной философии\cite{Newton1687}(лат. Philosophiae Naturalis Principia Mathematica) в 1687
году.
Звучит он следующим образом. Две частицы с массами $m_1$ и $m_2$ взаимно притягиваются с силой,
пропорциональной произведению их масс и обратно пропорционально расстоянию $r$ между ними, то есть
\begin{equation*}
    F_{12} = G\dfrac{m_1m_2}{r_{12}^2},
\end{equation*}
где $G$ (также обозначается как $f,\gamma$)--- гравитационная постоянная, 

В системе СИ рекомендованное Комитетом данных для науки и техники (CODATA) значение в 2014
году\cite{CODATA2014} такое
\begin{equation*}
    G = (6,67408 \pm 0,00031)\times10^{-11}\,\text{м}^3\text{кг}^{-1}\text{с}^{-2},
\end{equation*}
где $\pm0,00031\times10^{-11}\text{м}^3\text{кг}^{-1}\text{с}^2$ -- стандартная неопределенность
(погрешность), а относительная стандартная неопределенность гравитационной постоянной
равна $4,7\times10^{-5}$. Точность измерений гравитационной постоянной на несколько порядков ниже
точности измерений других физических величин,  часто новые определения $G$ не совпадают со старыми
на величины, превышающие стандартную неопределённость.

Пусть теперь точка $m_1 = m$ притягивает единичную массу $m_2 = 1\,\text{кг}$, тогда
\begin{equation}
    F = G\dfrac{m}{r^2},
    \label{eq:dynamic}
\end{equation}
--- сила притяжения, \textbf{численно} равная ускорению.
Действительно, из второго закона Ньютона $F_{12} = m_2 a$, тогда
\begin{equation*}
    m_2 a = G\dfrac{m_1m_2}{r_{12}^2},
\end{equation*}
откуда 
\begin{equation}
    a = G\dfrac{m_1}{r_{12}^2} = G\dfrac{m}{r^2}.
    \label{eq:kinematic}
\end{equation}
В дальнейшем, мы не будем различать эти две величины в силу эквивалентности
инерциальной и гравитационной массы.

Единица ускорения в системе СИ --- $\text{м/c}^2$. В геодезической гравиметрии, однако, мы будем
использовать внесистемную единицу --- Гал, названную в честь Галилео Галилея, который определяется
следующим образом:
\begin{align*}
    &1\,\text{Гал} = 10^{-2}\,\text{м/c}^2 = \text{см/c}^2\\
    &1\,\text{мГал} = 10^{-5}\,\text{м/c}^2 \\
    &1\,\text{мкГал} = 10^{-8}\,\text{м/c}^2. 
\end{align*}

\begin{problem}
    \textbf{Велика ли сила притяжения?}
\end{problem}

Произведение гравитационной постоянной $G$ на массу притягивающего объекта $M$ называется
гравитационным параметром. Для Земли существует свой термин --- геоцентрическая гравитационная
постоянная. Согласно рекомендациям Международной службы вращения Земли\cite{iers2010}, она равна
\begin{equation*}
    GM = 3,986004418\times10^{14}\,\text{м}^3\text{с}^{-2}
\end{equation*}
со стандартной неопределенностью $8\times10^{5}\,\text{м}^3\text{с}^{-2}$. Это означает, что
геоцентрическая гравитационная постоянная определена точнее, чем гравитационная постоянная. 
\begin{problem}
    Найдите массу Земли по известной гравитационной постоянной $G$ и геоцентрической гравитационной
    постоянной $GM$. Оцените точность.
\end{problem}
Земля не является замкнутой системой. Она постоянно теряет и набирает массу в силу различных природных
процессов и взаимодействия с космосом.
\begin{problem}
    Средний радиус Земли $R = 6371\,\text{км}$. Найдите притяжение, создаваемое Землёй
    на её поверхности, предполагая, что вся её масса сосредоточена в центре.
\end{problem}

\printbibliography
\end{document}
